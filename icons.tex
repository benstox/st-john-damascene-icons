\documentclass[10pt]{book}

\usepackage{fontspec}
\usepackage{fullpage} % to reduce the margins
\usepackage{titlesec}
\usepackage{xcolor}
\usepackage{paracol}
\usepackage{lettrine}

\setmainfont{Gentium Plus}
\usepackage[polutonikogreek,english]{babel}
\newcommand{\switchgreek}[1][]{\selectlanguage{polutonikogreek} \switchcolumn*[#1]}
\newcommand{\switchenglish}{\selectlanguage{english} \switchcolumn}

% RED
\definecolor{benred8}{HTML}{E82C00} 

% BLUE
\definecolor{benblue1}{HTML}{2B22C7}

% YELLOWS
\definecolor{benyellow1}{HTML}{FFD435}
\definecolor{benyellow2}{HTML}{7C6F3B}

% Luke Smith's title formatting
\titleformat{\section}
{\scshape\centering\Large}
{§\thesection}
{1em}
{}
\titleformat{\subsection}
{\scshape\centering\large}
{§\thesubsection}
{1em}
{}
\titleformat{\subsubsection}
{\itshape\centering\large}
{§\thesubsubsection}
{1em}
{}

\title{On Holy Images\\Λόγοι ἀπολογητικοὶ πρὸς τοὺς διαβάλλοντας τὰς ἁγίας εἰκόνας}
\author{St. John Damascene}

\begin{document}

\maketitle

% English: check for missing apostrophes, God s
% English: check for footnotes
% English: check for spaces before :;?! etc.
% English: St Peter -> St. Peter
% English: 2nd -> 2\textsuperscript{nd}
% Greek: capitalize works like theos, theotokos,
%        pater, pneuma, hios, meter, cheroubeim, etc.

\begin{paracol}{2}

\selectlanguage{polutonikogreek}

\section*{ΤΟΥ ΑΓΙΟΥ ΙΩΑΝΝΟΥ ΤΟΥ ΔΑΜΑΣΚΗΝΟΥ ΛΟΓΟΣ ΠΡΩΤΟΣ ΑΠΟΛΟΓΗΤΙΚΟΣ ΠΡΟΣ ΤΟΥΣ ΔΙΑΒΑΛΛΟΝΤΑΣ ΤΑΣ ΑΓΙΑΣ ΕΙΚΟΝΑΣ}

\switchenglish

\section*{First Apologia of St. John Damascene against those who decry Holy Images}

\switchgreek

\lettrine{Ἐ}{χρῆν} μὲν ἡμᾶς ἀεὶ τῆς ἑαυτῶν συναισθανομένους
ἀναξιότητος σιγὴν ἄγειν καὶ θεῷ τὴν τῶν ἡμαρτημένων ἡμῖν προσάγειν
ἐξομολόγησιν, ἀλλ’ ἐπειδὴ πάντα καλὰ ἐν καιρῷ αὐτῶν, ὁρῶ δὲ τὴν ἐκκλησίαν,
ἣν ὁ θεὸς ᾠκοδόμησεν ἐπὶ τῷ θεμελίῳ τῶν ἀποστόλων καὶ προφητῶν ὄντος
ἀκρογωνιαίου Χριστοῦ τοῦ υἱοῦ αὐτοῦ βαλλομένην ὥσπερ θαλαττίῳ κλύδωνι
κύμασιν ἀλλεπαλλήλοις κορυφουμένῳ, ἐξ ἐπαχθεστάτης φορᾶς τῶν πονηρῶν
πνευμάτων κυκωμένην τε καὶ ταραττομένην, καὶ τὸν χιτῶνα Χριστοῦ τὸν ἄνωθεν
ὑφαντὸν διαιρούμενον, ὃν ἀσεβῶν διελεῖν ηὐθαδιάσαντο παῖδες, καὶ τὸ σῶμα
αὐτοῦ εἰς διαφόρους δόξας κατατεμνόμενον, ὅ ἐστιν ὁ τοῦ θεοῦ λαὸς καὶ ἡ
τῆς ἐκκλησίας ἄνωθεν κεκρατηκυῖα παράδοσις, οὐκ εὔλογον ἡγησάμην σιγᾶν καὶ
δεσμὸν ἐπιθεῖναι τῇ γλώσσῃ τὴν ἠπειλημένην ἀπόφασιν ὑφορώμενος τὴν
φάσκουσαν· «Ἐὰν ὑποστείλῃ, οὐκ εὐδοκεῖ ἐν σοὶ ἡ ψυχή μου», καὶ «ἐὰν ἴδῃς
τὴν ῥομφαίαν ἐρχομένην καὶ μὴ ἀναγγείλῃς τῷ ἀδελφῷ σου, ἐκ σοῦ ἐκζητήσω τὸ
αἷμα αὐτοῦ.» Φόβῳ τοίνυν ἀφορήτῳ βαλλόμενος ἐπὶ τὸ λέγειν ἐλήλυθα οὐ
βασιλέων ὕψος πρὸ τῆς ἀληθείας τιθείς· «ἐλάλουν γάρ,» ἤκουσα τοῦ
θεοπάτορος λέγοντος Δαυίδ, «ἐναντίον βασιλέων καὶ οὐκ ᾐσχυνόμην», ἀλλὰ
μᾶλλον καὶ μᾶλλον τούτῳ πρὸς τὸ λέγειν νυττόμενος. Δεινὸν γὰρ βασιλέως
λόγος πρὸς ὑπαγωγὴν τῶν ὑπηκόων· ὀλίγοι γάρ, ὅσοι τῶν ἀνέκαθεν τῶν
βασιλικῶν κατωλιγώρησαν θεσπισμάτων, ὅσοι τὸν ἐπὶ γῆς βασιλέα
βασιλευόμενον οἴδασιν ἄνωθεν, καὶ ὡς κρατοῦσιν οἱ νόμοι τῶν βασιλέων.

\switchenglish

\lettrine{W}{ith} the ever-present conviction of my own
unworthiness, I ought to have kept silence and
confessed my shortcomings before God, but all
things are good at the right time. I see the
Church which God founded on the Apostles
and Prophets, its corner-stone being Christ His
Son, tossed on an angry sea, beaten by rushing
waves, shaken and troubled by the assaults of
evil spirits. I see rents in the seamless robe of
Christ, which impious men have sought to part
asunder, and His body cut into pieces, that is,
the word of God and the ancient tradition of
the Church. Therefore I have judged it un
reasonable to keep silence and to hold my
tongue, bearing in mind the Scripture warning:\textemdash
``If thou withdrawest thyself, my soul 
shall not delight in thee,'' and ``If thou seest 
the sword coming and dost not warn thy 
brother, I shall require his blood at thy hand.'' 
Fear, then, compelled me to speak ; the truth 
was stronger than the majesty of kings.
``I bore testimony to Thee before kings,''
I heard  the royal\footnote{θεοπάτωρ, not easily rendered in English.}
David saying, ``and I was not ashamed.''
No, I was the more incited to 
speak. The King's command is all powerful 
over his subjects. For few men have hitherto 
been found who, whilst recognising the power 
of the earthly king to come from above, have 
resisted his unlawful demands. 

\switchgreek

Πρῶτον μὲν οὖν ἁπάντων οἷόν τινα τρόπιν ἢ θεμέλιον τῷ λογισμῷ καταπήξας τὴν
τῆς ἐκκλησιαστικῆς θεσμοθεσίας συντήρησιν, δι’ ἧς ἡ σωτηρία προσγίνεσθαι
πέφυκε, τοῦ λόγου τὴν βαλβίδα ἠνέῳξα καὶ τοῦτον ὥσπερ ἵππον εὐχάλινον τῆς
ἀφετηρίας παρώρμησα.  Δεινὸν γὰρ ὄντως ᾠήθην καὶ πέρα δεινῶν τοσούτοις τὴν
ἐκκλησίαν ἀμαρύσσουσαν προτερήμασι καὶ ταῖς τῶν εὐσεβεστάτων ἀνδρῶν ἄνωθεν
παραδόσεσιν ὡραϊσθεῖσαν παλινοστεῖν ἐπὶ τὰ πτωχὰ στοιχεῖα, φοβουμένην φόβον,
οὗ οὐκ ἔστι φόβος, καὶ ὥσπερ οὐκ ἐγνωκυῖαν τὸν ὄντως θεὸν ὑφορᾶσθαι τὸν εἰς
εἰδωλολατρείαν ὄλισθον καὶ κἂν γοῦν ἐν σμικροτάτῳ τῆς τελειότητος λείπεσθαι,
ὥσπερ τινὰ στιγμὴν ἐπίμωμον ἐν μέσῳ προσώπου λίαν ὡραϊσμένου φέρουσαν, τῷ
ἀπόσῳ τοῦ παρ εγγράμματος τοῦ κάλλους τὸ πᾶν λυμαινομένην· οὐ γὰρ μικρὸν τὸ
μικρόν, ὅταν εἰς μέγα ἐκφέρῃ, ὅπου γε οὐδὲ σμικρὸν τὸ παρεγχάραγμα ἄνωθεν
κεκρατηκυῖαν ἐκκλησίας ἀνατραπῆναι παράδοσιν, οἷα κατεγνωσμένων τῶν
προκαθηγησαμένων ἡμᾶς, ὧν ἐχρῆν ἀναθεωροῦντας τὴν ἀναστροφὴν μιμεῖσθαι τὴν
πίστιν.

\switchenglish

In the first place, grasping as a kind of
pillar, or foundation, the teaching of the Church,
which is our salvation, I have opened out its
meaning, giving, as it were, the reins to a well-
caparisoned charger. For I look upon it as a
great calamity that the Church, adorned with
her great privileges and the holiest examples of
saints in the past, should go back to the first
rudiments, and fear where there is no fear. It
is disastrous to suppose that the Church does
not know God as He is, that she degenerates
into idolatry, for if she declines from perfection
in a single iota, it is as an enduring mark on a
comely face, destroying by its unsightliness the
beauty of the whole. A small thing is not
small when it leads to something great, nor
indeed is it a thing of no matter to give up the
ancient tradition of the Church held by our
forefathers, whose conduct we should observe,
and whose faith we should imitate.

\switchgreek

Ἐκλιπαρῶ τοίνυν πρῶτον μὲν τὸν παντοκράτορα κύριον, ᾧ γυμνὰ πάντα καὶ τετραχηλισμένα, πρὸς ὃν ἡμῖν ὁ λόγος, εἰδότα τῆς ταπεινῆς μου γνώμης ἐν τούτῳ τὸ ἀκραιφνὲς καὶ τοῦ σκοποῦ τὸ εἰλικρινές, δοῦναί μοι λόγον ἐν ἀνοίξει τοῦ στόματός μου καὶ τοῦ νοῦ τὰς ἡνίας οἰκείαις χερσὶν ἀναδέξασθαι καὶ τοῦτον πρὸς ἑαυτὸν ἐπισπάσασθαι, πρὸς ἐνώπιόν τε καὶ εὐθεῖαν τρίβον τὴν ῥύμην ποιούμενον μὴ ἐγκλίνοντα πρὸς τὰ δοκοῦντα δεξιὰ ἢ ἀναφανδὸν ἀριστερὰ γνωριζόμενα, –μεθ’ ὃν ἅπαντα τὸν τοῦ θεοῦ λαόν, τὸ ἔθνος τὸ ἅγιον, τὸ βασίλειον ἱεράτευμα, σὺν τῷ καλῷ ποιμένι τῆς λογικῆς Χριστοῦ ποίμνης, τῷ τὴν Χριστοῦ ἱεραρχίαν ἐν ἑαυτῷ ὑπογράφοντι, δέξασθαί μου τὸν λόγον μετ’ εὐμενείας, μὴ τῷ ἐλαχίστῳ τῆς ἀξίας προσέχοντας ἢ λόγων ἐπιζητοῦντας στροφάς, ἐπεὶ τούτων οὐ παντελῶς, ἴδρις ὁ πένης ἐγώ, ἀλλὰ τῆς τῶν νοημάτων φροντίσαι δυνάμεως («οὐ γὰρ ἐν λόγῳ ἡ βασιλεία τῶν οὐρανῶν, ἀλλ’ ἐν δυνάμει»)· οὐ γὰρ νικῆσαι σκοπός, ἀλλὰ τῇ ἀληθείᾳ πολεμουμένῃ χεῖρα ὀρέξαι, τῆς προαιρέσεως ὀρεγούσης χεῖρα δυνάμεως.
Ἀρωγὸν τοίνυν τὴν ἐνυπόστατον ἐπικεκλημένος ἀλήθειαν ἐντεῦθεν τοῦ λόγου τὰς ἀρχὰς ποιήσομαι.

\switchenglish

In the first place, then, before speaking to
you, I beseech Almighty God, to whom all
things lie open, who knows my small capacity
and my genuine intention, to bless the words
of my mouth, and to enable me to bridle my
mind and direct it to Him, to walk in His
presence straightly, not declining to a plausible
right hand, nor knowing the left. Then I ask
all God's people, the chosen ones of His royal
priesthood, with the holy shepherd of Christ's
orthodox flock, who represents in his own
person Christ's priesthood, to receive my
treatise with kindness. They must not dwell
on my unworthiness, nor seek for eloquence,
for I am only too conscious of my shortcom
ings. They must consider the thoughts them
selves. The kingdom of heaven is not in word
but in deed. Conquest is not my object. I
raise a hand which is fighting for the truth\textemdash a
willing hand under the divine guidance. Relying, then, upon substantial truth as my auxiliary,
I will enter on my subject matter.

\switchgreek

Οἶδα τὸν ἀψευδῶς εἰπόντα· «Κύριος ὁ θεός σου κύριος εἷς ἐστι», καὶ «κύριον τὸν θεόν σου προσκυνήσεις καὶ αὐτῷ μόνῳ λατρεύσεις», καὶ «οὐκ ἔσονταί σοι θεοὶ ἕτεροι», καὶ «οὐ ποιήσεις γλυπτὸν πᾶν ὁμοίωμα, ὅσα ἐν τῷ οὐρανῷ ἄνω καὶ ὅσα ἐν τῇ γῇ κάτω», καὶ «αἰσχυνθήτωσαν πάντες οἱ προσκυνοῦντες τοῖς γλυπτοῖς», καὶ «θεοί, οἳ τὸν οὐρανὸν καὶ τὴν γῆν οὐκ ἐ ποίησαν, ἀπολέσθωσαν», καὶ ὅσα τοιουτοτρόπως «πάλαι ὁ θεὸς λαλήσας τοῖς πατράσιν ἐν τοῖς προφήταις ἐπ' ἐσχάτων τῶν ἡμερῶν ἐλάλησεν ἡμῖν ἐν τῷ μονογενεῖ αὐτοῦ υἱῷ, δι' οὗ καὶ τοὺς αἰῶνας ἐποίησεν.»
Οἶδα τὸν εἰπόντα· «Αὕτη δέ ἐστιν ἡ αἰώνιος ζωή, ἵνα γινώσκωσί σε, τὸν μόνον ἀληθινὸν θεὸν καὶ ὃν ἀπέστειλας Ἰησοῦν Χριστόν.»
Πιστεύω εἰς ἕνα θεόν, μίαν τῶν πάντων ἀρχήν, ἄναρχον, ἄκτιστον, ἀνώλεθρον καὶ ἀθάνατον, αἰώνιον καὶ ἀίδιον, ἀκατάληπτον, ἀσώματον, ἀόρατον, ἀπερίγραπτον, ἀσχημάτιστον, μίαν ὑπερούσιον οὐσίαν, ὑπέρθεον θεότητα, ἐν τρισὶν ὑποστάσεσι, πατρὶ καὶ υἱῷ καὶ ἁγίῳ πνεύματι, καὶ τούτῳ μόνῳ λατρεύω καὶ τούτῳ μόνῳ προσάγω τὴν τῆς λατρείας προσκύνησιν.
Ἑνὶ θεῷ προσκυνῶ, μιᾷ θεότητι, ἀλλὰ καὶ τριάδι λατρεύω ὑποστάσεων, θεῷ πατρὶ καὶ θεῷ υἱῷ σεσαρκωμένῳ καὶ θεῷ ἁγίῳ πνεύματι, ἑνὶ θεῷ.
Οὐ προσκυνῶ τῇ κτίσει παρὰ τὸν κτίσαντα, ἀλλὰ προσκυνῶ τὸν κτίστην κτισθέντα τὸ κατ' ἐμὲ καὶ εἰς κτίσιν ἀταπεινώτως καὶ ἀκαθαιρέτως κατεληλυθότα, ἵνα τὴν ἐμὴν δοξάσῃ φύσιν καὶ θείας κοινωνὸν ἐπεργάσηται φύσεως.
Συμπροσκυνῶ τῷ βασιλεῖ καὶ θεῷ τὴν ἁλουργίδα τοῦ σώματος οὐχ ὡς ἱμάτιον οὐδ' ὡς τέταρτον πρόσωπον (ἄπαγε) ἀλλ' ὡς ὁμόθεον χρηματίσασαν καὶ γενομένην, ὅπερ τὸ χρῖσαν, ἀμεταβλήτως· οὐ γὰρ θεότης ἡ φύσις γέγονε τῆς σαρκός, ἀλλ' ὥσπερ ὁ λόγος σὰρξ ἀτρέπτως γέγονε μείνας, ὅπερ ἦν, οὕτω καὶ ἡ σὰρξ λόγος γέγονεν οὐκ ἀπολέσασα τουθ', ὅπερ ἐστί, ταυτι ζομένη δὲ μᾶλλον πρὸς τὸν λόγον καθ' ὑπόστασιν.
Διὸ θαρρῶν εἰκονίζω θεὸν τὸν ἀόρατον οὐχ ὡς ἀόρατον, ἀλλ' ὡς ὁρατὸν δι' ἡμᾶς γενόμενον μεθέξει σαρκός τε καὶ αἵματος.
Οὐ τὴν ἀόρατον εἰκονίζω θεότητα, ἀλλ' εἰκονίζω θεοῦ τὴν ὁραθεῖσαν σάρκα. Εἰ γὰρ ψυχὴν εἰκονίσαι ἀμήχανον, πόσῳ μᾶλλον θεὸν τὸν καὶ τῇ ψυχῇ δόντα τὸ ἄυλον;

\switchenglish

I have taken heed to the words of Truth 
Himself:\textemdash ``The Lord thy God is one.'' And 
``Thou shalt fear the Lord thy God, and shalt 
serve Him only, and thou shalt not have 
strange gods.'' Again, ``Thou shalt not make 
to thyself a graven thing, nor the likeness of 
anything that is in heaven above, or in the 
earth beneath''; and ``Let them be all con 
founded that adore graven things.'' Again, 
``The gods that have not made heaven and 
earth, let them perish.'' In this way God spoke 
of old to the patriarchs through the prophets, 
and lastly, through His only-begotten Son, on 
whose account He made the ages. He says, 
This is eternal life, that they may know Thee, 
the only true God, and Jesus Christ whom 
Thou didst send. I believe in one God, the 
source of all things, without beginning, un 
created, immortal, everlasting, incomprehen 
sible, bodiless, invisible, uncircumscribed,\footnote{ἀπερίγραπτος, i.e., not in place} with 
out form. I believe in one supersubstantial 
being, one divine Godhead in three entities, 
the Father, the Son, and the Holy Ghost, 
and I adore Him alone with the worship 
of latreia. I adore one God, one Godhead 
but three Persons, God the Father, God the 
Son made flesh, and God the Holy Ghost, one 
God. I do not adore creation more than the 
Creator, but I adore the creature created as I 
am, adopting creation freely and spontaneously 
that He might elevate our nature and make us 
partakers of His divine nature. Together with 
my Lord and King I worship Him clothed in 
the flesh, not as if it were a garment or He 
constituted a fourth person of the Trinity\textemdash 
God forbid. That flesh is divine, and endures 
after its assumption. Human nature was not 
lost in the Godhead, but just as the Word 
made flesh remained the Word, so flesh became 
the Word remaining flesh, becoming, rather, 
one with the Word through union. Therefore I venture to draw an image 
of the invisible God, not as invisible, but as 
having become visible for our sakes through 
flesh and blood. I do not draw an image of 
the immortal Godhead. I paint the visible 
flesh of God, for it is impossible to represent 
a spirit, how much more God who gives 
breath to the spirit.

\switchgreek

Ἀλλά φασιν· Εἶπεν ὁ θεὸς διὰ Μωσέως τοῦ νομοθέτου· «Κύριον τὸν θεόν σου προσκυνήσεις καὶ αὐτῷ μόνῳ λατρεύσεις», καὶ «οὐ ποιήσεις πᾶν ὁμοίωμα, ὅσα ἐν τῷ οὐρανῷ καὶ ὅσα ἐν τῇ γῇ.»

\switchenglish

Now adversaries say: God's commands to 
Moses the law-giver were, ``Thou shalt adore 
the Lord thy God, and thou shalt worship him 
alone, and thou shalt not make to thyself a 
graven thing that is in heaven above, or in the 
earth beneath.'' 

\switchgreek

Ἀδελφοί, ὄντως πλανῶνται οἱ μὴ εἰδότες τὰς γραφὰς οἱ μὴ εἰδότες, ὡς «τὸ γράμμα ἀποκτένει, τὸ δὲ πνεῦμα ζωοποιεῖ», οἱ μὴ ἐρευνῶντες τὸ ὑπὸ τῷ γράμματι κεκρυμμένον πνεῦμα.
Πρὸς οὓς ἂν ἀξίως εἴποιμι· Ὁ τοῦτο διδά ξας ὑμᾶς διδαξάτω καὶ τὸ ἑπόμενον.
Μάθε, ὅπως ἑρμηνεύει ὁ νομοθέτης ὧδέ πως ἐν τῷ Δευτερονομίῳ λέγων· «Καὶ ἐλάλησε κύριος πρὸς ὑμᾶς ἐκ μέσου τοῦ πυρός· φωνὴν ῥημάτων ὑμεῖς ἠκούσατε καὶ ὁμοίωμα οὐκ εἴδετε, ἀλλ’ ἢ φωνήν», καὶ μετ’ ὀλίγα «καὶ φυλάξασθε σφόδρα τὰς ψυχὰς ὑμῶν, ὅτι ὁμοίωμα οὐκ εἴδετε ἐν τῇ ἡμέρᾳ, ᾗ ἐλάλησε κύριος πρὸς ὑμᾶς ἐν Χωρὴβ ἐν τῷ ὄρει ἐκ μέσου τοῦ πυρός, μήποτε ἀνομήσητε καὶ ποιήσητε ὑμῖν ἑαυτοῖς γλυπτὸν ὁμοίωμα, πᾶσαν εἰκόνα, ὁμοίωμα ἀρσενικοῦ ἢ θηλυκοῦ, ὁμοίωμα παντὸς κτήνους τῶν ὄντων ἐπὶ τῆς γῆς, ὁμοίωμα παντὸς ὀρνέου πτερωτοῦ» καὶ τὰ ἑξῆς, καὶ μετὰ βραχέα «μήποτε ἀναβλέψας εἰς τὸν οὐρανὸν καὶ ἰδὼν τὸν ἥλιον καὶ τὴν σελήνην καὶ τοὺς ἀστέρας καὶ πάντα τὸν κόσμον τοῦ οὐρανοῦ, πλανηθεὶς προσκυνήσῃς αὐτοῖς καὶ λατρεύσῃς αὐτοῖς.»

\switchenglish

They err truly, not knowing the Scriptures, 
for the letter kills whilst the spirit quickens\textemdash
not finding in the letter the hidden meaning. 
I could say to these people, with justice, He 
who taught you this would teach you the 
following. Listen to the law-giver's interpretation in Deuteronomy: ''And the Lord spoke 
to you from the midst of the fire. You heard 
the voice of His words, but you saw not any 
form at all.'' And shortly afterwards: ``Keep 
your souls carefully. You saw not any similitude in the day that the Lord God spoke to 
you in Horeb from the midst of the fire, lest 
perhaps being deceived you might make you a 
graven similitude, or image of male and female, 
the similitude of any beasts that are upon the 
earth, or of birds that fly under heaven.'' And 
again, ``Lest, perhaps, lifting up thy eyes to 
heaven, thou see the sun and the moon, and 
all the stars of heaven, and being deceived by 
error thou adore and serve them.'' 

\switchgreek

Ὁρᾷς, ὡς εἷς ἐστιν ὁ σκοπός, ὥστε μὴ λατρεῦσαι τῇ κτίσει παρὰ τὸν κτίσαντα μηδὲ προσάγειν τὴν τῆς λατρείας προσκύνησιν, ἀλλ’ ἢ μόνῳ τῷ δημιουργῷ.
Διὸ πανταχῆ συνάπτει τῇ προσκυνήσει τὴν λατρείαν· πάλιν γάρ φησιν· «Οὐκ ἔσονταί σοι θεοὶ ἕτεροι πλὴν ἐμοῦ.
Οὐ ποιήσεις σεαυτῷ γλυπτὸν οὐδὲ πᾶν ὁμοίωμα, οὐ προσκυνήσεις αὐτοῖς οὐδ’ οὐ μὴ λα τρεύσῃς αὐτοῖς, ὅτι ἐγώ εἰμι κύριος ὁ θεὸς ὑμῶν», καὶ πάλιν «τοὺς βωμοὺς αὐτῶν καθελεῖτε καὶ τὰς στήλας αὐτῶν συντρίψετε καὶ τὰ ἄλση αὐτῶν ἐκκόψετε καὶ τὰ γλυπτὰ τῶν θεῶν αὐτῶν κατακαύσετε πυρί· οὐ γὰρ μὴ προσκυνήσητε θεῷ ἑτέρῳ», καὶ μετ’ ὀλίγα «καὶ θεοὺς χωνευτοὺς οὐ ποιήσεις σεαυτῷ.»
καὶ πάλιν «καὶ θεοὺς χωνευτοὺς οὐ ποιήσεις σεαυτῷ.»

\switchenglish

You see the one thing to be aimed at is not 
to adore a created thing more than the Creator, 
nor to give the worship of latreia except to 
Him alone. By worship, consequently, He 
always understands the worship of latreia. 
For, again, He says: ``Thou shalt not have 
strange gods other than Me. Thou shalt not 
make to thyself a graven thing, nor any 
similitude. Thou shalt not adore them, and 
thou shalt not serve them, for I am the Lord 
thy God.'' And again, ``Overthrow their altars, 
and break down their statues ; burn their groves 
with fire, and break their idols in pieces. For 
thou shalt not adore a strange god.'' And 
a little further on: ``Thou shalt not make to 
thyself gods of metal.''

\switchgreek

Ὁρᾷς, ὡς τῆς εἰδωλολατρείας ἕνεκα ἀπαγορεύει τὴν εἰκονογραφίαν καὶ ὅτι ἀδύνατον εἰκονίζεσθαι θεὸν τὸν ἄποσον καὶ ἀπερίγραπτον καὶ ἀόρατον. «Οὐ γὰρ εἶδος αὐτοῦ», φησίν, «ἑωράκατε», καθὰ καὶ Παῦλος ἑστὼς ἐν μέσῳ τοῦ Ἀρείου πάγου φησίν· «Γένος οὖν ὑπάρχοντες τοῦ θεοῦ οὐκ ὀφείλομεν νομίζειν χρυσίῳ ἢ ἀργυρίῳ ἢ λίθῳ, χαράγματι τέχνης καὶ ἐνθυμήσεως ἀνθρώπου, τὸ θεῖον εἶναι ὅμοιον.»

\switchenglish

You see that He forbids image-making on 
account of idolatry, and that it is impossible to 
make an image of the immeasurable, un- 
circumscribed, invisible God. You have not 
seen the likeness of Him, the Scripture says, 
and this was St. Paul's testimony as he stood in 
the midst of the Areopagus: ``Being, therefore, 
the offspring of God, we must not suppose the 
divinity to be like unto gold, or silver, or stone, 
the graving of art, and device of man.''

\switchgreek

Ἰουδαίοις μὲν οὖν διὰ τὸ πρὸς εἰδωλολατρείαν εὐόλισθον ταῦτα νενομοθέτητο· ἡμεῖς δέ, θεολογικῶς εἰπεῖν, οἷς ἐδόθη φυγοῦσι τὴν δεισιδαίμονα πλάνην καθαρῶς μετὰ τοῦ θεοῦ γενέσθαι, ἐπεγνωκόσι τὴν ἀλήθειαν καὶ θεῷ μόνῳ λατρεύειν καὶ τῆς θεογνωσίας καταπλουτῆσαι τὴν τελειότητα καὶ εἰς ἄνδρα καταντῆσαι τέλειον παρελθοῦσι τὴν νηπιότητα, οὐκέτι ὑπὸ παιδαγωγόν ἐσμεν, λαβόντες τὴν διακριτικὴν ἕξιν παρὰ θεοῦ καὶ εἰδότες, τί τὸ εἰκονιζόμενον καὶ τί τὸ μὴ εἰκόνι περιγραφόμενον.
«Οὐ γὰρ εἶδος αὐτοῦ», φησίν, «ἑωράκατε.» Βαβαὶ τῆς σοφίας τοῦ νομοθέτου.
Πῶς εἰκονισθήσεται τὸ ἀόρατον; Πῶς εἰκασθήσεται τὸ ἀνείκαστον; Πῶς γραφήσεται τὸ ἄποσον καὶ ἀμέγεθες καὶ ἀόριστον; Πῶς ποιωθήσεται τὸ ἀνείδεον; Πῶς χρωματουργηθήσεται τὸ ἀσώματον; Τί οὖν τὸ αἰνιγματικῶς μηνυόμενον; Δῆλον ὡς, ὅταν ἴδῃς διὰ σὲ γενόμενον ἄνθρωπον τὸν ἀσώματον, τότε δράσεις τῆς ἀνθρωπίνης μορφῆς τὸ ἐκτύπωμα· ὅταν ὁρατὸς σαρκὶ ὁ ἀόρατος γένηται, τότε εἰκονίσεις τὸ τοῦ ὁραθέντος ὁμοίωμα· ὅτε ὁ ἀσώματος καὶ ἀσχημάτιστος ἄποσός τε καὶ ἀπήλικος καὶ ἀμεγέθης ὑπεροχῇ τῆς ἑαυτοῦ φύσεως ὁ ἐν μορφῇ θεοῦ ὑπάρχων μορφὴν δούλου λαβὼν ταύτῃ συσταλῇ πρὸς ποσότητά τε καὶ πηλικότητα καὶ χαρακτῆρα περίθηται σώματος, τότε ἐν πίναξι χάραττε καὶ ἀνατίθει πρὸς θεωρίαν τὸν ὁραθῆναι καταδεξάμενον.
Χάραττε τούτου τὴν ἄφατον συγκατάβασιν, τὴν ἐκ παρθένου γέννησιν, τὴν ἐν Ἰορδάνῃ βάπτισιν, τὴν ἐν Θαβὼρ μεταμόρφωσιν, τὰ πάθη τὰ τῆς ἀπαθείας πρόξενα, τὰ θαύματα, τὰ τῆς θείας αὐτοῦ σύμβολα φύσεως δι’ ἐνεργείας σαρκὸς ἐνεργείᾳ θείᾳ πραττόμενα, τὸν σταυρὸν τὸν σωτήριον, τὴν ταφήν, τὴν ἀνάστασιν, τὴν εἰς οὐρανοὺς ἄνοδον.
Πάντα γράφε καὶ λόγῳ καὶ χρώμασι.
Μὴ φοβοῦ, μὴ δέδιθι· οἶδα διαφορὰν προσκυνήσεων.
Προσεκύνησέ ποτε Ἀβραὰμ τοῖς υἱοῖς Ἐμμώρ, ὅτε τὸ σπήλαιον τὸ διπλοῦν εἰς τάφου κλῆρον ὠνήσατο, ἀνδράσιν ἀσεβέσι καὶ ἀγνωσίαν νοσοῦσι θεοῦ.
Προσεκύνησεν Ἰακὼβ Ἠσαῦ τῷ ἀδελφῷ καὶ Φαραὼ ἀνδρὶ Αἰγυπτίῳ, ἀλλὰ μὴν καὶ ἐπὶ τὸ ἄκρον τῆς ῥάβδου προσεκύνησε· προσεκύνησαν μέν, ἀλλ’ οὐκ ἐλάτρευσαν.
Προσεκύνησαν Ἰησοῦς ὁ τοῦ Ναυῆ καὶ Δανιὴλ ἀγγέλῳ θεοῦ, ἀλλ’ οὐκ ἐλάτρευσαν.
Ἕτερον γάρ ἐστιν ἡ τῆς λατρείας προσκύνησις καὶ ἕτερον ἡ ἐκ τιμῆς προσαγομένη τοῖς κατά τι ἀξίωμα ὑπερέχουσιν.

\switchenglish

These injunctions were given to the Jews on 
account of their proneness to idolatry. Now 
we, on the contrary, are no longer in leading 
strings. Speaking theologically, it is given to 
us to avoid superstitious error, to be with God 
in the knowledge of the truth, to worship God 
alone, to enjoy the fulness of His knowledge. 
We have passed the stage of infancy, and 
reached the perfection of manhood. We receive 
our habit of mind from God, and know what 
may be imaged and what may not. The 
Scripture says, ``You have not seen the likeness 
of Him.'' What wisdom in the law-giver. How 
depict the invisible? How picture the in 
conceivable? How give expression to the 
limitless, the immeasurable, the invisible? 
How give a form to immensity? How paint 
immortality? How localise mystery? It is 
clear that when you contemplate God, who is 
a pure spirit, becoming man for your sake, 
you will be able to clothe Him with the human 
form. When the Invisible One becomes visible 
to flesh, you may then draw a likeness of His 
form. When He who is a pure spirit, without 
form or limit, immeasurable in the boundless 
ness of His own nature, existing as God, takes 
upon Himself the form of a servant in substance 
and in stature, and a body of flesh, then you 
may draw His likeness, and show it to anyone 
willing to contemplate it. Depict His ineffable 
condescension, His virginal birth, His baptism 
in the Jordan, His transfiguration on Thabor, 
His all-powerful sufferings, His death and 
miracles, the proofs of His Godhead, the deeds 
which He worked in the flesh through divine 
power, His saving Cross, His Sepulchre, and 
resurrection, and ascent into heaven. Give to 
it all the endurance of engraving and colour. 
Have no fear or anxiety ; worship is not all of 
the same kind. Abraham worshipped the sons 
of Emmor, impious men in ignorance of God, 
when he bought the double cave for a tomb. 
Jacob worshipped his brother Esau and Pharao, 
the Egyptian, but on the point of his staff.
He worshipped, he did not adore. Josue and 
Daniel worshipped an angel of God ; they did 
not adore him. The worship of latreia is one 
thing, and the worship which is given to merit 
another.

\switchgreek

Ἀλλ’ ἐπειδὴ περὶ εἰκόνος ὁ λόγος καὶ προσκυνήσεως, φέρε τὸν περὶ τούτων λόγον διευκρινήσωμεν. Εἰκὼν μὲν οὖν ἐστιν ὁμοίωμα χαρακτηρίζον τὸ πρωτότυπον μετὰ τοῦ καί τινα διαφορὰν ἔχειν πρὸς αὐτό· οὐ γὰρ κατὰ πάντα ἡ εἰκὼν ὁμοιοῦται πρὸς τὸ ἀρχέτυπον. Εἰκὼν τοίνυν ζῶσα, φυσικὴ καὶ ἀπαράλλακτος τοῦ ἀοράτου θεοῦ ὁ υἱὸς ὅλον ἐν ἑαυτῷ φέρων τὸν πατέρα, κατὰ πάντα ἔχων τὴν πρὸς αὐτὸν ταυτότητα, μόνῳ δὲ διαφέρων τῷ αἰτιατῷ. Αἴτιον μὲν γὰρ φυσικὸν ὁ πατήρ, αἰτιατὸν δὲ ὁ υἱός· οὐ γὰρ πατὴρ ἐξ υἱοῦ, ἀλλὰ υἱὸς ἐκ πατρός. Ἐξ αὐτοῦ γάρ, εἰ καὶ μὴ μετ’ αὐτὸν ἔχει τὸ εἶναι, ὅπερ ἐστὶν ὁ γεννήσας πατήρ.

\switchenglish

Now, as we are talking of images 
and worship, let us analyse the exact meaning 
of each. An image is a likeness of the original 
with a certain difference, for it is not an exact 
reproduction of the original. Thus, the Son is 
the living, substantial, unchangeable Image of 
the invisible God, bearing in Himself the whole 
Father, being in all things equal to Him, differ 
ing only in being begotten by the Father, who 
is the Begetter; the Son is begotten. The 
Father does not proceed from the Son, but the 
Son from the Father. It is through the Son, 
though not after Him, that He is what He 
is, the Father who generates.

\switchgreek

Εἰσὶ δὲ καὶ ἐν τῷ θεῷ εἰκόνες καὶ παραδείγματα τῶν ὑπ’ αὐτοῦ ἐσομένων, τουτέστιν ἡ βουλὴ αὐτοῦ ἡ προαιώνιος καὶ ἀεὶ ὡσαύτως ἔχουσα.
Ἄτρεπτον γὰρ τὸ θεῖον κατὰ πάντα, καὶ οὐκ ἔστιν ἐν αὐτῷ μεταβολὴ ἢ τροπῆς ἀποσκίασμα.
Ταύτας τὰς εἰκόνας καὶ τὰ παραδείγματα προορισμούς φησιν ὁ ἅγιος Διονύσιος ὁ πολὺς τὰ θεῖα καὶ μετὰ θεοῦ τὰ περὶ θεοῦ διασκεψάμενος.
Ἐν γὰρ τῇ βουλῇ αὐτοῦ ἐχαρακτηρίζετο καὶ εἰκονίζετο πάντα τὰ ὑπ’ αὐτοῦ προωρισμένα καὶ ἀπαραβάτως ἐσόμενα πρὶν γενέσεως αὐτῶν, ὥσπερ, εἴ τις βούλοιτο οἰκοδομῆσαι οἶκον, ἀνατυποῖ καὶ εἰκονίζει πρῶτον τὸ σχῆμα κατὰ διάνοιαν.

\switchenglish

In God, too, 
there are representations and images of His 
future acts,\textemdash that is to say, His counsel from 
all eternity, which is ever unchangeable. That 
which is divine is immutable; there is no 
change in Him, nor shadow of change. 
Blessed Denis (the Carthusian) who has made 
divine things in God's presence his study, says 
that these representations and images are 
marked out beforehand. In His counsels, God 
has noted and settled all that He would do, the 
unchanging future events before they came to 
pass. In the same way, a man who wished to 
build a house, would first make and think out 
a plan.

\switchgreek

Εἶτα πάλιν εἰκόνες εἰσὶ τῶν ἀοράτων καὶ ἀτυπώτων, σωματικῶς τυπουμένων πρὸς ἀμυδρὰν κατανόησιν. Καὶ γὰρ ἡ θεία γραφὴ τύπους θεῷ καὶ ἀγγέλοις περιτίθησι καὶ τὴν αἰτίαν διδάσκων ὁ αὐτὸς θεῖος ἀνήρ φησιν, ὅτι μὲν γὰρ εἰκότως προ βέβληνται τῶν ἀτυπώτων οἱ τύποι καὶ τὰ σχήματα τῶν ἀσχηματίστων, οὐ μόνην αἰτίαν φαίη τις εἶναι τὴν καθ’ ἡμᾶς ἀναλογίαν ἀδυνατοῦσαν ἀμέσως ἐπὶ τὰς νοητὰς ἀνατείνεσθαι θεωρίας καὶ δεομένην οἰκείων καὶ συμφυῶν ἀναγωγῶν. Εἰ τοίνυν τῆς ἡμῶν προνοῶν ἀναλογίας ὁ θεῖος λόγος πάντοθεν τὸ ἀνατατικὸν ἡμῖν ποριζόμενος καὶ τοῖς ἁπλοῖς καὶ ἀτυπώτοις τύπους τινὰς περιτίθησι, πῶς μὴ εἰκονίσει τὰ σχήμασι μεμορφωμένα κατὰ τὴν οἰκείαν φύσιν καὶ ποθούμενα μέν, διὰ δὲ τὸ μὴ παρεῖναι ὁρᾶσθαι μὴ δυνάμενα; Διὰ γὰρ τῆς αἰσθήσεως φαντασία τις συνίσταται ἐν τῇ ἐμπροσθίῳ κοιλίᾳ τοῦ ἐγκεφάλου καὶ οὕτω τῷ κριτικῷ παραπέμπεται καὶ τῇ μνήμῃ ἐνθησαυρίζεται. Φησὶ γοῦν καὶ ὁ θεορρήμων Γρηγόριος, ὅτι πολλὰ κάμνων ὁ νοῦς ἐκβῆναι τὰ σωματικὰ πάντη ἀδυνατεῖ· ἀλλὰ καὶ «τὰ ἀόρατα τοῦ θεοῦ ἀπὸ κτίσεως κόσμου τοῖς ποιήμασι νοούμενα καθορᾶται.» Ὁρῶμεν γὰρ εἰκόνας ἐν τοῖς κτίσμασι μηνυούσας ἡμῖν ἀμυδρῶς τὰς θείας ἐμφάσεις· ὡς ὅτε λέγομεν τὴν ἁγίαν τριάδα, τὴν ὑπεράρχιον, εἰκονίζεσθαι δι’ ἡλίου καὶ φωτὸς καὶ ἀκτῖνος· ἢ πηγῆς ἀναβλυζούσης καὶ πηγαζομένου νάματος καὶ προχοῆς· ἢ νοῦ καὶ λόγου καὶ πνεύματος τοῦ καθ’ ἡμᾶς· ἢ ῥόδου φυτοῦ καὶ ἄνθους καὶ εὐωδίας.

\switchenglish

Again, visible things are images of 
invisible and intangible things, on which they 
throw a faint light. Holy Scripture clothes in 
figure God and the angels, and the same holy 
man (Blessed Denis) explains why. When 
sensible things sufficiently render what is 
beyond sense, and give a form to what is 
intangible, a medium would be reckoned 
imperfect according to our standard, if it did 
not fully represent material vision, or if it 
required effort of mind. If, therefore, Holy 
Scripture, providing for our need, ever putting 
before us what is intangible, clothes it in flesh, 
does it not make an image of what is thus 
invested with our nature, and brought to the 
level of our desires, yet invisible ? A certain 
conception through the senses thus takes place 
in the brain, which was not there before, and is 
transmitted to the judicial faculty, and added to 
the mental store. Gregory, who is so eloquent 
about God, says that the mind which is set 
upon getting beyond corporeal things, is in 
capable of doing it. For the invisible things of 
God since the creation of the world are made 
visible through images. We see images in 
creation which remind us faintly of God, as 
when, for instance, we speak of the holy and 
adorable Trinity, imaged by the sun, or light, 
or burning rays, or by a running fountain, or a 
full river, or by the mind, speech, or the spirit 
within us, or by a rose tree, or a sprouting 
flower, or a sweet fragrance. 

\switchgreek

Πάλιν εἰκὼν λέγεται τῶν ἐσομένων αἰνιγματωδῶς σκιαγραφοῦσα τὰ μέλλοντα, ὡς ἡ κιβωτὸς τὴν ἁγίαν παρθένον καὶ θεοτόκον καὶ ἡ ῥάβδος καὶ ἡ στάμνος, καὶ ὡς ὁ ὄφις τὸν τὸ δῆγμα διὰ σταυροῦ καταργήσαντα τοῦ ἀρχεκάκου ὄφεως, ἥ τε θάλασσα, τὸ ὕδωρ καὶ ἡ νεφέλη τὸ τοῦ βαπτίσματος πνεῦμα.

\switchenglish

Again, an image is expressive of something 
in the future, mystically shadowing forth what 
is to happen. For instance, the ark represents 
the image of Our Lady, Mother of God, so 
does the staff and the earthen jar. The serpent 
brings before us Him who vanquished on the 
Cross the bite of the original serpent; the sea, 
water, and the cloud the grace of baptism. 

\switchgreek

Πάλιν εἰκὼν λέγεται τῶν γεγονότων ἢ κατά τινος θαύματος μνήμην ἢ τιμῆς ἢ αἰσχύνης ἢ ἀρετῆς ἢ κακίας πρὸς τὴν εἰς ὕστερον τῶν θεωμένων ὠφέλειαν, ὡς ἂν τὰ μὲν κακὰ φεύγωμεν, τὰς δὲ ἀρετὰς ζηλώσωμεν. Διπλῆ δὲ αὕτη διά τε λόγου ταῖς βίβλοις ἐγγραφομένου, ὡς ὁ θεὸς τὸν νόμον ταῖς πλαξὶν ἐνεκόλαψε καὶ τοὺς τῶν θεοφιλῶν ἀνδρῶν βίους ἀναγράπτους γενέσθαι προσέταξε, καὶ διὰ θεωρίας αἰσθητῆς, ὡς τὴν στάμνον καὶ τὴν ῥάβδον ἐν τῇ κιβωτῷ τεθῆναι προσέταξεν εἰς μνημόσυνον. Οὕτω καὶ νῦν τὰς εἰκόνας τῶν γεγονότων καὶ τὰς ἀρετὰς διαγράφομεν. Ἢ τοίνυν πᾶσαν εἰκόνα ἄνελε καὶ ἀντινομοθέτει τῷ ταύτας προστάξαντι γενέσθαι ἢ ἑκάστην δέχου κατὰ τὸν ἑκάστῃ πρέποντα λόγον καὶ τρόπον. Εἰπόντες τοίνυν τοὺς τῆς εἰκόνος τρόπους εἴπωμεν καὶ περὶ προσκυνήσεως.

\switchenglish

Again, things which have taken place are 
expressed by images for the remembrance 
either of a wonder, or an honour, or dishonour, 
or good or evil, to help those who look upon 
it in after times that we may avoid evils and 
imitate goodness. It is of two kinds, the 
written image in books, as when God had the 
law inscribed on tablets, and when He enjoined 
that the lives of holy men should be recorded 
and sensible memorials be preserved in remembrance; as, for instance, the earthen jar 
and the staff in the ark. So now we preserve 
in writing the images and the good deeds of 
the past. Either, therefore, take away images 
altogether and be out of harmony with God 
who made these regulations, or receive them 
with the language and in the manner which 
befits them. In speaking of the manner let us 
go into the question of worship. 

\switchgreek

Ἡ προσκύνησις ὑποπτώσεως καὶ τιμῆς ἐστι σύμβολον. Καὶ ταύτης δὲ διαφόρους ἔγνωμεν τρόπους· Πρώτην τὴν κατὰ λατρείαν, ἣν προσάγομεν μόνῳ τῷ φύσει προσκυνητῷ θεῷ. Ἔπειτα τὴν διὰ τὸν φύσει προσκυνητὸν θεὸν προσαγομένην τοῖς αὐτοῦ φίλοις τε καὶ θεράπουσιν, ὡς τῷ ἀγγέλῳ Ἰησοῦς ὁ τοῦ Ναυῆ καὶ Δανιὴλ προσεκύνησαν, ἢ τοῖς θεοῦ τόποις, ὥς φησιν ὁ Δαυίδ· «Προσκυνήσωμεν εἰς τὸν τόπον, οὗ ἔστησαν οἱ πόδες αὐτοῦ», ἢ τοῖς αὐτοῦ ἀναθήμασιν, ὡς ἅπας Ἰσραὴλ τῇ σκηνῇ προσεκύνει καὶ τῷ ἐν Ἱερουσαλὴμ ναῷ κύκλῳ ἑστῶτες καὶ πρὸς αὐτὸν ἁπανταχόθεν προσκυνοῦντες εἰσέτι καὶ νῦν, ἢ τοῖς ὑπ’ αὐτοῦ χειροτονηθεῖσιν ἄρχουσιν, ὡς Ἰακὼβ τῷ τε Ἠσαῦ ὡς προγενεστέρῳ ἀδελφῷ ὑπὸ θεοῦ γενομένῳ καὶ Φαραὼ ὑπὸ θεοῦ χειροτονηθέντι ἄρχοντι καὶ τῷ Ἰωσὴφ οἱ αὐτοῦ ἀδελφοί. Οἶδα καὶ κατὰ τιμὴν τὴν πρὸς ἀλλήλους προσαγομένην προσκύνησιν ὡς Ἀβραὰμ τοῖς υἱοῖς Ἐμμώρ. Ἢ τοίνυν πᾶσαν προσκύνησιν ἄνελε ἢ πάσας δέχου μετὰ τοῦ ὀφείλοντος λόγου καὶ τρόπου.

\switchenglish

Worship is the symbol of veneration and of 
honour. Let us understand that there are 
different degrees of worship. First of all the 
worship of latreia, which we show to God, who 
alone by nature is worthy of worship. Then, 
for the sake of God who is worshipful by 
nature, we honour His saints and servants, as 
Josue and Daniel worshipped an angel, and 
David His holy places, when he says, ``Let 
us go to the place where His feet have stood.'' 
Again, in His tabernacles, as when all the 
people of Israel adored in the tent, and stand 
ing round the temple in Jerusalem, fixing their 
gaze upon it from all sides, and worshipping 
from that day to this, or in the rulers estab 
lished by Him, as Jacob rendered homage to 
Esau, his elder brother, and to Pharao, the 
divinely established ruler. Joseph was worshipped by his brothers. I am aware that 
worship was based on honour, as in the case 
of Abraham and the sons of Emmor. Either, 
then, do away with worship, or receive it alto 
gether according to its proper measure. 

\switchgreek

Λέγε μοι ἐρωτῶντι· Εἷς θεὸς ὁ θεός; Ναί, φήσεις, ὡς ἔμοιγε δοκεῖ, εἷς νομοθέτης.
Τί οὖν νομοθετεῖ τὰ ἐναντία; Οὐ γὰρ ἔξω τῆς κτίσεως τὰ χερουβίμ.
Τί οὖν προστάττει χερουβὶμ γλυπτὰ ἀνθρώπων χερσὶν τεκταινόμενα σκιάζειν τὸ ἱλαστήριον; Ἢ δῆλον, ὡς θεοῦ μὲν ὡς ἀπεριγράπτου καὶ ἀνεικάστου ποιεῖν εἰκόνα ἀμήχανον ἤ τινος ὡς θεοῦ, ἵνα μὴ ὡς θεὸς λατρευομένη προσκυνῆται ἡ κτίσις.
Τῶν δὲ χερουβὶμ ὡς περιγραπτῶν καὶ τῷ θείῳ θρόνῳ δουλοπρεπῶς παρεστώτων τὴν εἰκόνα προστάττει ποιεῖν δουλοπρεπῶς σκιάζουσαν τὸ ἱλαστήριον· ἔπρεπε γὰρ τῇ εἰκόνι τῶν οὐρανίων λειτουργῶν τὴν εἰκόνα τῶν θείων μυστηρίων σκιάζεσθαι.
Τί δὲ φὴς τὴν κιβωτόν, τὴν στάμνον, τὸ ἱλαστήριον; Οὐ χειρότευκτα; Οὐκ ἔργα χειρῶν ἀνθρώπων; Οὐκ ἐξ ἀτίμου, ὡς σὺ φής, ὕλης κατεσκευασμένα; Τί δὲ ἡ σκηνὴ ἅπασα; Οὐχὶ εἰκὼν ἦν; Οὐ σκιὰ καὶ ὑπόδειγμα; Φησὶ τοιγαροῦν ὁ θεῖος ἀπόστολος περὶ τῶν κατὰ τὸν νόμον ἱερέων διεξιών· «Οἵτινες ὑποδείγματι καὶ σκιᾷ λατρεύουσι τῶν ἐπουρανίων, καθὼς κεχρημάτισται Μωσῆς μέλλων ἐπιτελεῖν τὴν σκηνήν.
Ὅρα γάρ φησι, ποιήσεις πάντα κατὰ τὸν τύπον τὸν δειχθέντα σοι ἐν τῷ ὄρει.» Ἀλλ’ οὐδὲ εἰκὼν ἦν ὁ νόμος, ἀλλ’ εἰκόνος προσκίασμα· φησὶ γοῦν ὁ αὐτὸς ἀπόστολος· «Σκιὰν γὰρ ἔχων ὁ νόμος τῶν μελλόντων ἀγαθῶν, οὐκ αὐτὴν τὴν εἰκόνα τῶν πραγμάτων.» Εἰ οὖν ὁ νόμος εἰκόνας ἀπαγορεύει, αὐτὸς δὲ εἰκόνος ἐστὶ προχάραγμα, τί φήσομεν; Εἰ ἡ σκηνὴ σκιὰ καὶ τύπου τύπος, πῶς μὴ εἰκονογραφεῖν ὁ νόμος διακελεύεται; Ἀλλ’ οὐκ ἔστιν οὕτω ταῦτα, οὐκ ἔστι· «καιρὸς» δὲ μᾶλλον «τῷ παντὶ πράγματι.»

\switchenglish


Answer me this question. Is there only one 
God ? You answer, ``Yes, there is only one 
Law-giver.'' Why, then, does He command 
contrary things? The cherubim are not out 
side of creation; why, then, does He allow 
cherubim carved by the hand of man to over 
shadow the mercy-seat? Is it not evident that 
as it is impossible to make an image of God, 
who is uncircumscribed and impassible, or of 
one like to God, creation should not be 
worshipped as God. He allows the image of 
the cherubim who are circumscribed,\footnote{A reference to the question treated by St. Thomas after St. John Damascene: \emph{utrum angelus sit in loco.}} and 
prostrate in adoration before the divine throne, 
to be made, and thus prostrate to overshadow 
the mercy-seat. It was fitting that the image 
of the heavenly choirs should overshadow the 
divine mysteries. Would you say that the ark 
and staff and mercy-seat were not made? Are 
they not produced by the hand of man? Are 
they not due to what you call contemptible 
matter? What was the tabernacle itself? 
Was it not an image ? Was it not a type 
and a figure? Hence the holy Apostle s words 
concerning the observances of the law, ``Who 
serve unto the example and shadow of heavenly 
things.'' As it was answered to Moses, when 
he was to finish the tabernacle: ``See'' (He 
says), ``that thou make all things according to 
the pattern which was shown thee on the 
Mount.'' But the law was not an image. It 
shrouded the image. In the words of the same 
Apostle, the law contains the shadow of the 
goods to come, not the image of those things. 
For if the law should forbid images, and yet be 
itself a forerunner of images, what should we 
say? If the tabernacle was a figure, and the 
type of a type, why does the law not prohibit 
image-making? But this is not in the least 
the case. There is a time for everything. 

\switchgreek

Πάλαι μὲν ὁ θεὸς ὁ ἀσώματός τε καὶ ἀσχημάτιστος οὐδαμῶς εἰκονίζετο, νῦν δὲ
σαρκὶ ὀφθέντος θεοῦ καὶ τοῖς ἀνθρώποις συναναστραφέντος εἰκονίζω θεοῦ τὸ
ὁρώμενον.  Οὐ προσκυνῶ τῇ ὕλῃ, προσκυνῶ δὲ τὸν τῆς ὕλης δημιουργόν, τὸν ὕλην
δι’ ἐμὲ γενόμενον καὶ ἐν ὕλῃ κατοικῆσαι καταδεξάμενον καὶ δι’ ὕλης τὴν
σωτηρίαν μου ἐργασάμενον, καὶ σέβων οὐ παύσομαι τὴν ὕλην, δι’ ἧς ἡ σωτηρία μου
εἴργασται.  Σέβω δὲ οὐχ ὡς θεόν–ἄπαγε· πῶς γὰρ τὸ ἐξ οὐκ ὄντων τὴν γένεσιν
ἐσχηκὸς θεός; – εἰ καὶ τὸ τοῦ θεοῦ σῶμα θεὸς διὰ τὴν καθ’ ὑπόστασιν ἕνωσιν
γεγονὸς ἀμεταβλήτως, ὅπερ τὸ χρῖσαν, καὶ μεῖναν, ὅπερ ἦν τῇ φύσει, σὰρξ
ἐψυχωμένη ψυχῇ λογικῇ τε καὶ νοερᾷ, ἠργμένη, οὐκ ἄκτιστος.  Τὴν δέ γε λοιπὴν
ὕλην σέβω καὶ δι’ αἰδοῦς ἄγω, δι’ ἧς ἡ σωτηρία μου γέγονεν, ὡς θείας ἐνεργείας
καὶ χάριτος ἔμπλεων.  Ἢ οὐχ ὕλη τὸ τοῦ σταυροῦ ξύλον τὸ τρισόλβιον καὶ
τρισμακάριστον; Ἢ οὐχ ὕλη τὸ ὄρος τὸ σεπτὸν καὶ ἅγιον, ὁ τοῦ κρανίου τόπος; Ἢ
οὐχ ὕλη ἡ φερέσβιος πέτρα καὶ ζωηφόρος, ὁ τάφος ὁ ἅγιος, ἡ πηγὴ τῆς ἡμῶν
ἀναστάσεως; Ἢ οὐχ ὕλη τὸ μέλαν καὶ ἡ τῶν εὐαγγελίων παναγία βίβλος; Ἢ οὐχ ὕλη
ἡ ζωηφόρος τράπεζα ἡ τὸν ἄρτον ἡμῖν τῆς ζωῆς χορηγοῦσα; Ἢ οὐχ ὕλη ὁ χρυσός τε
καὶ ὁ ἄργυρος, ἐξ ὧν σταυροὶ καὶ πίνακες καὶ κρατῆρες κατασκευάζονται; Ἢ οὐχ
ὕλη πρὸ τούτων ἁπάντων τὸ τοῦ κυρίου μου σῶμα καὶ αἷμα; Ἢ πάντων τούτων ἄνελε
τὸ σέβας καὶ τὴν προσκύνησιν ἢ παραχώρει τῇ ἐκκλησιαστικῇ παραδόσει καὶ τὴν
τῶν εἰκόνων προσκύνησιν θεοῦ καὶ φίλων θεοῦ ὀνόματι ἁγιαζομένων καὶ διὰ τοῦτο
θείου πνεύματος ἐπισκιαζομένων χάριτι. Μὴ κάκιζε τὴν ὕλην· οὐ γὰρ ἄτιμος.
Οὐδὲν γὰρ ἄτιμον, ὃ παρὰ θεοῦ γεγένηται· τῶν Μανιχαίων τοῦτο τὸ φρόνημα.
Μόνον δὲ ἄτιμον, ὃ μὴ τὴν αἰτίαν ἔσχεν ἐκ θεοῦ, ἀλλ’ ἡμέτερόν ἐστιν εὕρεμα τῇ
ἐκ τοῦ κατὰ φύσιν εἰς τὸ παρὰ φύσιν αὐτεξουσίῳ ἐκκλίσει τε καὶ ῥοπῇ τοῦ
θελήματος, τουτέστιν ἡ ἁμαρτία.  Εἰ διὰ τὸν νόμον τὰς εἰκόνας ἀτιμάζεις καὶ
ἀπαγορεύεις ὡς ἐξ ὕλης κατ εσκευασμένας, ὅρα, τί φησιν ἡ γραφή· «Καὶ ἐλάλησε
κύριος πρὸς Μωσῆν λέγων· Ἰδοὺ ἀνακέκληκα τὸ ὄνομα Βεσελεὴλ τὸν τοῦ Ὀρεὶ τὸν
τοῦ Ὣρ ἐκ φυλῆς Ἰούδα.  Καὶ ἐνέπλησα αὐτὸν πνεῦμα θεῖον σοφίας καὶ συνέσεως
καὶ ἐπιστήμης ἐν παντὶ ἔργῳ διανοεῖσθαι καὶ ἀρχιτεκτονεῖν καὶ ἐργάζεσθαι
χρυσίον καὶ ἀργύριον καὶ τὸν χαλκὸν καὶ τὴν ὑάκινθον καὶ τὴν πορφύραν καὶ τὸ
κόκκινον νηστὸν καὶ τὴν βύσσον τὴν κεκλωσμένην καὶ τὰ λιθουργικὰ εἰς τὰ ἔργα
καὶ τεκτονικῆς εἰς τὰ ξύλα, ἐργάζεσθαι κατὰ πάντα τὰ ἔργα· καὶ ἐγὼ δέδωκα
αὐτὸν καὶ τὸν Ἐλιὰβ τὸν τοῦ Ἀχισαμὰχ ἐκ φυλῆς Δάν· καὶ παντὶ συνετῷ καρδίᾳ ἐγὼ
δέδωκα σύνεσιν, καὶ ποιήσουσι πάντα, ὅσα σοι συνέταξα.» καὶ πάλιν «εἶπε Μωσῆς
πρὸς πᾶσαν συναγωγὴν υἱῶν Ἰσραήλ· Ἠκούσατε τοῦτο τὸ ῥῆμα, ὃ συνέταξε κύριος
λέγων· Λάβετε παρ’ ὑμῶν αὐτῶν ἀφαίρεμα τῷ κυρίῳ.  Πᾶς καταδεχόμενος τὴν
καρδίαν, καὶ οἴσουσι τὰς ἀπαρχὰς τῷ κυρίῳ, χρυσίον, ἀργύριον, χαλκόν,
ὑάκινθον, πορφύραν, κόκκινον διπλοῦν διανενησμένον καὶ βύσσον κεκλωσμένην καὶ
τρίχας αἰγείας καὶ δέρματα κριῶν ἠρυθροδανωμένα καὶ δέρματα ὑακίνθινα καὶ ξύλα
ἄσηπτα καὶ ἔλαιον τῆς χρίσεως καὶ τὸ θυμίαμα τῆς συνθέσεως καὶ λίθους σαρδίους
καὶ λίθους εἰς τὴν γλυφὴν καὶ εἰς τὴν ἐπωμίδα καὶ τὸν ποδήρη.  Καὶ πᾶς σοφὸς
καρδίᾳ ἐν ὑμῖν ἐλθὼν ἐργαζέσθω πάντα, ὅσα συνέταξε κύριος, τὴν σκηνήν», καὶ τὰ
ἑξῆς, καὶ μεθ’ ἕτερα «καὶ ἐποίησεν ἀμφοτέρους τοὺς λίθους τῆς σμαράγδου
συμπεπορπημένους καὶ περισεσιαλωμένους χρυσίῳ καὶ γεγλυμμένους ἐκκόλαμμα
σφραγῖδος», καὶ πάλιν «οἱ λίθοι ἦσαν ἐξ ὀνομάτων τῶν υἱῶν Ἰσραὴλ δώδεκα ἐκ τῶν
ὀνομάτων αὐτῶν ἐγγεγλυμμέναι σφραγῖδες, ἕκαστος ἐκ τοῦ αὐτοῦ ὀνόματος εἰς τὰς
δώδεκα φυλάς», καὶ αὖθις «καὶ ἐποίησαν τὸ καταπέτασμα τῆς σκηνῆς τοῦ μαρτυρίου
ἐξ ὑακίνθου καὶ πορφύρας καὶ κοκκίνου νενησμένου καὶ βύσσου κεκλωσμένης, ἔργον
ὑφαντὸν χερουβίμ», καὶ πάλιν «καὶ ἐποίησαν τὸ ἱλαστήριον ἄνωθεν τῆς κιβωτοῦ ἐκ
χρυσίου καθαροῦ καὶ τοὺς δύο χερουβίμ.» Ἰδοὺ δὴ ὕλη τιμία καὶ καθ’ ὑμᾶς
ἄτιμος.  Τί γὰρ τριχῶν αἰγείων εὐτελέστερον καὶ χρωμάτων; Καὶ οὐ χρώματα τὸ
κόκκινον καὶ ἡ πορφύρα καὶ ἡ ὑάκινθος; Ἰδοὺ δὴ καὶ ὁμοίωμα χερουβίμ.  Πῶς οὖν
διὰ τὸν νόμον ἀπαγορεύειν φής, ὃ ποιεῖν ὁ νόμος προσέταξεν; Εἰ διὰ τὸν νόμον
τὰς εἰκόνας ἀπαγορεύεις, ὥρα σοι καὶ σαββατίζειν καὶ περιτέμνεσθαι· ταῦτα γὰρ
ἀπαραχωρήτως ὁ νόμος κελεύει.  Ἀλλ’ ἴστε, «ὡς, ἐὰν τὸν νόμον τηρῆτε, Χριστὸς
ὑμᾶς οὐδὲν ὠφελήσει. Οἵτινες ἐν νόμῳ δικαιοῦσθε, τῆς χάριτος ἐξεπέσετε.» Οὐχ
ἑώρα θεὸν ὁ Ἰσραὴλ ὁ πάλαι, «ἡμεῖς δὲ ἀνακεκαλυμμένῳ προσώπῳ τὴν δόξαν κυρίου
κατοπτριζόμεθα.»

\switchenglish

Of old, God the incorporeal and uncircumscribed was never depicted.
Now, however, when God is seen clothed in flesh,
and conversing with men, I make an image of the 
God whom I see. I do not worship matter, I 
worship the God of matter, who became 
matter for my sake, and deigned to inhabit 
matter, who worked out my salvation through 
matter. I will not cease from honouring that 
matter which works my salvation. I venerate 
it, though not as God. How could God be 
born out of lifeless things? And if God s body 
is God by union, it is immutable. 
The nature of God remains the same as before, 
the flesh created in time is quickened by a 
logical and reasoning soul. I honour all matter 
besides, and venerate it. Through it, filled, as 
it were, with a divine power and grace, my 
salvation has come to me. Was not the thrice 
happy and thrice blessed wood of the Cross 
matter? Was not the sacred and holy mountain 
of Calvary matter? What of the life-giving 
rock, the Holy Sepulchre, the source of our 
resurrection: was it not matter ? Is not the 
most holy book of the Gospels matter? Is not 
the blessed table matter which gives us the 
Bread of Life ? Are not the gold and silver 
matter, out of which crosses and altar-plate and 
chalices are made? And before all these 
things, is not the body and blood of our Lord 
matter? Either do away with the veneration 
and worship due to all these things, or submit 
to the tradition of the Church in the worship of 
images, honouring God and His friends, and 
following in this the grace of the Holy Spirit. 
Do not despise matter, for it is not despicable. 
Nothing is that which God has made. This is 
the Manichean heresy. That alone is despic 
able which does not come from God, but is 
our own invention, the spontaneous choice of 
will to disregard the natural law,\textemdash that is to 
say, sin. If, therefore, you dishonour and give 
up images, because they are produced by 
matter, consider what the Scripture says : And 
the Lord spoke to Moses, saying, ``Behold 
I have called by name Beseleel, the son of Uri, 
the son of Hur, of the tribe of Juda. And I have 
filled him with the spirit of God, with wisdom 
and understanding, and knowledge in all 
manner of work. To devise whatsoever may 
be artificially made of gold, and silver, and 
brass, of marble and precious stones, and 
variety of wood. And I have given him for 
his companion, Ooliab, the son of Achisamech, 
of the tribe of Dan. And I have put wisdom 
in the heart of every skilful man, that they may 
make all things which I have commanded thee.'' 
And again : ``Moses said to all the assembly of 
the children of Israel : This is the word the 
Lord hath commanded, saying : Set aside with 
you first fruits to the Lord. Let every one 
that is willing and hath a ready heart, offer 
them to the Lord, gold, and silver, and brass, 
violet, and purple, and scarlet twice dyed, and 
fine linen, goat s hair, and ram s skins died red 
and violet, coloured skins, selim-wood, and oil 
to maintain lights and to make ointment, and 
most sweet incense, onyx stones, and precious 
stones for the adorning of the ephod and the 
rational. Whosoever of you is wise, let him 
come, and make that which the Lord hath 
commanded.'' See you here the glorification 
of matter which you make inglorious. What 
is more insignificant than goat's hair or colours? 
Are not scarlet and purple and hyacinth colours? 
Now, consider the handiwork of man becoming 
the likeness of the cherubim. How, then, can 
you make the law a pretence for giving up 
what it orders? If you invoke it against 
images, you should keep the Sabbath, and 
practise circumcision. It is certain that ``if 
you observe the law, Christ will not profit 
you. You who are justified in the law, you 
are fallen from grace.'' Israel of old did not see 
God, but we see the Lord's glory face to face. 

\switchgreek

Καὶ αἰσθητῶς τὸν αὐτοῦ χαρακτῆρα τοῦ σαρκωθέντος φημὶ θεοῦ λόγου προτίθεμεν ἁπανταχῆ καὶ τὴν πρώτην ἁγιαζόμεθα τῶν αἰσθήσεων (πρώτη γὰρ αἰσθήσεων ὅρασις) ὥσπερ καὶ τοῖς λόγοις τὴν ἀκοήν· ὑπόμνημα γάρ ἐστιν ἡ εἰκών. Καὶ ὅπερ τοῖς γράμματα μεμυημένοις ἡ βίβλος, τοῦτο τοῖς ἀγραμμάτοις ἡ εἰκών· καὶ ὅπερ τῇ ἀκοῇ ὁ λόγος, τοῦτο τῇ ὁράσει ἡ εἰκών· νοητῶς δὲ αὐτῷ ἑνούμεθα. Διὰ τοῦτο προσέταξεν ὁ θεὸς γενέσθαι κιβωτὸν ἐκ ξύλων ἀσήπτων καὶ καταχρυσῶσαι ταύτην ἔσωθέν τε καὶ ἔξωθεν καὶ ἐνθεῖναι τὰς πλάκας, τὴν ῥάβδον, τὴν στάμνον τὴν χρυσῆν ἔχουσαν τὸ μάννα πρὸς ὑπόμνησιν τῶν γεγονότων καὶ προτύπωσιν τῶν μελλόντων. Καὶ τίς οὐκ ἐρεῖ ταύτας εἰκόνας διαπρυσίους τε κήρυκας; Καὶ ταῦτα οὐκ ἐκ πλαγίων ἔκειτο τῆς σκηνῆς, ἀλλὰ κατενώπιον παντὸς τοῦ λαοῦ, πρὸς ἃ βλέποντες τῷ δι’ αὐτῶν ἐνεργήσαντι θεῷ προσέφερον τὴν προσκύνησιν καὶ τὴν λατρείαν. Δῆλον οὐχ ὡς αὐτοῖς λατρεύοντες, ἀλλὰ δι’ αὐτῶν εἰς ὑπόμνησιν τῶν θαυμάτων ἀγόμενοι καὶ τῷ τερατουργῷ θεῷ τὴν προσκύνησιν νέμοντες. Εἰκόνες γὰρ ἦσαν πρὸς ὑπόμνησιν κείμεναι, τιμώμεναι οὐχ ὡς θεοί, ἀλλ’ ὡς θείας ἐνεργείας ὑπόμνησιν ἄγουσαι.

\switchenglish

We proclaim Him also by our senses on all 
sides, and we sanctify the noblest sense, which 
is that of sight. The image is a memorial, just 
what words are to a listening ear. What a 
book is to the literate, that an image is to the 
illiterate. The image speaks to the sight as 
words to the ear ; it brings us understanding. 
Hence God ordered the ark to be made of 
imperishable wood, and to be gilded outside 
and in, and the tablets to be put in it, and the 
staff and the golden urn containing the manna, 
for a remembrance of the past and a type of the 
future. Who can say these were not images 
and far-sounding heralds ? And they did not 
hang on the walls of the tabernacle ; but in 
sight of all the people who looked towards 
them, they were brought forward for the 
worship and adoration of God, who made 
use of them. It is evident that they were not 
worshipped for themselves, but that the people 
were led through them to remember past signs, 
and to worship the God of wonders. They 
were images to serve as recollections, not divine, 
but leading to divine things by divine power. 

\switchgreek

Καὶ δώδεκα λίθους προσέταξεν ὁ θεὸς ληφθῆναι ἐκ τοῦ Ἰορδάνου καὶ τὴν αἰτίαν προστίθησι· φησὶ γάρ· «Ἵνα, ὅταν ἐρωτᾷ σε ὁ υἱός σου, τί εἰσιν οἱ λίθοι οὗτοι, διηγῇ, πῶς ἐξέλιπε τὸ ὕδωρ τοῦ Ἰορδάνου θείᾳ προστάξει καὶ διέβη ἡ κιβωτὸς κυρίου καὶ πᾶς ὁ λαός.» Πῶς οὖν ἡμεῖς οὐκ εἰκονογραφήσομεν τὰ σωτήρια Χριστοῦ τοῦ θεοῦ ἡμῶν πάθη καὶ θαύματα, ἵν’, ὅταν ἐρωτᾷ με ὁ υἱός μου, τί τοῦτό ἐστιν, ἐρῶ, ὅτι ὁ θεὸς λόγος ἄνθρωπος γέγονε καὶ δι’ αὐτοῦ οὐχ ὁ Ἰσραὴλ μόνος τὸν Ἰορδάνην διῆλθεν, ἀλλ’ ἡ φύσις ἅπασα πρὸς τὴν ἀρχαίαν ἐπανῆλθε μακαριότητα, δι’ οὗ ἡ φύσις ἐκ τῶν κατωτάτων τῆς γῆς ἀνῆλθεν ὑπεράνω πάσης ἀρχῆς καὶ ἐν αὐτῷ τῷ πατρικῷ κεκάθικε θρόνῳ.

\switchenglish

And God ordered twelve stones to be taken 
out of the Jordan, and specified why. For he 
says: ``When your son asks you the meaning 
of these stones, tell him how the water left the 
Jordan by the divine command, and how the 
ark was saved and the whole people.'' How, 
then, shall we not record on image the saving 
pains and wonders of Christ our Lord, so that 
when my child asks me, ``What is this?'' I 
may say, that God the Word became man, and 
that for His sake not Israel alone passed 
through the Jordan, but all the human race 
gained their original happiness. Through 
Him human nature rose from the lowest 
depths of the earth higher than the skies, and 
in His Person sat down on the throne His 
Father had prepared for Him. 

\switchgreek

Ἀλλὰ φησι· Ποίει Χριστοῦ εἰκόνα, καὶ ἀρκέσθητι, ἢ τῆς τούτου μητρὸς τῆς θεοτόκου.
Ὢ τῆς ἀτοπίας!
Ἐχθρὸν σεαυτὸν διαῥῥήδην τῶν ἁγίων καθωμολόγησας· εἰ γὰρ Χριστοῦ μὲν εἰκόνα ποιεῖς, τῶν δὲ ἁγίων οὐδαμῶς, δῆλον ὡς οὐχὶ τὴν εἰκόνα ἀπαγορεύεις, ἀλλὰ τὴν τῶν ἁγίων τιμήν.
Τοῦ γὰρ Χριστοῦ, ὡς δεδοξασμένου, ποιεῖς εἰκόνας, τῶν δὲ ἁγίων, ὡς ἀδοξάστων, ἀποποιεῖς τὸ ἀπεικόνισμα, καὶ ψευδῆ τὴν ἀλήθειαν ἀποκαλεῖς.
«Ζῶ γὰρ ἐγώ,» λέγει Κύριος, «καὶ τοὺς δοξάζοντάς με δοξάσω», καὶ ὁ θεῖος ἀπόστολος· «Ὥστε οὐκέτι εἶ δοῦλος, ἀλλ’ υἱός· εἰ δὲ υἱός, καὶ κληρονόμος θεοῦ διὰ Χριστοῦ.» καὶ «εἴπερ συμπάσχομεν, ἵνα καὶ συνδοξασθῶμεν.» Οὐ κατὰ τῶν εἰκόνων ἤρω τὸν πόλεμον, ἀλλ’ ἦ κατὰ τῶν ἁγίων.
Φησὶ γοῦν Ἰωάννης ὁ θεόλογος καὶ ἐπιστήθιος, «ὅτι ὅμοιοι αὐτῷ ἐσόμεθα.» Ὥσπερ ὁ τῷ πυρὶ ἑνούμενος, οὐ τῇ φύσει, ἀλλὰ τῇ ἑνώσει, καὶ πυρώσει, καὶ μεθέξει πῦρ γίνεται, οὕτω καὶ τὴν σάρκα φημὶ τοῦ σαρκωθέντος υἱοῦ τοῦ θεοῦ· ἐκείνη γὰρ, τῇ καθ’ ὑπόστασιν καὶ μεθέξει τῆς θείας φύσεως, ἀτρέπτως θεὸς ἐχρημάτισεν, οὐκ ἐνεργείᾳ χρισθεῖσα θεοῦ, ὥσπερ τῶν προφητῶν ἕκαστος, παρουσίᾳ δὲ ὅλου τοῦ χρίοντος.
Ὅτι δὲ Θεὸς, καὶ θεοὶ οἱ ἅγιοι, «ὁ Θεός», φησίν, «ἔστη ἐν συναγωγῇ θεῶν.»
Καὶ ὅτι ἵσταται Θεὸς ἐν μέσῳ θεῶν, τὰς ἀξίας διαιρῶν, ὡς ἑρμηνεύων φησὶν ὁ θεῖος Γρηγόριος.
Οἱ γὰρ ἅγιοι καὶ ζῶντες πεπληρωμένοι ἦσαν Πνεύματος ἁγίου, καὶ τελευτησάντων αὐτῶν, ἡ χάρις τοῦ ἁγίου Πνεύματος ἀνεκφοιτήτως ἔνεστι καὶ ταῖς ψυχαῖς, καὶ τοῖς σώμασιν ἐν τοῖς τάφοις, καὶ τοῖς χαρακτῆρσι, καὶ ταῖς ἁγίαις εἰκόσιν αὐτῶν, οὐ κατ’ οὐσίαν, ἀλλὰ χάριτι καὶ ἐνεργείᾳ.

\switchenglish

But the adversary says: ``Make an image of 
Christ or of His mother who bore Him, and let that be sufficient.'' O what 
folly this is! On your own showing, you are 
absolutely against the saints. For if you make 
an image of Christ and not of the saints, it is 
evident that you do not disown images, but 
the honour of the saints. You make statues 
indeed of Christ as of one glorified, whilst you 
reject the saints as unworthy of honour, and 
call truth a falsehood. ``I live,'' says the Lord, 
``and I will glorify those who glorify Me.'' 
And the divine Apostle: therefore now he is 
not a servant, but a son. ``And if a son, an 
heir also through God.'' Again, ``If we suffer 
with Him, that we also may be glorified:'' 
you are not waging war against images, but 
against the saints. St. John, who rested on 
His breast, says, that we shall be like to Him : 
just as a man by contact with fire becomes 
fire, not by nature, but by contact and by 
burning and by participation, so is it, I appre 
hend, with the flesh of the Crucified Son of 
God. That flesh, by participation through 
union with the divine nature, 
was unchangeably God, not in virtue of grace 
from God as was the case with each of the 
prophets, but by the presence of the Fountain 
Head Himself. God, the Scripture says, 
stood in the synagogue of the gods, so that the 
saints, too, are gods. Holy Gregory takes the 
words, ``God stands in the midst of the gods,''
to mean that He discriminates their several 
merits. The saints in their lifetime were 
filled with the Holy Spirit, and when they are 
no more, His grace abides with their spirits 
and with their bodies in their tombs, and also 
with their likenesses and holy images, not by 
nature, but by grace and divine power. 

\switchgreek

Ὅν δὲ τῷ Δαβὶδ ἐπηγγείλατο ὁ Θεὸς οἰκοδομῆσαι αὐτῷ ναὸν διὰ τοῦ ἰδίου υἱοῦ, καὶ κατασκευάσαι τόπον ἀναπαύσεως, τοῦτον Σολομὼν οἰκοδομῶν, ἐποίησε Χερουβὶμ, ὥς φησιν ἡ βίβλος τῶν Βασιλειῶν.
Καὶ περιέσχε τὰ χερουβὶμ χρυσίῳ,
καὶ πάντας τοὺς τοίχους κύκλῳ·
καὶ κολαπτοὺς ἐνέγλυψεν Χερουβὶμἐν,
καὶ φοίνικας τῷ ἐσωτέρῳ καὶ ἐξωτέρῳ·
καὶ οὐκ εἶπεν ἐκ πλαγίων, ἀλλὰ «κύκλῳ»·
καὶ βόας καὶ λέοντας καὶ ῥοΐσκους.
Οὐ πολλῷ οὖν τιμιώτερον πάντας τοὺς τοίχους οἴκου Κυρίου κοσμῆσαι ἁγίων μορφαῖς καὶ ἐξεικονίσμασιν, ἤπερ ἀλόγων καὶ δένδρων; Ποῦ ὁ διαγορεύων νόμος·
«Οὐ ποιήσεις πᾶν ὁμοίωμα;»
Ἀλλὰ σοφίας χύσιν δεξάμενος Σολομὼν, οὐρανὸν εἰκονίζων, ἐποίει Χερουβὶμ, καὶ λεόντων, καὶ βοῶν ὁμοιώματα·
τοῦτο δὲ ἀπαγορεύει ὁ νόμος.
Εἰ δὲ ἡμεῖς Χριστὸν εἰκονίζοντες, καὶ ποιοῦμεν τὰ τῶν ἁγίων εἰκονίσματα, οὐκ ἄρα τοῦτο πάντως εὐσεβέστερον, ὅτι καὶ Πνεύματος ἁγίου εἰσὶ πεπληρωμένα;
Ὥσπερ γὰρ τότε αἵματι ἡγνίζετο ὁ λαός τε, καὶ ὁ ναὸς, καὶ σποδῷ δαμάλεως, οὕτω καὶ νῦν δὲ Χριστοῦ αἵματι, μαρτυρήσαντος ἐπὶ Ποντίου Πιλάτου, καὶ ἑαυτὸν ἀπαρχὴν τῶν μαρτύρων δείξαντος, ἔτι δὲ καὶ τῷ τῶν ἁγίων αἵματι,
ἐφ’ ὧν ἡ Ἐκκλησία οἰκοδομεῖται.
Καὶ τότε μὲν, μορφαῖς τε καὶ ἐκτυπώμασιν ἀλόγων ἐμψύχους, καὶ λογικοὺς ναοὺς ἑαυτοὺς εἰς κατοικητήριον Θεοῦ κατεσκευακότων.

\switchenglish

God charged David to build Him a temple 
through his son, and to prepare a place of rest. 
Solomon, in building the temple, made the 
cherubim, as the book of Kings says. And he 
encompassed the cherubim with gold, and all 
the walls in a circle, and he had the cherubim 
carved, and palms inside and out, in a circle, 
not from the sides, be it observed. And there 
were bulls and lions and pomegranates. Is it 
not more seemly to decorate all the walls of 
the Lord's house with holy forms and images 
rather than with beasts and plants? Where is 
the law declaring ``thou shalt not make any 
graven image''? But Solomon receiving the 
gift of wisdom, imaging heaven, made the 
cherubim, and the likenesses of bulls and lions, 
which the law forbade. Now if we make a 
statue of Christ, and likenesses of the saints, 
does not their being filled with the Holy Ghost 
increase the piety of our homage? As then 
the people and the temple were purified in 
blood and in burnt offerings, so now the Blood 
of Christ giving testimony under Pontius 
Pilate, and being Himself the first fruits of 
the martyrs, the Church is built up on the 
blood of the saints. Then the signs and 
forms of lifeless animals figured forth the 
human tabernacle, the martyrs themselves 
whom they were preparing for God's abode. 

\switchgreek

Ἱστοροῦμεν Χριστὸν τὸν βασιλέα καὶ κύριον οὐ γυμνοῦντες αὐτὸν τοῦ στρατεύματος· στρατὸς γὰρ τοῦ κυρίου οἱ ἅγιοι. Γυμνωσάτω ἑαυτὸν τοῦ οἰκείου στρατεύματος ὁ ἐπίγειος βασιλεὺς καὶ τότε τὸν ἑαυτοῦ βασιλέα καὶ κύριον.
Ἀποθέσθω τὴν ἁλουργίδα καὶ τὸ διάδημα καὶ τότε τῶν κατὰ τοῦ τυράννου ἀριστευσάντων καὶ βασιλευσάντων τῶν παθῶν τὸ σέβας περιαιρείτω.
Εἰ γὰρ κληρονόμοι θεοῦ καὶ συγκληρονόμοι Χριστοῦ καὶ τῆς θείας δόξης καὶ βασιλείας κοινωνοὶ ἔσονται, πῶς οὐχὶ καὶ τῆς ἐπὶ γῆς δόξης συμμέτοχοι γένωνται οἱ φίλοι Χριστοῦ; «Οὐ λέγω ὑμᾶς δούλους», φησὶν ὁ θεός, «ὑμεῖς φίλοι μού ἐστε.» Τῆς οὖν δεδομένης αὐτοῖς παρὰ τῆς ἐκκλησίας τιμῆς στερήσωμεν αὐτούς; Ὢ θρασείας χειρός.
Ὢ τολμηρᾶς γνώμης ἀνταιρούσης θεῷ καὶ τοῖς αὐτοῦ ἀντιπραττούσης προστάγμασιν. Οὐ προσκυνεῖς εἰκόνι, μηδὲ τῷ υἱῷ τοῦ θεοῦ προσκύνει, «ὅς ἐστιν εἰκὼν τοῦ ἀοράτου θεοῦ» ζῶσα καὶ χαρακτὴρ ἀπαράλλακτος.
Προσκυνῶ Χριστοῦ εἰκόνι ὡς σεσαρκωμένου θεοῦ, τῆς δεσποίνης τῶν ἁπάντων τῆς θεοτόκου οἷα μητρὸς τοῦ υἱοῦ τοῦ θεοῦ, τῶν ἁγίων ὡς φίλων θεοῦ τῶν μέχρις αἵματος ἀντικαταστάντων πρὸς τὴν ἁμαρτίαν καὶ Χριστὸν μιμησαμένων τῇ ὑπὲρ αὐτοῦ ἐκχύσει τοῦ αἵματος τὸ οἰκεῖον αἷμα ὑπὲρ αὐτῶν προεκχέαντος καὶ τῶν κατ’ ἴχνος αὐτοῦ πολιτευσαμένων.
Τούτων τὰς ἀριστείας καὶ τὰ πάθη ἀναγράπτους καθίστημι ὡς δι’ αὐτῶν ἁγιαζόμενος καὶ πρὸς ζῆλον μιμήσεως ἀλειφόμενος.
Καὶ ταῦτα δι’ αἰδοῦς ἄγω καὶ προσκυνήσεως· «ἡ γὰρ τῆς εἰκόνος τιμὴ πρὸς τὸ πρωτότυπον διαβαίνει», φησὶν ὁ θεῖος Βασίλειος.
Εἰ ναοὺς ἐγείρεις ἁγίοις θεοῦ, καὶ τὰ τούτων ἔγειρε τρόπαια.
Οὐκ ἠγείρετο πάλαι ναὸς ἐπ’ ἀνθρώπων ὀνόματι, οὐχ ἑωρτάζετο τῶν δικαίων ὁ θάνατος, ἀλλ’ ἐπενθεῖτο, καὶ ὁ ἁπτόμενος νεκροῦ ἀκάθαρτος ἐλογίζετο, καὶ Μωσέως αὐτοῦ.
Νῦν δὲ τῶν ἁγίων ἑορτάζεται τὰ μνημόσυνα· ἐπενθήθη ὁ νεκρὸς τοῦ Ἰακώβ, ἀλλ’ ὁ Στεφάνου πανηγυρίζεται θάνατος.
Ἢ τοίνυν καὶ τὰς πανηγυρικὰς μνήμας τῶν ἁγίων ἄνελε παρὰ τὸν παλαιὸν νόμον ἀγομένας ἢ καὶ τὰς εἰκόνας παρὰ τὸν νόμον, ὡς σὺ φής, οὔσας συγχώρησον.
Ἀλλ’ ἀμήχανον μὴ ἑορτάζειν τὰ τῶν ἁγίων μνημόσυνα· ὁ γὰρ τῶν ἁγίων ἀποστόλων καὶ θεοφόρων πατέρων χορὸς ταῦτα προστάττει γίνεσθαι.
Ἀφ’ οὗ γὰρ ὁ θεὸς λόγος σὰρξ ἐγένετο ὁμοιωθεὶς ἡμῖν κατὰ πάντα χωρὶς ἁμαρτίας καὶ ἀσυγχύτως ἐκράθη πρὸς τὸ ἡμέτερον καὶ ἀμεταβλήτως τὴν σάρκα ἐθέωσε διὰ τῆς ἐν ἀλλήλαις τῆς αὐτοῦ θεότητος καὶ τῆς αὐτοῦ σαρκὸς ἀσυγχύτου περιχωρήσεως, ὄντως ἡγιάσμεθα.
Καὶ ἀφ’ οὗ ὁ υἱὸς τοῦ θεοῦ καὶ θεὸς ὁ ἀπαθὴς ὢν τῇ θεότητι τῷ προσλήμματι πέπονθε καὶ τὸ ἡμέτερον ἀπέτισεν ὄφλημα λύτρον ἐκχέας ὑπὲρ ἡμῶν ἀξιόχρεών τε καὶ ἀξιάγαστον (δυσωπητικὸν γὰρ πατρὶ καὶ αἰδέσιμον αἷμα υἱοῦ), ὄντως ἠλευθερώμεθα.
Καὶ ἀφ’ οὗ κατελθὼν εἰς ᾅδην ταῖς ἀπ’ αἰῶνος πεπεδημέναις ψυχαῖς ὡς αἰχμαλώτοις ἐκήρυξεν ἄφεσιν, ὡς τυφλοῖς ἀνάβλεψιν, καὶ δήσας τὸν ἰσχυρὸν τῷ ὑπερέχοντι τῆς δυνάμεως ἀνέστη τὸ ἐξ ἡμῶν αὐτῷ προσληφθὲν ἀφθαρ τίσας σαρκίον, ὄντως ἠφθαρτίσμεθα.
Ἀφ’ οὗ τε δι’ ὕδατος καὶ πνεύματος γεγεννήμεθα, ὄντως υἱοθετήμεθα καὶ κληρονόμοι θεοῦ γεγενήμεθα.
Ἐντεῦθεν τοὺς πιστοὺς ἁγίους ὁ Παῦλος καλεῖ.
Ἐντεῦθεν τὸν τῶν ἁγίων οὐ πενθοῦμεν, ἀλλ’ ἑορτάζομεν θάνατον.
Ἐντεῦθεν «οὐχ ὑπὸ νόμον, ἀλλ’ ὑπὸ χάριν ἐσμέν», «δικαιωθέντες διὰ τῆς πίστεως» καὶ θεὸν μόνον εἰδότες τὸν ἀληθινόν–»δικαίῳ δὲ νόμος οὐ κεῖται»–, οὐχ ὑπὸ τὰ στοιχεῖα τοῦ νόμου ἐσμὲν δεδουλωμένοι ὡς νήπιοι, ἀλλ’ εἰς ἄνδρα καταρτισθέντες τέλειον στερεὰν τροφὴν τρεφόμεθα, οὐ τὴν πρὸς εἰδωλολατρείαν.
Καλὸς ὁ νόμος ὡς λύχνος φαίνων ἐν αὐχμηρῷ τόπῳ, ἀλλ’ ἕως οὗ ἡ ἡμέρα διαυγάσῃ· ἤδη δὲ ἀνέτειλε φωσφόρος ἐν ταῖς καρδίαις ἡμῶν καὶ ὕδωρ ζῶν τῆς θεογνωσίας θαλάσσας ἐθνῶν ἐπεκάλυψε καὶ πάντες τὸν κύριον ἔγνωμεν.
«Παρῆλθε τὰ παλαιά, ἰδοὺ γέγονε τὰ πάντα καινά.» Φησὶν γοῦν ὁ θεῖος ἀπόστολος πρὸς Πέτρον, τὴν κορυφαίαν ἀκρότητα τῶν ἀποστόλων· «Εἰ σὺ Ἰουδαῖος ὢν ἐθνικῶς ζῇς καὶ οὐκ Ἰουδαϊκῶς, πῶς τὰ ἔθνη ἀναγκάζεις ἰουδαΐζειν», καὶ πρὸς Γαλάτας γράφει· «Μαρτύρομαι παντὶ ἀνθρώπῳ περιτεμνομένῳ, ὅτι ὀφειλέτης ἐστὶν ὅλον τὸν νόμον πληρῶσαι.»

\switchenglish

We depict Christ as our King and Lord, 
and do not deprive Him of His army. The 
saints constitute the Lord's army. Let the 
earthly king dismiss his army before he gives 
up his King and Lord. Let him put off the 
purple before he takes honour away from his 
most valiant men who have conquered their 
passions. For if the saints are heirs of God, 
and co-heirs of Christ, they will be also par 
takers of the divine glory of sovereignty. If 
the friends of God have had a part in the 
sufferings of Christ, how shall they not receive 
a share of His glory even on earth? ``I call 
you not servants,'' our Lord says, ``you are my 
friends.'' Should we then deprive them of the 
honour given to them by the Church ? What 
audacity! What boldness of mind, to fight God 
and His commands! You, who refuse to 
worship images, would not worship the Son of 
God, the Living Image of the invisible God, 
and His unchanging form. I worship the 
image of Christ as the Incarnate God ; that 
of Our Lady the Mother of us 
all, as the Mother of God's Son; that of the 
saints as the friends of God. They have with 
stood sin unto blood, and followed Christ in 
shedding their blood for Him, who shed His 
blood for them. I put on record the excel 
lencies and the sufferings of those who have 
walked in His footsteps, that I may sanctify 
myself, and be fired with the zeal of imitation. 
St. Basil says, ``Honouring the image leads to 
the prototype.'' If you raise churches to the 
saints of God, raise also their trophies. The 
temple of old was not built in the name of 
any man. The death of the just was a cause 
of tears, not of feasting. A man who touched 
a corpse was considered unclean, even if the 
corpse was Moses himself. But now the 
memories of the saints are kept with rejoicings. 
The dead body of Jacob was wept over, whilst 
there is joy over the death of Stephen. There 
fore, either give up the solemn commemorations of the saints, which are not according 
to the old law, or accept images which are 
also against it, as you say. But it is impossible 
not to keep with rejoicing the memories of the 
saints. The Holy Apostles and Fathers are at 
one in enjoining them. From the time that 
God the Word became flesh He is as we are 
in everything except sin, and of our nature, 
without confusion. He has deified our flesh 
for ever, and we are in very deed sanctified 
through His Godhead and the union of His 
flesh with it. And from the time that God, 
the Son of God, impassible by reason of His 
Godhead, chose to suffer voluntarily He wiped 
out our debt, also paying for us a most full 
and noble ransom. We are truly free through 
the sacred blood of the Son pleading for us 
with the Father. And we are indeed delivered 
from corruption since He descended into hell 
to the souls detained there through centuries 
and gave the captives their freedom, sight to 
the blind, and chaining the strong one. He 
rose in the plenitude of His power, keeping the 
flesh of immortality which He had taken for 
us. And since we have been born again of 
water and the Spirit, we are truly sons and 
heirs of God. Hence St. Paul calls the faithful 
holy; hence we do not grieve but rejoice over 
the death of the saints. We are then no 
longer under grace, being justified through 
faith, and knowing the one true God. The 
just man is not bound by the law. We are 
not held by the letter of the law, nor do we 
serve as children, but grown into the perfect 
estate of man we are fed on solid food, not 
on that which conduces to idolatry. The law 
is good as a light shining in a dark place 
until the day breaks. Your hearts have already 
been illuminated, the living water of God s 
knowledge has run over the tempestuous seas 
of heathendom, and we may all know God. 
The old creation has passed away, and all 
things are renovated. The holy Apostle Paul 
said to St. Peter, the chief of the Apostles:
``If you, being a Jew, live as a heathen and 
not a Jew, how will you persuade heathens 
to do as Jews do?'' And to the Galatians: 
``I will bear witness to every circumcised man 
that it is salutary to fulfil the whole law.''

\switchgreek

Πάλαι μὲν οὖν μὴ εἰδότες θεὸν ἐδουλεύομεν τοῖς μὴ φύσει οὖσι θεοῖς, νῦν δὲ γνόντες θεόν, μᾶλλον δὲ γνωσθέντες ὑπὸ θεοῦ πῶς ἐπιστρέψομεν πάλιν ἐπὶ τὰ ἀσθενῆ καὶ πτωχὰ στοιχεῖα; Εἶδον εἶδος θεοῦ τὸ ἀνθρώπινον, «καὶ ἐσώθη μου ἡ ψυχή.»
Θεωρῶ εἰκόνα θεοῦ, ὡς εἶδεν Ἰακώβ, εἰ καὶ ἄλλως καὶ ἄλλως· ἐκεῖνος μὲν γὰρ ἄυλον, τὸ ἐσόμενον προμηνύουσαν ἀύλοις νοὸς ὀφθαλμοῖς, ἐγὼ δὲ τοῦ σαρκὶ ὁραθέντος μνήμης ἐμπύρευμα.
Ἡ τῶν ἀποστόλων σκιὰ τὰ σουδάριά τε καὶ σιμικίνθια νόσους ἀπήλαυνε, δαίμονας ἐφυγάδευε· καὶ πῶς ἡ σκιὰ καὶ ἡ εἰκὼν τῶν ἁγίων οὐ δοξασθήσεται; Ἢ πάσης ὕλης προσκύνησιν ἄνελε ἢ μὴ καινοτόμει «μηδὲ μέταιρε ὅρια αἰώνια, ἃ ἔθεντο οἱ πατέρες σου.»

\switchenglish

Of old they who did not know God, worshipped false gods. But now, knowing God,
or rather being known by Him, how can we 
return to bare and naked rudiments? I have 
looked upon the human form of God, and my 
soul has been saved. I gaze upon the image 
of God, as Jacob did, though in a different 
way. Jacob sounded the note of the future, 
seeing with immaterial sight, whilst the image 
of Him who is visible to flesh is burnt into my 
soul. The shadow and winding sheet and relics 
of the apostles cured sickness, and put demons 
to flight. How, then, shall not the shadow 
and the statues of the saints be glorified? 
Either do away with the worship of all matter, 
or be not an innovator. Do not disturb the 
boundaries of centuries, put up by your fathers. 

\switchgreek

Οὐ μόνον γράμμασι τὴν ἐκκλησιαστικὴν θεσμοθεσίαν παρέδωκαν, ἀλλὰ καὶ ἀγράφοις τισὶ παραδόσεσι.
Φησὶ γοῦν ὁ θεῖος Βασίλειος ἐν εἰκοστῷ ἑβδόμῳ τῶν πρὸς Ἀμφιλόχιον περὶ τοῦ ἁγίου πνεύματος τριάκοντα κεφαλαίων ἐπὶ λέξεως οὕτως· «Τῶν ἐν τῇ ἐκκλησίᾳ πεφυλαγμένων δογμάτων καὶ κηρυγμάτων τὰ μὲν ἐκ τῆς ἐγγράφου διδασκαλίας ἔχομεν, τὰ δὲ ἐκ τῆς τῶν ἀποστόλων παραδόσεως διαδοθέντα ἡμῖν ἐν μυστηρίῳ παρεδεξάμεθα, ἅπερ ἀμφότερα τὴν αὐτὴν ἰσχὺν ἔχει πρὸς τὴν εὐσέβειαν.
Καὶ τούτοις οὐδεὶς ἀντερεῖ οὐκοῦν, ὅστις γε κἂν μικρὸν γοῦν θεσμῶν ἐκκλησίας πεπείραται· εἰ γὰρ ἐπιχειρήσαιμεν τὰ ἄγραφα τῶν ἐθῶν ὡς μὴ μεγάλην ἔχοντα τὴν δύναμιν παραιτεῖσθαι, λάθοιμεν ἂν εἰς αὐτὰ τὰ καίρια ζημιοῦντες τὸ εὐαγγέλιον.» Ταῦτα τοῦ μεγάλου Βασιλείου τὰ ῥήματα.
Πόθεν γὰρ ἴσμεν τὸν κρανίου τόπον τὸν ἅγιον, τὸ μνῆμα τῆς ζωῆς; Οὐ παῖδες παρὰ πατρὸς ἀγράφως παρειληφότες; Τὸ μὲν γὰρ ἐν τόπῳ κρανίου ἐσταυρῶσθαι τὸν κύριον γέγραπται καὶ τετάφθαι ἐν μνημείῳ, ὃ ἐλατόμησεν Ἰωσὴφ ἐν τῇ πέτρᾳ· ὅτι δὲ ταῦτά ἐστι τὰ νῦν προσκυνούμενα, ἐξ ἀγράφου παραδόσεως ἴσμεν καὶ πλεῖστα τούτοις παρόμοια.
Πόθεν τὸ τρὶς βαπτίζειν; Πόθεν τὸ κατ’ ἀνατολὰς εὔχεσθαι; Πόθεν ἡ τῶν μυστηρίων παράδοσις; Διὸ καὶ ὁ θεῖος ἀπόστολος Παῦλός φησιν· «Ἄρα οὖν, ἀδελφοί, στήκετε καὶ κρατεῖτε τὰς παραδόσεις, ἃς ἐδιδάχθητε εἴτε διὰ λόγου, εἴτε δι’ ἐπιστολῶν ἡμῶν.» Πολλῶν τοιγαροῦν καὶ τοσούτων ἀγράφως τῇ ἐκκλησίᾳ παραδεδομένων καὶ μέχρι τοῦ νῦν πεφυλαγμένων, τί περὶ τὰς εἰκόνας σμικρολογεῖς;

\switchenglish

It is not in writing only that they have be 
queathed to us the tradition of the Church, but 
also in certain unwritten examples. In the 
twenty-seventh book of his work, in thirty 
chapters addressed to Amphilochios concern 
ing the Holy Spirit, St. Basil says, ``In the 
cherished teaching and dogmas of the Church, 
we hold some things by written documents; 
others we have received in mystery from the 
apostolical tradition.'' Both are of equal value 
for the soul's growth. No one will dispute 
this who has considered even a little the discipline of the Church. For if we neglect unwritten customs, as not having much weight, 
we bury in oblivion the most pertinent facts 
connected with the Gospel. These are the 
great Basil's words. How do we know the 
Holy place of Calvary, or the Holy Sepulchre? 
Does it not rest on a tradition handed clown 
from father to son? It is written that our 
Lord was crucified on Calvary, and buried in 
a tomb, which Joseph hewed out of the rock ; 
but it is unwritten tradition which identifies 
these spots, and does more things of the same 
kind. Whence come the three immersions 
at baptism, praying with face turned towards 
the east, and the tradition of the mysteries?\footnote{τὰ θεία μυστήρια\textemdash the Mass.}
Hence St. Paul says, ``Therefore, brethren, stand 
fast, and hold the traditions which you have 
learned either by word, or by our epistle.'' As, 
then, so much has been handed down in the 
Church, and is observed down to the present 
day, why disparage images? 

\switchgreek

Ἃς μέντοι χρήσεις παράγεις, οὐ τῶν παρ’ ἡμῖν εἰκόνων βδελύσσονται τὴν προσκύνησιν, ἀλλὰ τῶν ταύτας θεοποιούντων Ἑλλήνων. Οὐ δεῖ τοίνυν διὰ τὴν τῶν Ἑλλήνων ἄτοπον χρῆσιν καὶ τὴν ἡμετέραν εὐσεβῶς γινομένην ἀναιρεῖν. Ἐφορκίζουσιν ἐπαοιδοί τε καὶ γόητες, ἐφορκίζει καὶ τοὺς κατηχουμένους ἡ ἐκκλησία, ἀλλ’ ἐκεῖνοι μὲν ἐπικαλούμενοι δαίμονας, αὕτη δὲ θεὸν κατὰ δαιμόνων· δαίμοσι τὰς εἰκόνας ἀνατιθέασιν Ἕλληνες καὶ θεοὺς ταύτας προσαγορεύουσιν, ἡμεῖς δὲ ἀληθεῖ θεῷ σαρκωθέντι καὶ θεοῦ δούλοις καὶ φίλοις δαιμόνων ἀπελαύνουσι στίφη.

\switchenglish

If you bring forward certain practices, they 
do not inculpate our worship of images, but 
the worship of heathens who make them 
idols. Because heathens do it foolishly, this 
is no reason for objecting to our pious practice. 
If the same magicians and sorcerers use supplication, so does the Church with catechumens;
the former invoke devils, but the Church calls 
upon God against devils. Heathens have 
raised up images to demons, whom they call 
gods. Now we have raised them to the one 
Incarnate God, to His servants and friends, 
who are proof against the diabolical hosts. 

\switchgreek

Εἰ δὲ φὴς τὸν θεῖον καὶ θαυμαστὸν Ἐπιφάνιον διαρρήδην ταύτας ἀπαγορεῦσαι, πρῶτον μὲν τυχὸν παρεγγεγραμμένος καὶ ἐπίπλαστος ὁ λόγος, ἄλλου μὲν ὢν πόνος, ἑτέρου δὲ τὴν ἐπωνυμίαν ἔχων, ὃ πολλοῖς εἴθισται δρᾶν.
Δεύτερον, ἴσμεν τὸν μακάριον Ἀθανάσιον ἀπηγορευκότα τὸ ἐν λάρναξι τιθέναι τὰ τῶν ἁγίων λείψανα, μᾶλλον δὲ προστάττοντα ὑπὸ γῆν ταῦτα καλύπτειν, τὸ ἄτοπον ἔθος τῶν Αἰγυπτίων καταργῆσαι βουλόμενον, οἳ τοὺς ἑαυτῶν νεκροὺς οὐχ ὑπὸ γῆν ἔκρυπτον, ἀλλ’ ἐπὶ κλινῶν καὶ σκιμπόδων ἐτίθουν.
Τάχα τοιοῦτόν τι καὶ ὁ μέγας Ἐπιφάνιος ἐπιδιορθώσασθαι θέλων, τὸ μὴ χρῆναι ποιεῖν εἰκόνας ἐνομοθέτησεν, εἴ γε καὶ αὐτοῦ δῶμεν εἶναι τὸν λόγον, ἐπεί, ὅτι γε τούτου σκοπὸς ταύτας οὐκ ἀπωθεῖτο, μάρτυς ἡ τοῦ αὐτοῦ θείου Ἐπιφανίου ἐκκλησία εἰκόσι μέχρις ἡμῶν περικεκοσμημένη.
Τρίτον, οὐ τὸ σπάνιον νόμος τῇ ἐκκλησίᾳ «οὐδὲ μία χελιδὼν ἔαρ ποιεῖ», ὡς καὶ τῷ θεολόγῳ Γρηγορίῳ καὶ τῇ ἀληθείᾳ δοκεῖ· οὐδὲ λόγος εἷς δυνατὸς ὅλης ἐκκλησίας τῆς ἀπὸ γῆς περάτων μέχρι τῶν αὐτῆς περάτων ἀνατρέψαι παράδοσιν.

\switchenglish

If, again, you object that the great Epiphanius thoroughly rejected images, I would say 
in the first place the work in question is fictitious and unauthentic. It bears the name of 
some one who did not write it, which used to 
be commonly done. Secondly, we know that 
blessed Athanasius objected to the bodies of 
saints being put into chests, and that he 
preferred their burial in the ground, wishing 
to set at nought the strange custom of the 
Egyptians, who did not bury their dead under 
ground, but set them upon beds and couches. 
Thus, supposing that he really wrote this work, 
the great Epiphanius, wishing to correct some 
thing of the same kind, ordered that images 
should not be used. The proof that he did 
not object to images, is to be found in his 
own church, which is adorned with images 
to this day. Thirdly, the exception is not a 
law to the Church, neither does one swallow 
make summer, as it seems to Gregory the 
theologian, and to the truth. Neither can one 
expression overturn the tradition of the whole 
Church which is spread throughout the world. 

\switchgreek

Δέχου τοίνυν τῶν γραφικῶν καὶ πατρικῶν χρήσεων τὸν ἑσμόν, ὅτι, εἰ καὶ λέγει ἡ γραφή· «Τὰ εἴδωλα τῶν ἐθνῶν ἀργύριον καὶ χρυσίον, ἔργα χειρῶν ἀνθρώπων», ἀλλ’ οὖν οὐ τὸ μὴ προσκυνεῖν ἀψύχοις ἢ ἔργοις χειρῶν κωλύει, ἀλλὰ ταῖς δαιμόνων εἰκόσιν.

\switchenglish

Accept, therefore, the teaching of Scripture 
and spiritual writers. If the Scripture \emph{does} call 
the idols of heathens silver and gold, and the 
works of man's hand, it does not forbid the 
adoration of inanimate things, or man's handi 
work, but the adoration of demons. 

\switchgreek

Ὅτι μὲν οὖν ἀγγέλοις καὶ ἀνθρώποις καὶ βασιλεῦσι καὶ ἀσεβέσι προσεκύνησαν οἱ προφῆται καὶ ῥάβδῳ, εἴρηται· λέγει δὲ καὶ Δαυίδ· «Καὶ προσκυνεῖτε τῷ ὑποποδίῳ τῶν ποδῶν αὐτοῦ.»
Ἡσαΐας δὲ ἐκ προσώπου τοῦ θεοῦ· «Ὁ οὐρανός μοι θρόνος», φησίν, «ἡ δὲ γῆ ὑποπόδιον τῶν ποδῶν μου.»
Οὐρανὸς δὲ καὶ γῆ παντί που δῆλον, ὅτι κτίσματα.
Καὶ Μωσῆς δὲ καὶ Ἀαρὼν σὺν παντὶ τῷ λαῷ χειροποιήτοις προσεκύνησαν.
Φησὶ γοῦν Παῦλος ὁ χρυσοῦς τέττιξ τῆς ἐκκλησίας ἐν τῇ πρὸς Ἑβραίους ἐπιστολῇ·
«Χριστὸς δὲ παραγενόμενος ἀρχιερεὺς τῶν μελλόντων ἀγαθῶν διὰ τῆς μείζονος καὶ
τελειοτέρας σκηνῆς οὐ χειροποιήτου, τουτέστιν οὐ ταύτης τῆς κτίσεως», καὶ
πάλιν «οὐ γὰρ εἰς χειροποίητα εἰσῆλθεν ἅγια ὁ Χριστός, ἀντίτυπα τῶν ἀληθινῶν,
ἀλλ’ εἰς τὸν οὐρανόν.»
Ὥστε τὰ πρότερα ἅγια, ἥ τε σκηνὴ καὶ πάντα τὰ ἐν αὐτῇ, χειροποίητα ἦν· καὶ ὅτι προσεκυνεῖτο, οὐδεὶς ἀντερεῖ.

\switchenglish

We have seen that prophets worshipped 
angels, and men, and kings, and the impious, 
and even a staff. David says, ``And you 
adore His footstool.'' Isaias, speaking in God's 
name, says, ``The heavens are my throne, and 
the earth my footstool.'' Now, it is evident to 
every one that the heavens and the earth are 
created things. Moses, too, and Aaron with 
all the people adored the work of hands. St. 
Paul, the golden grasshopper of the Church, 
says in his Epistle to the Hebrews, ``But 
Christ being come, a high priest of the good 
things to come, by a greater and more perfect 
tabernacle not made by hand'' that is ``not of 
this creation.'' And, again, ``For Jesus is not 
entered into the Holies made by hands, the 
patterns of the true; but into heaven itself.'' 
Thus the former holy things, the tabernacle, 
and everything within it, were made by hands, 
and no one denies that they were adored. 

\switchgreek

\subsection*{Μαρτυρίαι παλαιῶν καὶ δοκίμων ἁγίων Πατέρων περὶ εἰκόνων.}

\switchenglish

\subsection*{Authentic Testimony of Ancient Fathers in favour of Images.}

\switchgreek

\subsubsection*{Τοῦ ἁγίου Διονυσίου τοῦ Ἀρεοπαγίτου, ἐκ τῆς πρὸς Τίτον Ἐπιστολῆς.}

\switchenglish

\subsubsection*{St. Denis the Areopagite. From his Letter to Bishop Titus.}

\switchgreek

«Χρὴ τοιγαροῦν καὶ ἡμᾶς ἀντὶ τῆς δημώδους περὶ αὐτῶν ὑπολήψεως εἴσω τῶν ἱερῶν
συμβόλων ἱεροπρεπῶς διαλαβεῖν καὶ μηδὲ ἀτιμάζειν αὐτὰ τῶν θείων ὄντα
χαρακτήρων ἔκγονα καὶ ἀποτυπώματα καὶ εἰκόνας ἐμφανεῖς τῶν ἀπορρήτων καὶ
ὑπερφυῶν θεαμάτων.»

\switchenglish

``Instead of attaching the common conception 
to images, we should look upon what they 
symbolise, and not despise the divine mark and 
character which they portray, as sensible images 
of mysterious and heavenly visions.''

\switchgreek

\emph{Σχόλιον.} Ἴδε, ὡς ἔφη μὴ ἀτιμάζειν τὰς τῶν τιμίων εἰκόνας.

\switchenglish

\emph{Commentary.} Mark that he cautions us not 
to despise sacred images. 

\switchgreek

\subsubsection*{Τοῦ αὐτοῦ, ἐκ τοῦ Περὶ θείων ὀνομάτων.}

\switchenglish

\subsubsection*{The Same, ``On the Names of God.''}

\switchgreek

Ταύτης καὶ ἡμεῖς μεμυήμεθα· νῦν μὲν ἀναλόγως ἡμῖν διὰ τῶν ἱερῶν παραπετασμάτων
τῆς τῶν λογίων καὶ ἱεραρχικῶν παραδόσεων φιλανθρωπίας αἰσθητοῖς τὰ νοητὰ καὶ
τοῖς οὖσι τὰ ὑπερούσια περικαλυπτούσης καὶ μορφὰς καὶ τύπους τοῖς ἀμορφώτοις
τε καὶ ἀτυπώτοις περιτιθείσης καὶ τὴν ὑπερφυᾶ καὶ ἀσχημάτιστον ἁπλότητα τῇ
ποικιλίᾳ τῶν μεριστῶν συμβόλων πληθυούσης τε καὶ διαπλαττούσης.

\switchenglish

We have taken the same line. On the one 
side, through the veiled language of Scripture 
and the help of oral tradition, intellectual things 
are understood through sensible ones, and the 
things above nature by the things that are. 
Forms are given to what is intangible and 
without shape, and immaterial perfection is 
clothed and multiplied in a variety of different 
symbols. 

\switchgreek

\emph{Σχόλιον.} Εἰ φιλανθρωπίας ἐστὶ τὸ τοῖς ἀτυπώτοις καὶ ἀμορφώτοις καὶ τοῖς
ἁπλοῖς καὶ ἀσχηματίστοις μορφὰς καὶ τύπους περιτιθέναι πρὸς τὴν ἡμετέραν
ἀναλογίαν, πῶς τὰ μορφαῖς καὶ σχήμασιν ὁραθέντα μὴ ἀναλόγως ἡμῖν εἰκονίσομεν
πρὸς μνήμην καὶ τὴν ἐκ τῆς μνήμης πρὸς ζῆλον κίνησιν;

\switchenglish

\emph{Commentary.} If it be a good work to clothe 
with shape and form, according to our standard, 
that which is formless, shapeless, and without 
consistency, how shall we not make images to 
ourselves in the same way of things perceived 
through form and shape, so that we may bear 
them in mind, and be moved to imitate what 
they represent. 

\switchgreek

\subsubsection*{Τοῦ αὐτοῦ, ἐκ τοῦ Περὶ ἐκκλησιαστικῆς ἱεραρχίας.}

\switchenglish

\subsubsection*{The Same, on the ``Ecclesiastical Hierarchy.''}

\switchgreek

Ἀλλ’ αἱ μὲν ὑπὲρ ἡμᾶς οὐσίαι καὶ τάξεις, ὧν ἤδη μνήμην ἱερὰν ἐποιησάμην,
ἀσώματοί τέ εἰσι, καὶ νοητὴ καὶ ὑπερκόσμιός ἐστιν ἡ κατ’ αὐτὰς ἱεραρχία. Τὴν
καθ’ ἡμᾶς δὲ ὁρῶμεν ἀναλόγως ἡμῖν αὐτοῖς τῇ τῶν αἰσθητῶν συμβόλων ποικιλίᾳ
πληθυνομένην, ὑφ’ ὧν ἱεραρχικῶς ἐπὶ τὴν ἑνοειδῆ θέωσιν ἐν συμμετρίᾳ τῇ καθ’
ἡμᾶς ἀναγόμεθα θεόν τε καὶ θείαν ἀρετήν, αἱ μὲν ὡς νόες νοοῦσιν κατὰ τὸ αὐταῖς
θεμιτόν, ἡμεῖς δὲ αἰσθηταῖς εἰκόσιν ἐπὶ τὰς θείας, ὡς δυνατόν, ἀναγόμεθα
θεωρίας.

\switchenglish

Now, if the substances and orders 
above us, of which we have already made 
reverent mention, are without bodies, their 
hierarchy is intellectual and above sense. 
We supply by the variety of sensible symbols 
the visible order, which is according to our 
own measure. Those sensible symbols lead us 
naturally to intellectual conception, to God and 
His divine attributes. Spiritual minds form 
their own spiritual conceptions, but we are led 
to the divine vision by sensible images. 

\switchgreek

\emph{Σχόλιον.} Εἰ τοίνυν ἀναλόγως ἡμῖν αὐτοῖς αἰσθηταῖς εἰκόσιν ἐπὶ τὴν θείαν
καὶ ἄυλον ἀναγόμεθα θεωρίαν καὶ φιλανθρώπως ἡ θεία πρόνοια τοῖς ἀσχηματίστοις
καὶ ἀτυπώτοις τύπους καὶ σχήματα τῆς ἡμῶν ἕνεκεν χειραγωγίας περιτίθησι, τί
ἀπρεπὲς τὸν σχήματι καὶ μορφῇ ὑποκύψαντα καὶ φύσει ὁραθέντα ὡς ἄνθρωπον
φιλανθρώπως δι’ ἡμᾶς εἰκονίζειν ἀναλόγως ἡμῖν αὐτοῖς;

\switchenglish

\emph{Commentary.} If, then, it be rational that 
we are led to the divine vision by sensible 
images, and if Divine Providence mercifully 
clothes in form and image that which is without 
either for our benefit, what is there unseemly 
about imaging, according to our capacity, Him 
who graciously disguised Himself for us in 
shape and form? 

\switchgreek

Λόγος ἄνωθεν εἰς ἡμᾶς παραδεδομένος κάτεισιν, Αὔγαρον, τὸν Ἐδέσσης ἄνακτα,
φήμῃ τῇ τοῦ κυρίου πρὸς θεῖον ἐκπυρσευθέντα ἔρωτα ἀπεσταλκέναι πρέσβεις τὴν
αὐτοῦ ἐπίσκεψιν ἐξαιτοῦντας. Εἰ δὲ ἀρνηθείη τοῦτο δράσειν, τὸ τούτου κελεύει
ὁμοίωμα ζωγράφῳ ἐκμάξασθαι· ὃ γνόντα τὸν πάντα εἰδότα καὶ πάντα δυνάμενον τὸ
ῥάκος εἰληφέναι καὶ τῷ προσώπῳ προσενεγκάμενον ἐν τούτῳ τὸν οἰκεῖον
ἐναπομάξασθαι χαρακτῆρα, ὃ καὶ μέχρι τοῦ νῦν σῴζεται.

\switchenglish

A tradition has come down to us that Angaros, 
King of Edessa, was drawn vehemently to 
divine love by hearing of our Lord,* and that 
he sent envoys to ask for His likeness. If this 
were refused, they were ordered to have a like 
ness painted. Then He, who is all-knowing 
and all-powerful, is said to have taken a strip of 
cloth, and pressing it to His face, to have left 
His likeness upon the cloth, which it retains to 
this day. 

\switchgreek

\subsubsection*{Τοῦ ἁγίου Βασιλείου, ἐκ τοῦ εἰς τὸν μακάριον Βαρλαὰμ τὸν μάρτυρα λόγου, οὗ ἡ
ἀρχή· Πρότερον μὲν τῶν ἁγίων οἱ θάνατοι.}

\switchenglish

\subsubsection*{St. Basil's Sermon on the Martyr St. Barlam, 
beginning, ``In the first place the death of 
the saints.''} 

\switchgreek

Ἀνάστητέ μοι νῦν, ὦ λαμπροὶ τῶν ἀθλητικῶν κατορθωμάτων ζωγράφοι, τὴν τοῦ
στρατηγοῦ κολοβωθεῖσαν εἰκόνα ταῖς ὑμετέραις μεγαλύνατε τέχναις· Ἀμαυρότερον
παρ’ ἐμοῦ τὸν στεφανίτην γραφέντα τοῖς τῆς ὑμετέρας σοφίας περιλάμψατε
χρώμασιν. Ἀπέλθω τῇ τῶν ἀριστευμάτων τοῦ μάρτυρος παρ’ ὑμῶν νενικημένος γραφῇ·
χαίρω τὴν τοιαύτην τῆς ὑμετέρας ἰσχύος σήμερον ἡττώμενος νίκην· ἴδω τῆς χειρὸς
πρὸς τὸ πῦρ ἀκριβέστερον παρ’ ὑμῶν γραφομένην τὴν πάλην· ἴδω φαιδρότερον ἐπὶ
τῆς ὑμετέρας τὸν παλαιστὴν γεγραμμένον εἰκόνος. Κλαυσάτωσαν δαίμονες καὶ νῦν
ταῖς τοῦ μάρτυρος ἐν ὑμῖν ἀριστείαις πληττόμενοι. Φλεγομένη πάλιν αὐτοῖς ἡ
χεὶρ καὶ νικῶσα δεικνύσθω. Ἐγγραφέσθω τῷ πίνακι καὶ ὁ τῶν παλαισμάτων
ἀγωνοθέτης Χριστός, ᾧ ἡ δόξα εἰς τοὺς αἰῶνας τῶν αἰώνων. Ἀμήν.

\switchenglish

Arise, you renowned painters of brave deeds, 
who set forth by your art a faint image of the 
General. My praise of the laurel-crowned 
victor is faint compared to the colours of your 
brush. I will give up writing on the excellencies 
of the martyr whom you have crowned. I 
rejoice at the victory won to-day by your 
strength. I contemplate the hand put out to 
the flames, more powerfully dealt with by you. 
I see the struggle more clearly depicted on your 
statue. Let demons be enraged even now, 
overcome by the martyr's excellencies which 
you reveal. Let the powerful hand be again 
outstretched to victory. May Christ our Lord, 
the supreme Judge of the warfare, appear in 
picture. To Him be glory for ever and ever. 
Amen. 

\switchgreek

\subsubsection*{Τοῦ αὐτοῦ, ἐκ τῶν πρὸς Ἀμφιλόχιον τριάκοντα κεφαλαίων,
περὶ τοῦ ἁγίου πνεύματος, ἀπὸ κεφαλαίου ιζʹ.}

\switchenglish

\subsubsection*{From the same, from the Thirty Chapters to Amphilochios,
on the Holy Ghost.\textemdash Chap. xviii.}

\switchgreek

Ὅτι βασιλεὺς λέγεται καὶ ἡ τοῦ βασιλέως εἰκών, καὶ οὐ δύο βασιλεῖς· οὔτε γὰρ
τὸ κράτος σχίζεται οὔτε ἡ δόξα διαμερίζεται. Ὡς γὰρ ἡ κρατοῦσα ἡμῶν ἀρχὴ καὶ
ἐξουσία μία, οὕτως καὶ ἡ παρ’ ἡμῶν δοξολογία μία καὶ οὐ πολλαί, διότι ἡ τῆς
εἰκόνος τιμὴ ἐπὶ τὸ πρωτότυπον διαβαίνει. Ὃ οὖν ἐστιν ἐνταῦθα μιμητικῶς ἡ
εἰκών, τοῦτο ἐκεῖ φυσικῶς ὁ υἱός. Καὶ ὥσπερ ἐπὶ τῶν τεχνητῶν κατὰ τὴν μορφὴν ἡ
ὁμοίωσις, οὕτω καὶ ἐπὶ τῆς θείας καὶ ἀσυνθέτου φύσεως ἐν τῇ κοινωνίᾳ τῆς
θεότητός ἐστιν ἡ ἕνωσις.

\switchenglish

The image of the king is also called the 
king, and there are not two kings in consequence.
Neither is power divided, nor is 
glory distributed. Just as the reigning power 
over us is one, so is our homage one, not 
many, and the honour given to the image 
reaches back to the original. What the image 
is in the one case as a representation, that the 
Son is by His humanity, and as in art likeness
is according to form, so in the divine and 
incommensurable nature union is 
effected in the indwelling Godhead. 

\switchgreek

\emph{Σχόλιον.} Εἰ ἡ εἰκὼν τοῦ βασιλέως βασιλεύς,
καὶ ἡ εἰκὼν τοῦ Χριστοῦ Χριστὸς
καὶ ἡ εἰκὼν τοῦ ἁγίου ἅγιος,
καὶ οὔτε τὸ κράτος σχίζεται οὔτε ἡ δόξα διαμερίζεται, ἀλλ’ ἡ
δόξα τῆς εἰκόνος τοῦ εἰκονιζομένου γίνεται. Δεδοίκασι τοὺς ἁγίους οἱ δαίμονες
καὶ τὴν σκιὰν αὐτῶν δραπετεύουσι· σκιὰ δὲ καὶ ἡ εἰκών, καὶ ταύτην ποιῶ ὡς
δαιμόνων ἐλάτειραν. Εἰ δὲ χρῆναι λέγοις νοερῶς μόνον θεῷ συνάπτεσθαι, ἄνελε
πάντα τὰ σωματικά, τὰ φῶτα, τὸ εὐῶδες θυμίαμα, αὐτὴν τὴν διὰ φωνῆς προσευχήν,
αὐτὰ τὰ ἐξ ὕλης τελούμενα θεῖα μυστήρια, τὸν ἄρτον, τὸν οἶνον, τὸ τῆς χρίσεως
ἔλαιον, τοῦ σταυροῦ τὸ ἐκτύπωμα. Ταῦτα γὰρ πάντα ὕλη· ὁ σταυρός, ὁ σπόγγος καὶ
κάλαμος, ἡ τὴν ζωηφόρον πλευρὰν νύξασα λόγχη. Ἢ τούτων ἁπάντων ἄνελε τὸ σέβας,
ὅπερ ἀδύνατον, ἢ μὴ ἀπαναίνου μηδὲ τὴν τῶν εἰκόνων τιμήν. Χάρις δίδοται θεία
ταῖς ὕλαις διὰ τῆς τῶν εἰκονιζομένων προσηγορίας. Ὥσπερ λιτὸν ἡ κογχύλη καθ’
ἑαυτὴν καὶ ἡ μέταξα καὶ τὸ ἐξ ἀμφοῖν ἐξυφασμένον ἱμάτιον καὶ καθ’ ἑαυτὸ
οὐδεμίαν ἔχει τιμήν, ἂν δὲ βασιλεὺς τοῦτο περίθηται, ἐκ τῆς προσούσης τῷ
ἠμφιεσμένῳ τιμῆς τῷ ἀμφιάσματι μεταδίδοται· οὕτως αἱ ὕλαι αὐταὶ μὲν καθ’
ἑαυτὰς ἀπροσκύνητοι, ἂν δὲ χάριτος εἴη πλήρης ὁ εἰκονιζόμενος, μέτοχοι χάριτος
γίνονται κατ’ ἀναλογίαν τῆς πίστεως. Εἶδον οἱ ἀπόστολοι τὸν κύριον σωματικοῖς
ὀφθαλμοῖς καὶ τοὺς ἀποστόλους ἕτεροι καὶ τοὺς μάρτυρας ἕτεροι. Ποθῶ κἀγὼ
τούτους ὁρᾶν ψυχῇ τε καὶ σώματι καὶ ἔχειν ἀλεξίκακον φάρμακον, ἐπεὶ διπλοῦς
ἔκτισμαι, καὶ ὁρῶν προσκυνῶ τὸ ὁρώμενον οὐχ ὡς θεόν, ἀλλ’ ὡς τιμίων εἰκόνισμα
τίμιον. Σὺ τυχὸν ὑψηλός τε καὶ ἄυλος καὶ ὑπὲρ τὸ σῶμα γενόμενος καὶ οἷον
ἄσαρκος καταπτύεις πᾶν τὸ ὁρώμενον, ἀλλ’ ἐγώ, ἐπεὶ ἄνθρωπός εἰμι καὶ σῶμα
περίκειμαι, ποθῶ καὶ σωματικῶς ὁμιλεῖν καὶ ὁρᾶν τὰ ἅγια. Συγκατάβηθι τῷ
ταπεινῷ μου φρονήματι, ὁ ὑψηλός, ἵνα σου τηρήσῃς τὸ ὑψηλόν. Ἀποδέχεται Χριστὸς
τὸν πρὸς αὐτόν μου πόθον καὶ τοὺς οἰκείους αὐτοῦ· χαίρει γὰρ δεσπότης
ἐγκωμιαζομένου δούλου εὐγνώμονος, ὁ μέγας ἔφη Βασίλειος ἐγκωμιάζων τοὺς
τεσσαράκοντα μάρτυρας. Ἄθρει δὲ οἷα καὶ εἰς Γόρδιον τὸν ἀοίδιμον τῷ λόγῳ
γεραίρων φησίν.

\switchenglish

\emph{Commentary.}
If the image of the king is 
the king, the image of Christ is Christ, and 
the image of a saint the saint, and if power 
is not divided nor glory distributed, honouring 
the image becomes honouring the one who is 
set forth in image. Devils have feared the 
saints, and have fled from their shadow. The 
shadow is an image, and I make an image 
that I may scare demons. If you say that 
only intellectual worship befits God, take away 
all corporeal things, light, and fragrance, prayer 
itself through the physical voice, the very divine 
mysteries which are offered through matter, 
bread, and wine, the oil of chrism, the sign of 
the Cross, for all this is matter. Take away 
the Cross, and the sponge of the Crucifixion, 
and the spear which pierced the life-giving- 
side. Either give up honouring these things 
as impossible, or do not reject the veneration 
of images. Matter is endued with a divine 
power through prayer made to those who are 
depicted in image. Purple by itself is simple, 
and so is silk, and the cloak which is made of 
both. But if the king put it on, the cloak 
receives honour from the honour due to the 
wearer. So is it with matter. By itself it is of 
no account, but if the one presented in image be 
full of grace, men become partakers of his grace 
according to their faith. The apostles knew 
our Lord with their bodily eyes ; others knew 
the apostles, others the martyrs. I, too, desire 
to see them in the spirit and in the flesh, and 
to possess a saving remedy as I am a com 
posite being. I see with my eyes, and revere 
that which represents what I honour, though I 
do not worship it as God. Now you, perhaps, 
are superior to me, and are lifted up above 
bodily things, and being, as it were, not of 
flesh, you make light of what is visible, but 
as I am human and clothed with a body, I 
desire to see and to be corporeally with the 
saints. Condescend to my humble wish that 
you may be secure on your heights. God 
accepts my longing for Him and for His saints. 
For He rejoices at the praises of His servant, 
according to the great St. Basil in his pane 
gyric of the Forty Martyrs. Listen to the 
words which he uttered in honour of the martyr 
St. Gordion. 

\switchgreek

\subsubsection*{Τοῦ ἁγίου Βασιλείου, ἐκ τοῦ εἰς Γόρδιον τὸν μάρτυρα λόγου.}

\switchenglish

\subsubsection*{From St. Basil's Sermon on St. Gordion.}

\switchgreek

Εὐφραίνονται λαοὶ εὐφροσύνην πνευματικὴν ἐπὶ μόνῃ τῇ ὑπομνήσει τῶν τοῖς
δικαίοις κατωρθωμένων εἰς ζῆλον καὶ μίμησιν τῶν ἀγαθῶν, ἀφ’ ὧν ἀκούουσιν,
ἐναγόμενοι· ἡ γὰρ τῶν εὐπολιτεύτων ἀνδρῶν ἱστορία οἷόν τι φῶς τοῖς σῳζομένοις
πρὸς τὴν τοῦ βίου ὁδὸν ἐμποιεῖ. Καὶ μετ’ ὀλίγα· Ὥστε, ὅταν διηγώμεθα τοὺς
βίους τῶν διαπρεψάντων ἐν εὐσεβείᾳ, δοξάζομεν πρῶτον τὸν δεσπότην διὰ τῶν
δούλων, ἐγκωμιάζομεν δὲ τοὺς δικαίους διὰ τῆς μαρτυρίας, ὧν ἴσμεν, εὐφραίνομεν
δὲ τοὺς λαοὺς διὰ τῆς ἀκοῆς τῶν καλῶν.

\switchenglish

The mere memory of just deeds is a source 
of spiritual joy to the whole world ; people are 
moved to imitate the holiness of which they 
hear. The life of holy men is as a light 
illuminating the way for those who would see 
it. And again, when we recount the story of 
holy lives we glorify in the first place the Lord 
of those servants, and we give praise to the 
servants on account of their testimony, which 
is known to us. We rejoice the world through 
good report. 

\switchgreek

\emph{Σκόλιον.} Ὅρα, ὡς θεοῦ μὲν δόξαν, τῶν δὲ ἁγίων ἐγκώμιον, τῶν δὲ λαῶν εὐφροσύνην καὶ
σωτηρίαν ἡ μνήμη τῶν ἁγίων συνίστησι. Τί οὖν ταύτην ἀφαιρεῖς; Ὅτι δὲ διὰ λόγου
καὶ εἰκόνων ἡ μνήμη γίνεται, φησὶν ὁ αὐτὸς θεῖος Βασίλειος.

\switchenglish

\emph{Commentary.} The remembrance of the saints 
is thus, you see, a glory to God, praise of the 
saints, joy and salvation to the whole world. 
Why, then, would you destroy it ? This remembrance is kept by preaching and by images, 
says the same great St. Basil.

\switchgreek

\subsubsection*{Τοῦ αὐτοῦ ἁγίου, εἰς τὸν Γόρδιον τὸν μάρτυρα.}

\switchenglish

\subsubsection*{The same, on the Martyr St. Gordion.}

\switchgreek

Ὥσπερ γὰρ τῷ πυρὶ αὐτομάτως ἕπεται τὸ φωτίζειν καὶ τῷ μύρῳ τὸ εὐωδεῖν, οὕτω
καὶ ταῖς ἀγαθαῖς πράξεσιν ἀναγκαίως ἀκολουθεῖ τὸ ὠφέλιμον. Καίτοι οὐδὲ τοῦτο
μικρόν, ἀκριβῶς τυχεῖν τῆς ἀληθείας τῶν τότε· ἀμυδρὰ γάρ τις μνήμη εἰς ἡμᾶς
διεδόθη τὰς ἐπὶ τῶν ἀγώνων ἀνδραγαθίας τοῦ ἀνδρὸς διασῴ ζουσα. Καί πως δοκεῖ
τὸ καθ’ ἡμᾶς τῷ τῶν ζωγράφων προσεοικέναι· καὶ γὰρ ἐκεῖνοι, ἐπειδὰν ἐξ
εἰκόνων μεταγράφωσι τὰς εἰκόνας, πλεῖστον ὡς εἰκὸς τῶν ἀρχετύπων
ἀπολιμπάνονται, καὶ ἡμᾶς αὐτῆς τῆς θέας τῶν πραγμάτων ἀπολειφθέντας κίνδυνος
οὐ μικρὸς τὴν ἀλήθειαν ἐλαττῶσαι.

\switchenglish

Just as burning follows naturally on fire, and 
fragrance on sweet ointment, so must good 
arise from holy actions. For it is no small 
thing to represent past events according to life. 
Is it a dim memory of the man's wrestlings 
which has come down to us, and does not the 
painter's picture tally with our present conflict?
Now, as painters draw images from images, 
they frequently depart from the original as 
much as the image itself does, and as we did 
not see what they represent, there is no little 
fear that we may injure the truth. 

\switchgreek

\subsubsection*{Ἐπὶ τέλει τοῦ αὐτοῦ λόγου.}

\switchenglish

\subsubsection*{The same, at the end.}

\switchgreek

Ὡς γὰρ τὸν ἥλιον ἀεὶ καθορῶντες ἀεὶ θαυμάζομεν, οὕτω καὶ τοῦ ἀνδρὸς ἐκείνου
ἀεὶ νεαρὰν τὴν μνήμην ἔχομεν.

\switchenglish

The sun fills us with perpetual wonder, 
though always before us, so the memory of 
this man is ever fresh. 

\switchgreek

\emph{Σκόλιον.} Δῆλον, ὡς ἀεὶ καθορῶντες διά τε λόγου καὶ εἰκονίσματος.

\switchenglish

\emph{Commentary.} It is evident that it is fresh 
through sermon and image.

\switchgreek

\subsubsection*{Καὶ ἐν τῷ λόγῳ δὲ τῷ εἰς τοὺς ὑπερτίμους τεσσαράκοντα μάρτυρας ταῦτά φησιν.}

\switchenglish

\subsubsection*{Testimony of the same, from his Sermon on the 
Forty Martyrs.}

\switchgreek

Μαρτύρων δὲ μνήμης τίς ἂν γένοιτο κόρος τῷ φιλομάρτυρι; Διότι ἡ πρὸς τοὺς
ἀγαθοὺς τῶν ὁμοδούλων τιμὴ ἀπόδειξιν ἔχει τῆς πρὸς τὸν κοινὸν δεσπότην
εὐνοίας. Καὶ πάλιν· Μακάρισον γνησίως τὸν μαρτυρήσαντα, ἵνα γένῃ μάρτυς τῇ
προαιρέσει καὶ ἐκβῇς χωρὶς διωγμοῦ, χωρὶς πυρός, χωρὶς μαστίγων τῶν αὐτῶν
ἐκείνοις ἐπαίνων ἠξιωμένος.

\switchenglish

Can the lover of the martyrs have too much 
of their memory ? For the honour shown to 
the just, our fellow-men, is a testimony to the 
goodness of our common Lord.

And again: 

Recognise the blessedness of the martyr 
heartily, that you may be a martyr in will ; 
thus, without persecutor, or fire, or blows, 
found worthy of the same reward. 

\switchgreek

\emph{Σκόλιον.} Πῶς οὖν με ἀπείργεις τῆς τῶν ἁγίων τιμῆς καὶ φθονεῖς μοι τῆς σωτηρίας; Ὅτι δὲ
συνεζευγμένην οἶδε τῷ λόγῳ τὴν διὰ τῶν χρωμάτων μορφήν, ἄκουε αὐτοῦ μετ’
ὀλίγα φάσκοντος.

\switchenglish

\emph{Commentary.} How, then, would you dissuade 
me from honouring the saints, and be envious 
of my salvation? Listen to what he says a 
little further on to show that he united the 
painter's art to oratory. 

\switchgreek

\subsubsection*{Βασιλείου.}

\switchenglish

\subsubsection*{St. Basil.}

\switchgreek

Δεῦρο δὴ οὖν εἰς μέσον αὐτοὺς ἀγαγόντες διὰ τῆς ὑπομνήσεως κοινὴν τὴν ἀπ’
αὐτῶν ὠφέλειαν τοῖς παροῦσι καταστησώμεθα, προδείξαντες πᾶσιν ὥσπερ ἐν γραφῇ
τὰς τῶν ἀνδρῶν ἀριστείας.

\switchenglish

See, then, that setting them before us in 
representation, we are making them helpful to 
the living, exhibiting their holiness to us all 
as if in a picture. 

\switchgreek

\emph{Σκόλιον.} Ὁρᾷς, ὡς ἓν εἰκόνος καὶ λόγου τὸ ἔργον; «Ὡς ἐν γραφῇ» γάρ,
ἔφη, «προδείξωμεν τῷ λόγῳ.» Καὶ πάλιν ἐχόμενα τοῦ λόγου· Ἐπεὶ καὶ πολέμων
ἀνδραγαθήματα καὶ λογογράφοι πολλάκις καὶ ζωγράφοι διασημαίνουσιν, οἱ μὲν τῷ
λόγῳ κοσμοῦντες, οἱ δὲ τοῖς πίναξιν ἐγχαράττοντες, καὶ πολλοὺς ἐπήγειραν εἰς
ἀνδρείαν ἑκάτεροι. Ἃ γὰρ ὁ λόγος τῆς ἱστορίας διὰ τῆς ἀκοῆς παρίστησι, ταῦτα
γραφικὴ σιωπῶσα διὰ μιμήσεως δείκνυσι.

\switchenglish

\emph{Commentary.} Do you understand that both 
image and sermon teach one lesson? He 
says: ``Let us show them forth in a sermon 
as if in a picture.'' And again: Writers and 
painters point out the struggles of war; the first 
by the art of style, the second with their brush, 
and each induce many to be brave. That 
which a spoken account presents to the hearing,
a silent picture portrays for imitation. 

\switchgreek

\emph{Σκὀλιον.} Τί τούτων τηλαυγέστερον πρὸς ἀπόδειξιν, ὅτι βίβλοι τοῖς
ἀγραμμάτοις εἰσὶν αἱ εἰκόνες καὶ τῆς τῶν ἁγίων τιμῆς ἀσίγητοι κήρυκες ἐν ἀήχῳ
φωνῇ τοὺς ὁρῶντας διδάσκουσαι καὶ τὴν ὅρασιν ἁγιάζουσαι; Οὐκ εὐπορῶ βίβλων, οὐ
σχολὴν ἄγω πρὸς τὴν ἀνάγνωσιν, εἴσειμι εἰς τὸ κοινὸν τῶν ψυχῶν ἰατρεῖον, τὴν
ἐκκλησίαν ὥσπερ ἀκάνθαις τοῖς λογισμοῖς συμπνιγόμενος· ἕλκει με πρὸς θέαν τῆς
γραφῆς τὸ ἄνθος καὶ ὡς λειμὼν τέρπει τὴν ὅρασιν καὶ λεληθότως ἐναφίησι τῇ ψυχῇ
δόξαν θεοῦ. Τεθέαμαι τὴν καρτερίαν τοῦ μάρτυρος, τῶν στεφάνων τὴν ἀνταπόδοσιν
καὶ ὡς πυρὶ πρὸς ζῆλον ἐξάπτομαι τῇ προθυμίᾳ καὶ πίπτων προσκυνῶ τῷ θεῷ διὰ
τοῦ μάρτυρος καὶ λαμβάνω τὴν σωτηρίαν. Οὐκ ἀκήκοας τοῦ αὐτοῦ θεοφόρου πατρὸς
λέγοντος ἐν τῷ εἰς τὴν ἀρχὴν τῶν ψαλμῶν λόγῳ, ὅτι «γινῶσκον τὸ πνεῦμα τὸ
ἅγιον, ὅτι δυσάγωγον πρὸς ἀρετὴν καὶ ῥᾴθυμον τὸ γένος τῶν ἀνθρώπων, τὸ μέλος
ταῖς ψαλμῳδίαις ἐγκατέμιξε»; Τί φής; Οὐ γράψω καὶ λόγῳ καὶ χρώμασι τὸ τῶν
μαρτύρων μαρτύριον καὶ ὀφθαλμοῖς καὶ χείλεσι περιπτύξομαι «τὸ θαυμαστὸν καὶ
ἀγγέλοις καὶ πάσῃ τῇ κτίσει, ὀδυνηρὸν δὲ τῷ διαβόλῳ καὶ φοβερὸν δαίμοσιν», ὡς
αὐτὸς ὁ φωστὴρ τῆς ἐκκλησίας ἔφησεν; Οἷα δέ φησι πρὸς τῷ τέλει τοῦ λόγου,
ἐγκωμιάζων τοὺς τεσσαράκοντα μάρτυρας· Ὦ χορὸς ἅγιος, ὦ σύστημα ἱερόν, ὦ
συνασπισμὸς ἀρραγής, ὦ κοινοὶ φύλακες τοῦ γένους τῶν ἀνθρώπων, ἀγαθοὶ κοινωνοὶ
φροντίδων, δεήσεων συνεργοί, πρεσβευταὶ δυνατώτατοι, ἀστέρες τῆς οἰκουμένης,
ἄνθη τῶν ἐκκλησιῶν (ἐγὼ δέ φημι, νοητά τε καὶ αἰσθητά). Ὑμᾶς οὐχ ἡ γῆ
κατέκρυψεν, ἀλλ’ οὐρανὸς ὑπεδέξατο. Ἠνοίγησαν ὑμῖν παραδείσου πύλαι, ἄξιον
θέαμα τῇ στρατιᾷ τῶν ἀγγέλων, ἄξιον πατριάρχαις, προφήταις, δικαίοις.

\switchenglish

\emph{Commentary.} What better proof have we 
that images are the books of the illiterate, the 
ever-speaking heralds of honouring the saints, 
teaching those who gaze upon them without 
words, and sanctifying the spectacle. I have 
not many books nor time for study, and I go 
into a church, the common refuge of souls, my 
mind wearied with conflicting thoughts. I see 
before me a beautiful picture and the sight 
refreshes me, and induces me to glorify God. 
I marvel at the martyr's endurance, at his 
reward, and fired with burning zeal, I fall 
down to adore God through His martyr, and 
receive a grace of salvation. Have you not 
heard the same holy father in his homily on 
the beginning of the Psalms, say that the Holy 
Spirit, knowing the human race were obstinate 
and hard to lead, mixed honey with the psalm-singing?
What do you say to this? Shall 
I not perpetuate the martyr's testimony both 
by word and paint brush? Shall I not em 
brace with my eyes that which is a wonder 
to the angels and to the whole world, formid 
able to the devil, a terror to demons, as the 
same great Father says? Again, towards the 
end of his homily on the forty martyrs, he 
exclaims, ``O sainted band! O sacred fraternity!
O invincible army! protectors of the human 
race, solace of the troubled, hope of your 
petitioners, most powerful intercessors, light 
of the world, bloom both intellectual and 
material of the Churches! The earth has 
not hidden you from sight, heaven has received you.
May its gates be opened to you. 
The spectacle is worthy of angels and patriarchs, prophets, and just.''

\switchgreek

\emph{Σκὀλιον.} Πῶς οὐ μὴ ποθήσω ἰδεῖν, ὃ ἰδεῖν ποθοῦσιν ἄγγελοι; Συνῳδὰ δὲ τούτοις καὶ ὁ
τούτου ἀδελφὸς καὶ ὁμογνώμων Γρηγόριος, ὁ τῆς Νυσαέων ἐπίσκοπος, φησίν.

\switchenglish

\emph{Commentary.} How shall I not desire to 
see what the angels desire? St. Basil's brother, 
who is one with him in thought, St. Gregory 
of Nyssa, shares his sentiments. 

\switchgreek

\subsubsection*{Τοῦ ἁγίου Γρηγορίου, ἐπισκόπου Νύσης, ἐκ τοῦ ἀναπληρωματικοῦ, τουτέστι τοῦ περὶ κατασκευῆς ἀνθρώπου, κεφαλαίου τετάρτου.}

\switchenglish

\subsubsection*{St. Gregory of Nyssa, from the Supplement, that is, from the ``Structure of Man,'' Fourth Chapter.}

\switchgreek

Ὥσπερ κατὰ τὴν ἀνθρωπίνην συνήθειαν οἱ τὰς εἰκόνας τῶν κρατούντων
κατασκευάζοντες τόν τε χαρακτῆρα τῆς μορφῆς ἀναμάσσονται καὶ τῇ περιβολῇ τῆς
πορφυρίδος τὴν βασιλικὴν ἀξίαν συμπαραγράφουσι καὶ λέγεται κατὰ τὴν συνήθειαν
καὶ εἰκὼν καὶ βασιλεύς, οὕτω καὶ ἡ ἀνθρωπίνη φύσις, ἐπειδὴ πρὸς ἀρχὴν τῶν
ἄλλων κατεσκευάζετο, οἷόν τις ἔμψυχος εἰκὼν ἀνεστάθη κοινωνοῦσα τῷ ἀρχετύπῳ
καὶ τῆς ἀξίας καὶ τοῦ ὀνόματος.

\switchenglish

Just as in human fashion 
the image makers of the powerful grasp the 
character of the form and set forth the royal 
dignity with the insignia of the purple, and 
their handiwork is called image or king, so is 
it with human nature. As it was created to 
rule over other creations, it was made as an 
animated type or image, partaking of the 
original in dignity and name. 

\switchgreek

\subsubsection*{Τοῦ αὐτοῦ, ἐκ τοῦ πέμπτου κεφαλαίου τῆς αὐτῆς πραγματείας.}

\switchenglish

\subsubsection*{The same, Fifth Chapter.}

\switchgreek

Τὸ δὲ θεῖον κάλλος οὐ σχήματί τινι καὶ μορφῆς εὐμοιρίᾳ διά τινος εὐχροίας
ἐναγλαΐζεται, ἀλλ’ ἐν ἀφράστῳ μακαριότητι κατ’ ἀρετὴν θεωρεῖται. Ὥσπερ τοίνυν
τὰς ἀνθρωπίνας μορφὰς διὰ χρωμάτων τινῶν ἐπὶ τοὺς πίνακας οἱ γραφεῖς
μεταφέρουσι τὰς οἰκείας τε καὶ καταλλήλους βαφὰς ἐπαλείφοντες τῷ μιμήματι, ὡς
ἂν δι’ ἀκριβείας τὸ ἀρχέτυπον κάλλος μετενεχθείη πρὸς τὸ ὁμοίωμα.

\switchenglish

The divine beauty is not set forth either in 
form or comeliness of design or colouring, but 
is contemplated in speechless blessedness, 
according to its virtue. So do painters 
transfer human forms to canvas through 
certain colours, laying on suitable and har 
monious tints to the picture, so as to transfer 
the beauty of the original to the likeness. 

\switchgreek

\emph{Σκόλιον.} Ὅρα, ὡς «τὸ μὲν θεῖον κάλλος οὐ σχήματί τινι διά τινος εὐχροίας ἐναγλαΐζεται»
καὶ διὰ τοῦτο οὐκ εἰκονίζεται, ἡ δὲ ἀνθρωπίνη μορφὴ διὰ χρωμάτων ἐπὶ τοὺς
πίνακας μεταφέρεται. Εἰ τοίνυν ὁ υἱὸς τοῦ θεοῦ ἐν ἀνθρώπου μορφῇ γέγονε μορφὴν
δούλου λαβὼν καὶ ἐν ὁμοιώματι ἀνθρώπων γενόμενος καὶ σχήματι εὑρεθεὶς ὡς
ἄνθρωπος, πῶς οὐκ εἰκονισθήσεται; Καὶ εἰ κατὰ συνήθειαν «λέγεται ἡ τοῦ
βασιλέως εἰκὼν βασιλεὺς» καὶ «ἡ τῆς εἰκόνος τιμὴ ἐπὶ τὸ πρωτότυπον ἀναβαίνει»,
ὥς φησιν ὁ θεῖος Βασίλειος, πῶς ἡ εἰκὼν οὐ τιμηθήσεται καὶ προσκυνηθήσεται οὐχ
ὡς θεός, ἀλλ’ ὡς θεοῦ σεσαρκωμένου εἰκών;

\switchenglish

\emph{Commentary.} You see that the divine 
beauty is not set forth in form or shape, and on 
this account it cannot be conveyed by an image:
it is the human form which 
is transferred to canvas by the artist's brush. 
If, therefore, the Son of God became man, 
taking the form of a servant, and appearing in 
man s nature, a perfect man, why should His 
image not be made? If, in common parlance, 
the king's image is called the king, and the 
honour shown to the image redounds to the 
original, as holy Basil says, why should the 
image not be honoured and worshipped, not as 
God, but as the image of God Incarnate?

\switchgreek

\subsubsection*{Τοῦ αὐτοῦ, ἀπὸ λόγου ῥηθέντος ἐν Κωνσταντινουπόλει περὶ
θεότητος Υἱοῦ καὶ Πνεύματος καὶ εἰς τὸν Ἀβραάμ.}

\switchenglish

\subsubsection*{The same, from his Sermon at Constantinople 
on the Godhead of the Son and of the 
Spirit, and on Abraham.}

\switchgreek

Ἐντεῦθεν δεσμοῖς πρότερον διαλαμβάνει ὁ πατὴρ τὸν παῖδα. Εἶδον πολλάκις ἐπὶ
γραφῆς εἰκόνα τοῦ πάθους καὶ οὐκ ἀδακρυτὶ τὴν θέαν παρῆλθον ἐναργῶς τῆς τέχνης
ὑπ’ ὄψιν ἀγούσης τὴν ἱστορίαν. Πρόκειται ὁ Ἰσαὰκ παρ’ αὐτῷ τῷ θυσιαστηρίῳ
ὀκλάσας ἐπὶ τὸ γόνυ καὶ περιηγμένας ἔχων εἰς τοὐπίσω τὰς χεῖρας, ὁ δὲ
ἐπιβεβηκὼς κατόπιν τῷ ποδὶ τῆς ἀγκύλης καὶ τῇ ἀριστερᾷ χειρὶ τὴν κόμην τοῦ
παιδὸς πρὸς ἑαυτὸν ἀνακλάσας ἐπικύπτει τῷ προσώπῳ ἐλεεινῶς πρὸς αὐτὸν
ἀναβλέποντι καὶ τὴν δεξιὰν καθωπλισμένος τῷ ξίφει πρὸς τὴν σφαγὴν κατευθύνει.
Καὶ ἅπτεται ἤδη τοῦ σώματος ἡ τοῦ ξίφους ἀκμή, καὶ τότε αὐτῷ γίνεται θεόθεν
φωνὴ τὸ ἔργον κωλύουσα.

\switchenglish

Then the father proceeds to bind his son. 
I have often seen paintings of this touching
scene, and could not look at it with dry eyes, 
art setting it forth so vividly. Isaac is lying 
before the altar, his legs bound, his hands tied 
behind his back. The father approaching the 
victim, clasping his hair with the left hand, 
stoops over the face so piteously turned to 
wards him, and holds in his right hand the 
sword, ready to strike. Already the point of 
the sword is on the body when the divine voice 
is heard, forbidding the consummation. 

\switchgreek

\subsubsection*{Τοῦ ἁγίου Ἰωάννου τοῦ Χρυσοστόμου ἐκ τῆς
ἑρμηνείας τῆς πρὸς Ἑβραίους ἐπιστολῆς.}

\switchenglish

\subsubsection*{St. John Chrysostom, from his commentary on the Epistle to the Hebrews.}

\switchgreek

Καί πως εἰκὼν τοῦ δευτέρου τὸ πρῶτον, ὁ Μελχισεδὲκ τοῦ Χριστοῦ, ὥσπερ ἄν τις εἴποι σκιὰν τῆς γραφῆς τῆς ἐν χρώμασι τὸ πρὸ ταύτης σκίασμα τοῦ γραφέως· διὰ τοῦτο γὰρ ὁ νόμος καλεῖται σκιά, ἡ δὲ χάρις ἀλήθεια, πράγματα δὲ τὰ μέλλοντα. Ὥστε ὁ μὲν νόμος καὶ ὁ Μελχισεδὲκ προσκίασμα τῆς ἐν χρώμασι γραφῆς, ἡ δὲ χάρις καὶ ἡ ἀλήθεια ἡ ἐν χρώμασι γραφή, τὰ δὲ πράγματα τοῦ μέλλοντος αἰῶνος, ὡς εἶναι τὴν παλαιὰν τύπου τύπον καὶ τὴν νέαν τῶν πραγμάτων τύπον.

\switchenglish

% Latin text from Patrologia Graeca XCIV
% Ac prius quidem posterioris imago quaedam fuit, Melchisedech scilicet,
% Christi: haud secus ac si quis adumbratam ante a pictore disgnationem,
% picturae ipsius, quae coloribus deinceps exprimitur, umbram nominet.
% Idcirco enim lex umbra vocatur, et gratia, veritas; res autem, illa quae
% futura sunt. Lex quidem et Melchisedech instar sunt adumbratae
% delineationis, quam pictura suis ornavit coloribus; gratia autem et
% veritas, coloribus aucta descriptio: res tandem iipsae, ad futurum aevum
% pertinent. Itaque Testamentum rerum figura.

% English translation not in Mary H. Allies' translation,
% used that from fordham_sourcebooks.html
How can what precedes be an image of what follows, as, for instance,
Melchisedech of Christ? Just in the same way as a sketch would be an
outline of the picture. On this account the old law is called a shadow, and
the new-the truth and what is to come-certainties. Thus Melchisedech,
who represents the law, is a foreshadowing of the picture. The new
dispensation is the truth; the picture fully completed shows forth
eternity. We might call the old dispensation a type of a type, and the new
a type of the things themselves.

\switchgreek

\subsubsection*{Λεοντίου Νεαπόλεως τῆς Κύπρου ἐκ τοῦ κατὰ Ἰουδαίων λόγου περὶ τοῦ προσκυνεῖν
τῷ σταυρῷ τοῦ Χριστοῦ καὶ ταῖς εἰκόσι τῶν ἁγίων καὶ ἀλλήλοις καὶ περὶ τῶν
λειψάνων τῶν ἁγίων.}

\switchenglish

\subsubsection*{Leo Bishop of Neapolis in Cyprus. From 
his book against the Jews, on the Adoration 
of the Cross, and the Statues of the Saints, 
and on Relics.}

\switchgreek

Ἐάν μοι ἐγκαλῇς πάλιν, ὦ Ἰουδαῖε, λέγων, ὅτι ὡς θεὸν προσκυνῶ τὸ ξύλον τοῦ
σταυροῦ, διὰ τί οὐκ ἐγκαλεῖς τῷ Ἰακὼβ προσκυνήσαντι ἐπὶ τὸ ἄκρον τῆς ῥάβδου;
Ἀλλὰ πρόδηλον, ὅτι οὐ τὸ ξύλον τιμῶν προσεκύνησεν, ἀλλὰ διὰ τοῦ ξύλου τῷ Ἰωσὴφ
προσεκύνησεν, ὥσπερ καὶ ἡμεῖς διὰ τοῦ σταυροῦ τὸν Χριστόν, ἀλλ’ οὐ τὸ ξύλον
δοξάζομεν.

\switchenglish

If you, O Jew, reproach me saying that 
I adore the wood of the Cross as God, why 
do you not reproach Jacob, who worshipped 
on the point of his staff? 
Now it is evident that he was not worshipping 
wood. So with us; we are worshipping Christ 
through the Cross, not the wood of the Cross. 

\switchgreek

\emph{Σκόλιον.} Εἰ οὖν τὸν τοῦ σταυροῦ τύπον προσκυνοῦμεν εἰκόνα τοῦ σταυροῦ
ποιοῦντες ἐξ οἱασοῦν ὕλης, πῶς τοῦ σταυρωθέντος εἰκόνι μὴ προσκυνήσομεν;

\switchenglish

\emph{Commentary.} If we adore the Cross, made 
of whatever wood it may be, how shall we not 
adore the image of the Crucified?

\switchgreek

\subsubsection*{Καὶ πάλιν τοῦ αὐτοῦ Λεοντίου.}

\switchenglish

\subsubsection*{From the same.}

\switchgreek

Ἐπεὶ καὶ Ἀβραὰμ τοῖς πωλήσασιν αὐτῷ τὸν τάφον ἀσεβέσιν ἀνθρώποις προσεκύνησε
καὶ γόνυ ἔκαμψεν ἐπὶ τὴν γῆν, ἀλλ’ οὐχ ὡς θεοῖς αὐτοῖς προσεκύνησεν· καὶ πάλιν
ὁ Ἰακὼβ τὸν Φαραὼ ηὐλόγησεν ἀσεβῆ καὶ εἰδωλολάτρην ὄντα, ἀλλ’ οὐχ ὡς θεὸν
αὐτὸν ηὐλόγησε· καὶ πάλιν, τὸν Ἠσαῦ πεσὼν προσεκύνησεν, ἀλλ’ οὐχ ὡς θεὸν
προσεκύνησεν. Καὶ πάλιν· Πῶς ἐντέλλεται ὑμῖν ὁ θεὸς προσκυνεῖν καὶ τῇ γῇ καὶ
τοῖς ὄρεσι; Λέγει γάρ· «Ὑψοῦτε κύριον τὸν θεὸν ἡμῶν καὶ προσκυνεῖτε εἰς ὄρος
ἅγιον αὐτοῦ. Καὶ προσκυνεῖτε τῷ ὑποποδίῳ τῶν ποδῶν αὐτοῦ, ὅτι ἅγιός ἐστι»,
τουτέστι τῇ γῇ. «Ὁ οὐρανὸς γάρ μοι θρόνος», φησίν, «ἡ δὲ γῆ ὑποπόδιον τῶν
ποδῶν μου», λέγει κύριος. Πῶς δὲ Μωσῆς προσεκύνησεν Ἰοθὸρ εἰδωλολάτρῃ ὄντι,
καὶ Δανιὴλ Ναβουχοδονόσορ; Πῶς ἐμοὶ ἐγκαλεῖς, ὅτι τιμῶ καὶ προσκυνῶ τοὺς τὸν
θεὸν τιμήσαντας καὶ προσκυνήσαντας; Οὐ συμφέρει, εἰπέ μοι, προσκυνεῖν τοῖς
ἁγίοις καὶ μὴ ὡς σὺ λιθοβολεῖν; Οὐ συμφέρει προσκυνεῖν καὶ μὴ τούτους
καταπρίζειν καὶ ἐν λάκκῳ βορβόρου τοὺς εὐεργέτας καταφέρειν; Εἰ τὸν θεὸν
ἠγάπας, πάντως ἂν καὶ τοὺς αὐτοῦ δούλους τιμᾶν ἔμελλες. Καὶ εἰ ἀκάθαρτά εἰσι
τῶν δικαίων τὰ ὀστᾶ, πῶς μετὰ τιμῆς πάσης μετεκομίσθησαν τὰ ὀστᾶ τοῦ Ἰακὼβ καὶ
Ἰωσὴφ ἐξ Αἰγύπτου; Πῶς νεκρὸς ἄνθρωπος ἁψάμενος τῶν ὀστέων Ἐλισσαίου εὐθέως
ἀνέστη; Εἰ δὲ δι’ ὀστέων θαυματουργεῖ ὁ θεός, εὔδηλον, ὅτι δύναται καὶ δι’
εἰκόνων καὶ λίθων καὶ ἑτέρων πολλῶν, καθὼς καὶ ἐπὶ Ἐλισσαιὲ ἐγένετο, ὃς ἔδωκε
τὴν ἰδίαν ῥάβδον τῷ ἑαυτοῦ παιδὶ καὶ εἶπε δι’ αὐτῆς πορευθέντα ἀναστῆσαι τὸν
παῖδα τῆς Σωμανίτιδος. Καὶ Μωσῆς ῥάβδῳ τὸν Φαραὼ ἐκόλασε καὶ θάλασσαν ἔσχισε
καὶ ὕδωρ ἐγλύκανε καὶ πέτραν ἔρρηξε καὶ ὕδωρ ἐξήγαγε. Καὶ Σολομών φησιν·
«Ηὐλόγηται ξύλον, δι’ οὗ γίνεται σωτηρία.» Καὶ Ἐλισσαιὲ ξύλον ἐν Ἰορδάνῃ
ἀπορρίψας σίδηρον ἀνήγαγε· καὶ «ξύλον ζωῆς» καὶ «φυτὸν Σαβὲκ» ἤγουν
συγχωρήσεως. Καὶ Μωσῆς ξύλῳ ὄφιν ὕψωσε καὶ λαὸν ἐζωοποίησε· ξύλῳ βλαστήσαντι
ἐν τῇ σκηνῇ τὴν ἱερατείαν ἐκύρωσεν. Ἀλλ’ ἴσως ἐρεῖς μοι, ὁ Ἰουδαῖος, ὅτι τὰ ἐν
τῇ σκηνῇ τοῦ μαρτυρίου ἅπαντα ὁ θεὸς γενέσθαι προσέταξε τῷ Μωσῇ· κἀγώ σοι
λέγω, ὅτι πολλὰ καὶ ποικίλα πράγματα ὁ Σολομὼν ἐν τῷ ναῷ γλυπτὰ καὶ χωνευτὰ
πεποίηκεν, ἅπερ οὐδὲ ὁ θεὸς αὐτῷ ποιῆσαι προσέταξεν οὐδὲ ἡ σκηνὴ τοῦ μαρτυρίου
ταῦτα ἐκέκτητο οὔτε ὁ ναός, ὃν ὁ θεὸς τῷ Ἰεζεκιὴλ ὑπέδειξε, καὶ οὐ κατεγνώσθη
ἐν τούτῳ ὁ Σολομών· εἰς δόξαν γὰρ θεοῦ τὰς τοιαύτας μορφὰς κατεσκεύασεν, ὥσπερ
δὴ καὶ ἡμεῖς. Εἶχες καὶ σὺ πολλὰς καὶ διαφόρους πρὸς ἀνάμνησιν θεοῦ εἰκόνας
καὶ σήμαντρα, πρὶν ἢ τούτων διὰ τὴν σὴν ἀγνωμοσύνην ἐστερήθης, τουτέστι τὴν
Μωσαϊκὴν ῥάβδον, τὰς θεοτύπους πλάκας, τὴν πυρένδροσον βάτον, τὴν ξηρένυδρον
πέτραν, τὴν μαννοφόρον κιβωτόν, τὸ πυρένθεον θυσιαστήριον, τὸ θεώνυμον
πέταλον, τὸ θεόδηλον Ἐφούδ, τὴν θεόσκιον σκηνήν· εἰ δὲ καὶ σὺ ἅπαντα ταῦτα
νυκτὸς καὶ ἡμέρας κατεσκιάζου λέγων· Δόξα σοι, ὁ μόνος παντοκράτωρ θεός, ὁ διὰ
πάντων τούτων ἐν Ἰσραὴλ θαυμάσια ποιήσας, εἰ δὲ διὰ πάντων τούτων τῶν νομικῶν,
ὧν εἶχές ποτε, προσπίπτων τῷ θεῷ προσεκύνεις, ὁρᾷς, ὅτι διὰ τῶν εἰκόνων
προσάγεται τῷ θεῷ ἡ προσκύνησις.

\switchenglish

Abraham worshipped the impious men who 
sold him the cave, and bent his knee to the 
ground, yet did not worship them as gods. 
Jacob praised Pharao, an impious idolater, yet 
not as God, and he fell down at the feet of 
Esau, yet did not worship him as God. And 
again, How does God order us to worship 
the earth and mountains? ``Exalt the Lord 
your God and worship Him upon His holy 
mountain, and adore His footstool,'' that is, 
the earth. For heaven is My throne, He 
says, and the earth My footstool. How was 
it that Moses worshipped Jothor, an idolator, 
and Daniel, Nabuchodonosor ? How can you 
reproach me because I honour those who 
honour God and show Him service? Tell 
me, is it not fitting to worship the saints, 
rather than to throw stones at them as you 
do? Is it not right to worship them, rather 
than to attack them, and to fling your bene 
factors into the mire ? If you loved God, 
you would be ready to honour His servants 
also. And if the bones of the just are 
unclean, why were the bones of Jacob and 
Joseph brought with all honour from Egypt ? 
How was it that a dead man arose again on 
touching the bones of Eliseus ? If God works 
wonders through bones, it is evident that He 
can work them through images, and stones, 
and many other things, as in the case of 
Eliseus, who gave his staff to his servant, 
saying, ``With this go and raise from the dead 
the son of the Sunamitess.'' With his staff 
Moses chastised Pharao, parted the waters, 
struck the rock, and drew forth the stream. 
And Solomon said, ``Blessed is the wood by 
which justice cometh.'' Eliseus took iron out 
of the Jordan with a piece of wood. And 
again, the wood is the wood of life, and the 
wood of Sabec, that is, of remission. Moses 
humbled the serpent with wood and saved the 
people. The blossoming rod in the tabernacle 
confirmed the priesthood of Aaron. Perhaps, 
O Jew, you will tell me that God prescribed 
to Moses beforehand all the things of the 
testimony in the tabernacle. Now, I say to 
you that Solomon made a great variety of 
things in the temple in carvings and sculpture, 
which God had not ordered him to do. Nor 
did the tabernacle of the testimony contain 
them, nor the temple which God showed to 
Ezechiel, nor was Solomon to be blamed in 
this. He had had these sculptured images 
made for the glory of God as we do. You, 
too, had many and varied images and signs 
in the Old Testament to serve as a reminder 
of God, if you had not lost them through 
ingratitude. For instance, the rod of Moses, 
the tablets of the law, the burning bush, the 
rock giving forth water, the ark containing 
the manna, the altar set on fire from above,
the lamina bearing the divine 
name, the ephod, the tabernacle overshadowed 
by God. If you had prepared all these things 
by day and by night, saying, ``Glory be to 
Thee, O Almighty God, who hast done 
wonders in Israel through all these things''; 
if through all these ordinances of the law, 
carried out of old, you had fallen on your 
knees to adore God, you would see that 
worship is given to Him by images. 

\switchgreek

\subsubsection*{Καὶ μετ’ ὀλίγα·}

\switchenglish

\subsubsection*{And further on:}

\switchgreek

Ὁ γὰρ ἀγαπῶν εἰλικρινῶς φίλον ἢ βασιλέα καὶ μάλιστα τὸν εὐεργέτην, κἂν υἱὸν
αὐτοῦ θεάσηται, κἂν ῥάβδον, κἂν θρόνον, κἂν στέφανον, κἂν οἶκον, κἂν δοῦλον,
κρατεῖ καὶ ἀσπάζεται καὶ τιμᾷ διὰ τούτων τὸν εὐεργέτην, βασιλέα καὶ μάλιστα
τὸν θεόν. –Εἴθε γάρ, πάλιν λέγω, ἐποίησας καὶ σὺ εἰκόνας Μωσαϊκὰς καὶ
προφητικὰς καὶ καθ’ ἡμέραν ἐν αὐταῖς προσεκύνεις τῷ δεσπότῃ αὐτῶν θεῷ. Ὅταν
τοίνυν ἴδῃς χριστιανῶν παῖδας προσκυνοῦντας τῷ σταυρῷ, γνῶθι, ὅτι τῷ
σταυρωθέντι Χριστῷ τὴν προσκύνησιν προσάγουσι καὶ οὐ τῷ ξύλῳ. Ἐπεὶ εἰ τὴν
φύσιν τοῦ ξύλου ἔσεβον, πάντως ἂν καὶ τὰ δένδρα καὶ τὰ ἄλση προσκυνεῖν εἶχον,
ὥσπερ σὺ ὁ Ἰσραὴλ προσεκύνησας τούτοις ποτέ, λέγων τῷ δένδρῳ καὶ τῷ λίθῳ, ὅτι
«σύ μου εἶ θεός, καὶ σύ με ἐγέννησας». Ἡμεῖς δὲ οὐχ οὕτως λέγομεν τῷ σταυρῷ
οὐδὲ ταῖς μορφαῖς τῶν ἁγίων· οὐ γὰρ θεοὶ ἡμῶν εἰσιν, ἀλλὰ βίβλοι ἀνεῳγμέναι
πρὸς ἀνάμνησιν θεοῦ καὶ τιμὴν αὐτοῦ ἐν ταῖς ἐκκλησίαις προφανῶς κείμεναι καὶ
προσκυνούμεναι. Ὁ γὰρ τιμῶν τὸν μάρτυρα τὸν θεὸν τιμᾷ, ᾧ ὁ μάρτυς ἐμαρτύρησεν·
ὁ προσκυνῶν τῷ ἀποστόλῳ τοῦ Χριστοῦ τῷ ἀποστείλαντι αὐτὸν προσκυνεῖ· καὶ ὁ
προσπίπτων τῇ μητρὶ τοῦ Χριστοῦ πρόδηλον, ὅτι τῷ υἱῷ αὐτῆς τὴν τιμὴν
προσφέρει. Οὐδεὶς γὰρ θεός, εἰ μὴ εἷς ὁ ἐν τριάδι καὶ μονάδι γνωριζόμενός τε
καὶ λατρευόμενος.

\switchenglish

He who truly loves a friend or the king, 
and especially his benefactor, if he sees that 
benefactor's son, or his staff, or his chair, or 
his crown, or his house, or his servant, he 
holds them fast in his embrace, and if he 
honours his benefactor, the king, how much 
more God. Again I repeat it, would that 
you had made images according to the law 
of Moses and the prophets, and that day by 
day you had worshipped the God of images. 
Whenever, then, you see Christians adoring 
the Cross, know that they are adoring the 
Crucified Christ, not the mere wood.\footnote{Compare\textemdash
\begin{verse}
Ce n'est ni la pierre ni le bois \\
\hspace{2em} Que le catholique adore; \\
Mais c'est le Roi qui mort en croix \\
\hspace{2em} De Son Sang la croix honore. \\
\textemdash\emph{Vie de St. Francois de Sales, par M. Hamond.} 
\end{verse}
}
If, indeed, they honoured wood as wood, they 
would be bound to worship trees of whatever 
kind, as you, O Israel, worshipped them of 
old, saying to the tree and to the stone, 
``Thou art my God and didst bring me forth.''
We do not speak either to the Cross or to 
the representations of the saints in this way. 
They are not our gods, but books which lie 
open and are venerated in churches in order 
to remind us of God and to lead us to 
worship Him. He who honours the martyr 
honours God, to whom the martyr bore 
testimony. He who worships the apostle 
of Christ worships Him who sent the apostle. 
He who falls at the feet of Christ s mother 
most certainly shows honour to her Son. 
There is no God but one, He who is known 
and adored in the Trinity. 

\switchgreek

\emph{Σκόλιον.} Οὗτος πιστὸς ἐξηγητὴς τῶν τοῦ μακαρίου Ἐπιφανίου λόγων ὁ τοῖς
ἑαυτοῦ λόγοις τὴν Κυπρίων κατακοσμήσας νῆσον ἢ οἱ ἀπὸ καρδίας λαλοῦντες.
Ἄκουε δὲ καὶ Σευηριανοῦ ἐπισκόπου Γαβάλων, οἷά φησιν.

\switchenglish

\emph{Commentary.} Who is the faithful inter 
preter of blessed Epiphanius\textemdash Leontius, whose 
teaching adorned the island of Cyprus, or 
those who spoke according to their own con 
ceits? Listen to the testimony of Severianus, 
Bishop of the Gabali. 

\switchgreek

\subsubsection*{Σευηριανοῦ, ἐπισκόπου Γαβάλων, ἀπὸ λόγου τοῦ εἰς τὰ ἐγκαίνια τοῦ σταυροῦ.}

\switchenglish

\subsubsection*{Severianus, Bishop of the Gabali, on the Dedication of the Cross.}

\switchgreek

Πῶς ἡ εἰκὼν τοῦ ἐπικαταράτου ζωὴν ἤνεγκε τοῖς ἡμετέροις προγόνοις; Καὶ μετ’
ὀλίγα· Πῶς οὖν ἡ εἰκὼν τοῦ ἐπικαταράτου ἤνεγκε τῷ λαῷ ἐν συμφορᾷ χειμαζομένῳ
σωτηρίαν; Ἆρα οὐκ ἦν ἀξιοπιστότερον εἰπεῖν· Ἐάν τις ὑμῶν δηχθῇ, βλέψει εἰς τὸν
οὐρανὸν ἄνω πρὸς τὸν θεὸν καὶ σωθήσεται ἢ εἰς τὴν σκηνὴν τοῦ θεοῦ; Ἀλλὰ ταῦτα
παριδὼν μόνον τοῦ σταυροῦ τὴν εἰκόνα ἔπηξεν. Διὰ τί οὖν ταῦτα ἐποίει Μωσῆς ὁ
εἰπὼν τῷ λαῷ· «Οὐ ποιήσεις σεαυτῷ γλυπτὸν οὐδὲ χωνευτὸν οὐδὲ πᾶν ὁμοίωμα, ὅσα
ἐν τῷ οὐρανῷ ἄνω καὶ ὅσα ἐν τῇ γῇ κάτω καὶ ὅσα ἐν τοῖς ὕδασιν ὑποκάτω τῆς
γῆς;» Ἀλλὰ τί ταῦτα πρὸς τὸν ἀγνώμονα φθέγγομαι; Εἰπέ, ὦ πιστότατε θεοῦ
θεράπον· ὃ ἀπαγορεύεις, ποιεῖς; ὃ ἀνατρέπεις, κατασκευάζεις; Ὁ λέγων· «Οὐ
ποιήσεις γλυπτόν», ὁ τὸν χωνευθέντα βοῦν κατελάσας, σὺ ὄφιν χαλκουργεῖς; Καὶ
τοῦτο οὐ λάθρᾳ, ἀλλὰ ἀναφανδὸν καὶ πᾶσι γνωστόν; Ἀλλ’ ἐκεῖνα, φησίν,
ἐνομοθέτησα, ἵνα ἐκκόψω τὰς ὕλας τῆς ἀσεβείας καὶ τὸν λαὸν ἀπαγάγω πάσης
ἀποστασίας καὶ εἰδωλολατρείας· νυνὶ δὲ χωνεύω τὸν ὄφιν χρησίμως εἰς προτύπωσιν
τῆς ἀληθείας. Καὶ καθάπερ σκηνὴν ἔπηξα καὶ τὰ ἐν αὐτῇ πάντα καὶ χερουβὶμ
ὁμοίωμα τῶν ἀοράτων διεπέτασα εἰς τὰ ἅγια ὡς τύπον καὶ σκιὰν τῶν μελλόντων,
οὕτω καὶ ὄφιν ἐστήλωσα εἰς σωτηρίαν τῷ λαῷ, ἵνα διὰ τῆς πείρας τούτων
προγυμνασθῶσι τοῦ σημείου τοῦ σταυροῦ τὴν εἰκόνα καὶ τὸν ἐν αὐτῷ σωτῆρα καὶ
λυτρωτήν. Καὶ ὅτι ἀψευδέστατος ὁ λόγος, ἀγαπητέ, ἄκουε τοῦ κυρίου τοῦτον
βεβαιοῦντος καὶ λέγοντος· «Καὶ καθὼς Μωσῆς ὕψωσε τὸν ὄφιν ἐν τῇ ἐρήμῳ, οὕτως
δεῖ ὑψωθῆναι τὸν υἱὸν τοῦ ἀνθρώπου, ἵνα πᾶς ὁ πιστεύων ἐν αὐτῷ μὴ ἀπόληται,
ἀλλ’ ἔχῃ ζωὴν αἰώνιον.»

\switchenglish

How was it that the image of the enemy 
gave life to our progenitors? \ldots
How was it that the image of the serpent 
worked salvation to the people in distress ? 
Would it not have been more reasonable to 
say, ``If any of you be bitten, let him look up to 
heaven, to God, and he shall be saved, or let 
him look towards the tabernacle of God''? 
Passing over this, he set up the image of the 
Cross alone. Why did Moses do this, who 
said to the people, ``Thou shalt not make to 
thyself a graven thing, nor the likeness of any 
thing that is in heaven above, or in the earth 
beneath, nor of those things that are in the 
waters under the earth''? However, why do 
I speak to unworthy people ? Tell me, devout 
servant of God, will you do what is forbidden, 
and disregard what you are told to do ? He 
who said, ``Thou shalt not make to thyself a 
graven thing,'' condemned the golden calf, and 
you make a brazen serpent, and this not 
secretly, but most openly, so that it is known to 
all. Moses answers, I laid down that com 
mandment in order to root out impiety, and to 
withdraw the people from all apostasy and 
idolatry ; now, I have the serpent cast for a 
good purpose\textemdash as a figure of the truth. And 
just as I have put up a tabernacle, and everything
in it, and cherubim, the likeness of the 
invisible powers, over the holy of holies, as a 
sign and figure of the future, so I have set 
up a serpent for the salvation of the people, to 
serve as a preliminary to the image of the 
Cross, and the redemption contained in it. As 
a confirmation of this, listen to the Lord saying, 
``As Moses exalted the serpent in the desert, so 
must you exalt the Son of Man, that every one 
believing in Him may not be lost, but may 
have eternal life.''

\switchgreek

\emph{Σκόλιον.} Σύνες, ὡς διὰ τὸ ἀπαγαγεῖν ἔφη τὸν λαὸν τῆς εἰδωλολατρείας
εὐόλισθον καὶ ἕτοιμον ὄντα πρὸς τοῦτο μὴ ποιεῖν πᾶν ὁμοίωμα ἐνομοθέτησε, καὶ
ὅτι εἰκὼν ἦν τοῦ πάθους τοῦ κυρίου ὁ ὑψωθεὶς ὄφις.

\switchenglish

\emph{Commentary.} 
Notice that His command 
ment not to make any graven thing was given 
to draw the people from idolatry, to which they 
were prone, and that the brazen serpent was an 
image of our Lord's suffering. 

\switchgreek

Ὅτι δὲ οὐ καινὸν ἐφεύρημα τὸ τῶν εἰκόνων, ἀλλὰ ἀρχαῖον καὶ τοῖς ἁγίοις καὶ
ἐκκρίτοις πατράσιν ἐγνωσμένον τε καὶ εἰθισμένον, ἄκουε·
Γέγραπται ἐν τῷ βίῳ Βασιλείου τοῦ μάκαρος τῷ δι’ Ἑλλαδίου, τοῦ αὐτοῦ μαθητοῦ
καὶ διαδόχου τῆς αὐτοῦ ἱεραρχίας, ὡς παρειστήκει ὁ ὅσιος τῇ τῆς δεσποίνης
εἰκόνι, ἐν ᾗ καὶ ὁ χαρακτὴρ ἐγέγραπτο Μερκουρίου τοῦ ἀοιδίμου μάρτυρος·
παρειστήκει δὲ ἐξαιτῶν τὴν Ἰουλιανοῦ τοῦ ἀθεωτάτου καὶ ἀποστάτου τυράννου
ἀναίρεσιν. Ἐξ ἧς εἰκόνος ἐμυήθη ταύτης τὴν ἀποκάλυψιν· ἑώρα γὰρ πρὸς μὲν βραχὺ
ἀφανῆ τὸν μάρτυρα, μετ’ οὐ πολὺ δὲ τὸ δόρυ ᾑμαγμένον κατέχοντα.

\switchenglish

Listen to what I am going to say as a proof 
that images are no new invention. It is an 
ancient practice well known to the best and 
foremost of the fathers. Elladios, the disciple 
of blessed Basil and his successor, says in his 
Life of Basil that the holy man was standing by 
the image of Our Lady, on which was painted 
also the likeness of Mercurius, the renowned 
martyr. He was standing by it asking for the 
removal of the impious apostate Julian, and he 
received this revelation from the statue. He 
saw the martyr vanish for a time, and then 
reappear, holding a bloody spear. 

\switchgreek

\subsubsection*{Ἐκ τοῦ βίου τοῦ Ἰωάννου τοῦ Χρυσοστόμου, οὕτως αὐτολεξεὶ γέγραπται.}

\switchenglish

\subsubsection*{Taken word for word from the Life of St. John Chrysostom.}

\switchgreek

Ἠγάπα δὲ ὁ μακάριος Ἰωάννης τὰς ἐπιστολὰς τοῦ σοφωτάτου Παύλου ἄγαν. Καὶ μετ’
ὀλίγα· Ἦν δὲ καὶ τὸ ἐκτύπωμα τοῦ αὐτοῦ ἀποστόλου Παύλου ἔχων ἐν εἰκόνι, ἔνθα
ἀνεπαύετο διὰ τὴν τοῦ σώματος ἀσθένειαν βραχύ τι· ἦν γὰρ πολυάγρυπνος ὑπὲρ
φύσιν. Καὶ ἡνίκα διήρχετο τὰς ἐπιστολὰς αὐτοῦ, ἐνητένιζεν αὐτῇ καὶ ὡς ἐπὶ
ζῶντος αὐτοῦ οὕτω προσεῖχεν αὐτῷ μακαρίζων αὐτόν, καὶ ὅλον αὑτοῦ τὸν λογισμὸν
πρὸς αὐτὸν εἶχε φανταζόμενος καὶ διὰ τῆς θεωρίας αὐτῷ ὁμιλῶν. Καὶ μεθ’ ἕτερα·
Ὡς δὲ ἐπαύσατο ὁ Πρόκλος λαλῶν, ἀτενίσας τῇ εἰκόνι τοῦ ἀποστόλου καὶ
θεασάμενος τὸν χαρακτῆρα αὐτοῦ ὅμοιον τοῦ ὀφθέντος αὐτῷ, βαλὼν μετάνοιαν τῷ
Ἰωάννῃ, εἶπε δακτυλοδεικτῶν τὴν εἰκόνα· Συγχώρησόν μοι, πάτερ. Ὃν εἶδον
λαλοῦντά σοι, ὅμοιός ἐστι τούτῳ· ὡς δὲ ὑπολαμβάνω, ὅτι καὶ αὐτός ἐστιν.

\switchenglish

Blessed John loved the epistles of St. Paul 
exceedingly. \ldots\ He had an image of the 
apostle in a place where he was wont to retire 
now and then on account of his physical weakness,
for he outdid nature in watchings and 
vigils. As he read through St. Paul's epistles, 
he had the image before him, and spoke to the 
apostle as if he had been present, praising him, 
and directing all his thoughts to him. \ldots
When Proclus had finished speaking, gazing 
intently at the image of the apostle, and recognising
the likeness to the man he had seen, 
saluting John, he said, pointing to the image : 
``Forgive me, father ; the man I saw talking to 
you is very like this statue. In fact, I should 
say he is the same.''

\switchgreek

\subsubsection*{Ἐν τῷ τῆς ὁσίας Εὐπραξίας βίῳ τὸν δεσποτικὸν δεδεῖχθαι αὐτῇ χαρακτῆρα
ἀναγέγραπται ὑπὸ τῆς προεστηκυίας τῆς ποίμνης.}

\switchenglish

\subsubsection*{In the life of St. Eupraxia we are told that her 
Superior showed her the likeness of our Lord.}

\switchgreek

Ἐν τῷ τῆς ὁσίας Μαρίας βίῳ τῆς Αἰγυπτίας γέγραπται τῇ εἰκόνι τῆς δεσποίνης
αὐτὴν εὔξασθαι καὶ ταύτην πρὸς ἐγγύην ἐξαιτήσασθαι καὶ οὕτω τυχεῖν τῆς εἰς τὸν
ναὸν εἰσόδου.

\switchenglish

We read in the life of St. Mary of Egypt 
that she prayed before the statue of Our Lady 
and besought her intercession, and so obtained 
leave to enter the Church.

\switchgreek

\subsubsection*{Ἐκ τοῦ Λειμωναρίου τοῦ ἁγίου πατρὸς ἡμῶν Σωφρονίου, ἀρχιεπισκόπου Ἱεροσολύμων.}

\switchenglish

\subsubsection*{From the \emph{Spiritual Meadow} of our holy father St. Sophronius, Archbishop of Jerusalem.}

\switchgreek

Ἔλεγεν ὁ ἀββᾶς Θεόδωρος ὁ Αἰλιώτης, ὅτι ἦν τις ἔγκλειστος εἰς τὸ ὄρος τῶν
ἐλαιῶν, ἀγωνιστὴς πάνυ· ἐπολέμει δὲ αὐτῷ ὁ δαίμων τῆς πορνείας. Ἐν μιᾷ οὖν, ὡς
ἐπέκειτο αὐτῷ σφοδρῶς, ἤρξατο ὁ γέρων ἀποδύρεσθαι καὶ λέγειν τῷ δαίμονι· «Ἕως
πότε οὐκ ἐνδίδως μοι; Ἀπόστα λοιπὸν ἀπ’ ἐμοῦ· συνεγήρασάς μοι.» Φαίνεται αὐτῷ
ὁ δαίμων ὀφθαλμοφανῶς λέγων· «Ὄμοσόν μοι, ὅτι οὐδενὶ λέγεις, ὃ μέλλω λέγειν
σοι, καὶ οὐκέτι σοι πολεμῶ.» Καὶ ὤμοσεν αὐτῷ ὁ γέρων, ὅτι «μὰ τὸν ἐνοικοῦντα
ἐν τοῖς ὑψίστοις· Οὐκ εἴπω τινί, ὅπερ λέγεις μοι.» Τότε λέγει αὐτῷ ὁ δαίμων·
«Μὴ προσκυνήσῃς ταύτῃ τῇ εἰκόνι, καὶ οὐκέτι σοι πολεμῶ.» Εἶχε δὲ ἡ εἰκὼν
ἐκτύπωμα τὴν δέσποιναν ἡμῶν τὴν ἁγίαν Μαρίαν τὴν θεοτόκον βαστάζουσαν τὸν
κύριον ἡμῶν Ἰησοῦν Χριστόν. Λέγει ὁ ἔγκλειστος τῷ δαίμονι· «Ἄφες, σκέψομαι.»
Τῇ οὖν ἐπαύριον δηλοῖ τῷ ἀββᾷ Θεοδώρῳ τῷ Αἰλιώτῃ οἰκοῦντι τότε ἐν τῇ λαύρᾳ
Φαράν, καὶ ἦλθε καὶ διηγεῖται ἅπαντα. Ὁ δὲ γέρων λέγει τῷ ἐγκλείστῳ· «Ὄντως,
ἀββᾶ, ἐνεπαίχθης, ὅτι ὤμοσας τῷ δαίμονι, πλὴν καλῶς ἐποίησας ἐξειπών. Συμφέρει
δέ σοι μὴ ἐᾶσαι εἰς τὴν πόλιν ταύτην πορνεῖον, εἰς ὃ μὴ εἰσέρχῃ, ἢ ἵνα ἀρνήσῃ
τὸ προσκυνεῖν τὸν κύριον καὶ θεὸν ἡμῶν Ἰησοῦν Χριστὸν μετὰ τῆς ἰδίας αὐτοῦ
μητρός.» Στηρίξας οὖν αὐτὸν καὶ ἐνδυναμώσας πλείοσι λόγοις ἀπῆλθεν εἰς τὸν
ἴδιον αὐτοῦ τόπον. Φαίνεται οὖν πάλιν ὁ δαίμων τῷ ἐγκλείστῳ καὶ λέγει αὐτῷ·
«Τί ἔνι, κακόγηρε; Οὐκ ὤμοσάς μοι, ὅτι οὐδενὶ λέγεις; Καὶ πῶς πάντα ἐξεῖπες τῷ
ἐλθόντι πρός σε; Λέγω σοι, κακόγηρε, ὡς ἐπίορκος ἔχεις κριθῆναι ἐν τῇ ἡμέρᾳ
τῆς κρίσεως.» Ἀπεκρίθη αὐτῷ ὁ ἔγκλειστος λέγων· «Ὅ τι μὲν ὤμοσα, ὤμοσα· καὶ ὅ
τι ἐπιώρκησα, οἶδα. Πλὴν τὸν ἐμὸν δεσπότην καὶ ποιητὴν ἐπιώρκησα· σοῦ δὲ οὐκ
ἀκούω.»

\switchenglish

% !!Need to find an English translation!! The edition just has "A testimony quoted from Sophronius is here suppreessed."
% Found one at https://orthocraft.net/Spiritual_Meadow/45%2C+THE+LIFE+OF+A+MONK%2C+A+RECLUSE+ON+THE+MOUNT+OF+OLIVES+AND+CONCERNING+THE+VENERATION%2C+OF+AN+ICON+OF+THE+MOST+HOLY+MOTHER+OF+GOD
One of the elders told us that Abba Theodore the Aeliote said that
there was a certain recluse on the Mount of Olives,
a great warrior against whom the demon of sexual desire waged battle.
One day when the demon attacked with vehemence,
the elder began to give up in despair and to say to the demon:
``How much longer are you not going to let me go?
Desist from growing old together with me!''
The demon appeared to him in visible form, saying:
``Swear to me that you will never reveal to anybody what
I am about to tell you and I will no longer wage war against you.''
The elder swore:
``By Him who dwelleth in the heavens I will not tell anybody what you say.''
The demon said to him:
``Desist from venerating this icon here
and I will call off my war against you.''
The icon in question bore the likeness of Our Lady Mary,
the holy Mother of God, carrying Our Lord Jesus Christ.
The recluse said to the demon: ``Let me go and think about it.''
The next day he sent for Abba Theodore the Aeliote
(the one who told us this story), for at that time he
was residing at the Lavra of Pharén. When Abba Theodore came, the recluse told
him all there was to tell and received this reply: ``In fact you were ensnared
when you swore, abba. But you are quite right to speak out. It were better for
you to leave no brothel in the town unentered than to diminish reverence from
Our Lord Jesus Christ and from his Mother.'' Abba Theodore strengthened and
comforted the recluse with many words and then returned to his own place. The
demon re-appeared to the recluse and said to him: ``What is this then, you
wicked old man? Did you not swear to me that you would not tell anybody? Why
then have you revealed everything to the man who came to see you? I tell you,
you wicked old man, you will be tried as an oath-breaker at the day of
judgement.'' The recluse answered: ``I know that I gave my oath and broke it,
but it was with my Lord and Creator that I broke faith; you I will not obey.''

\switchgreek

\emph{Σκόλιον.} Ὁρᾷς, ὅτι τὴν τῆς εἰκόνος προσκύνησιν τοῦ εἰκονιζομένου εἶπε, καὶ πόσον κακὸν τὸ ταύτῃ μὴ προσκυνεῖν, καὶ πῶς αὐτὸ τῆς πορνείας προετίμησεν ὁ δαίμων;

\switchenglish

\emph{Commentary.}
%!!Need to find an English translation!! The edition just has "A testimony quoted from Sophronius is here suppreessed."
% My own translation:
Do you understand that he called the worship of the icon the worship of
the one set forth in the icon, and how great an evil it is not to
worship the same, and how the demon preferred this to fornication?

\switchgreek

Πολλῶν τοίνυν ἀνέκαθεν ἱερέων τε καὶ βασιλέων χριστιανοῖς θεόθεν δεδωρημένων
σοφίᾳ τε καὶ θεοσεβείᾳ καὶ λόγῳ καὶ βίῳ διαπρεψάντων καὶ συνόδων πλείστων
γεγενημένων ἁγίων καὶ θεοπνεύστων πατέρων, τί μηδεὶς ταῦτα δρᾶν ἐπεχείρησεν;
Οὐκ ἀνεξόμεθα νέαν πίστιν διδάσκεσθαι. «Ἐκ γὰρ Σιὼν ἐξελεύσεται νόμος»,
προφητικῶς ἔφη τὸ πνεῦμα τὸ ἅγιον, «καὶ λόγος κυρίου ἐξ Ἱερουσαλήμ.» Οὐκ
ἀνεξόμεθα ἄλλοτε ἄλλα φρονεῖν καὶ καιροῖς μεταβάλλεσθαι καὶ τὴν πίστιν τοῖς
ἔξωθεν γέλωτα καὶ ἄθυρμα γίνεσθαι. Οὐκ ἀνεξόμεθα βασιλικῷ ὑποκύπτειν
θεσπίσματι τὴν ἐκ πατέρων πειρωμένῳ ἀνατρέπειν συνήθειαν· οὐ γὰρ εὐσεβῶν
βασιλέων ἀνατρέπειν ἐκκλησιαστικοὺς θεσμούς. Οὐ πατρικὰ ταῦτα· λῃστρικὰ γὰρ τὰ
βίᾳ καὶ οὐ πειθοῖ γινόμενα. Καὶ μάρτυς ἡ ἐν Ἐφέσῳ τὸ δεύτερον γεγενημένη
σύνοδος, μέχρι δεῦρο λῃστρικὴν δεδεγμένη τὴν ἐπωνυμίαν, βασιλικῇ χειρὶ
ἐκβιασθεῖσα, ὅτε Φλαβιανὸς ὁ μακάριος ἀπεκτέννυτο. Συνόδων ταῦτα, οὐ βασιλέων,
ὡς ὁ κύριος ἔφησεν· «Ὅπου συναχθῶσι δύο ἢ τρεῖς ἐπὶ τῷ ὀνόματί μου, ἐκεῖ εἰμι
ἐν μέσῳ αὐτῶν.» Οὐ βασιλεῦσι τοῦ δεσμεῖν καὶ λύειν τὴν ἐξουσίαν δέδωκεν ὁ
Χριστός, ἀλλ’ ἀποστόλοις καὶ τοῖς τούτων διαδόχοις καὶ ποιμέσι καὶ
διδασκάλοις. «Κἂν ἄγγελος», φησὶ Παῦλος ὁ ἀπόστολος, «εὐαγγελίσηται ὑμᾶς παρ’
ὃ παρελάβετε·» καὶ τὸ ἑξῆς σιωπήσομεν φειδοῖ τὴν ἐπιστροφὴν ἐκδεχόμενοι. Ἂν δὲ
ἴδωμεν τὴν διαστροφὴν ἀνεπίστροφον, τότε ἐπάξομεν καὶ τὸ λειπόμενον· ἀλλ’
ἀπηύχθω τοῦτο.

\switchenglish

In all the past array of Christian priests and 
kings, wise and pious, conspicuous by teaching 
and example, in so many councils of holy and 
inspired fathers, how is it that no one has 
pointed out these things? We are not advo 
cating a new faith. The law shall come out of 
Sion, the Holy Ghost said prophetically, and 
the word of the Lord from Jerusalem. We do 
not advocate one thing at one time, and another 
at another, nor that the faith should become a 
laughing-stock to those outside. We will not 
allow the king's commands to overturn the 
tradition handed down from the fathers. It is 
not for pious kings to overturn ecclesiastical 
boundaries. These are not patristic ways. 
Things done by force are impositions, and do 
not carry persuasion. A proof of this was given 
in the 2\textsuperscript{nd} Council of Ephesus, when a decree, 
which has never been recognised as valid, was 
enforced by the emperor's hand, and blessed 
Flavian was put to death. Councils do not 
belong to kings, as the Lord says: ``Wherever 
one or two are gathered together in My name, 
there I am in the midst of them.'' Christ did 
not give to kings the power to bind and to 
loose, but to the apostles, and to their successors
and pastors and teachers. ``If an angel 
were to teach you a different gospel to what 
you have received,'' St. Paul says\textemdash but we will 
be silent about what follows, in the hope of 
their conversion. And if we find the warning 
disregarded, which may God avert, we will 
then add the rest. Let us hope it will not be 
needed. 

\switchgreek

Εἴ τις εἰσέλθοι εἰς οἶκον, ἐν ᾧ ζωγράφος τοῖς τοίχοις ἱστορίαν Μωσέως καὶ
Φαραὼ ἔγραψε χρώμασιν, εἶτα ἐρωτήσει τυχὸν περὶ τῶν ὡς ἐπὶ ξηρὰν διοδευσάντων
τὴν θάλασσαν· «Τίνες οὗτοί εἰσιν;» Τί ἐρεῖς ἐρωτώμενος; Οὐχί· «Οἱ υἱοὶ
Ἰσραήλ;» «Τίς ὁ ῥάβδῳ παίων τὴν θάλασσαν;» Οὐχί· «Μωσῆς;» Οὕτω καί, εἴ τις τὸν
Χριστὸν εἰκονίσει σταυρούμενον, καὶ ἐρωτηθῇ· «Τίς οὗτός ἐστιν;» «Χριστὸς ὁ
θεός, ὁ δι’ ἡμᾶς σαρκωθείς», ἐρεῖ. Ναί, δέσποτα, προσ κυνῶ πάντα τὰ σὰ καὶ
πόθῳ ἐκκαεῖ περιπτύσσομαι, τὴν θεότητα, τὴν δύναμιν, τὴν ἀγαθότητα, τὸν περὶ
ἐμὲ ἔλεον, τὴν συγκατάβασιν, τὴν σάρκωσιν, τὴν σάρκα. Καὶ ὥσπερ ἅψασθαι
σιδήρου πεπυρακτωμένου δέδοικα οὐ διὰ τὴν τοῦ σιδήρου φύσιν, ἀλλὰ διὰ τὸ
ἡνωμένον αὐτῷ πῦρ, οὕτω τὴν σάρκα τὴν σὴν προσκυνῶ οὐ διὰ τὴν τῆς σαρκὸς
φύσιν, ἀλλὰ διὰ τὴν καθ’ ὑπόστασιν ἡνωμένην αὐτῇ θεότητα. Προσκυνοῦμέν σου τὰ
πάθη. Τίς εἶδε θάνατον προσκυνούμενον, τίς πάθη σεπτά; Ἀλλ’ ὄντως προσκυνοῦμεν
τὸν τοῦ θεοῦ μου σωματικὸν θάνατον καὶ τὰ σωτήρια πάθη, προσκυνοῦμέν σου τὴν
εἰκόνα· πάντα τὰ σὰ προσκυνοῦμεν, τοὺς θεράποντας, τοὺς φίλους καὶ πρὸ τούτων
τὴν Μητέρα τὴν Θεοτόκον. 

\switchenglish

If any one should enter a house and should 
see on the walls a history in painting of Moses 
and Aaron, perchance he might ask about the 
people who are walking across the sea as if 
it were dry land. ``Who are they?'' he asks. 
What would you say? ``Are they not the sons 
of Israel?'' ``Who is dividing the sea with his 
rod?'' Would you not say ``Moses''? So if a 
man makes an image of Christ crucified, and 
you are asked who he is, you reply, ``It is 
Christ our Lord, who became incarnate for us.''
Yes, O Lord, we adore all that belongs to 
Thee, and we take to our hearts Thy Godhead, 
Thy power and goodness, Thy mercy towards 
us, Thy condescension and Thy Incarnation. 
And as men fear touching red-hot iron, not 
because of the iron but because of the heat, 
so do we worship Thy flesh, not for the nature 
of flesh, but through the Godhead united to 
that flesh according to substance. We worship 
Thy sufferings. Who has ever known death 
worshipped, or suffering venerated? Yet we 
truly worship the physical death of our God 
and His saving sufferings. We adore Thy 
image and all that is Thine; Thy servants, 
Thy friends, and most of all Thy Mother, the 
Mother of God. 

\switchgreek

Δυσωποῦμεν δὲ καὶ τὸν τοῦ θεοῦ λαὸν, τὸ ἔθνος τὸ ἅγιον,
τῶν ἐκκλησιαστικῶν ἀνθέξεσθαι παραδόσεων·
ἡ γὰρ κατὰ μικρὸν τῶν παραδεδομένων ἀφαίρεσις ὡς ἐξ οἰκοδομῆς
λίθων θᾶττον ἅπασαν τὴν οἰκοδομὴν καταρρήγνυσιν. Εἴη δὲ ἡμᾶς ἑδραίους,
ἀκαμπεῖς, ἀκλονήτους διαμένειν, ἐπὶ πέτραν ὀχυρὰν ἐστηριγμένους, ἥτις ἐστὶν ὁ
Χριστός, ᾧ πρέπει δόξα, τιμὴ καὶ προσκύνησις σὺν τῷ Πατρὶ καὶ τῷ Πνεύματι νῦν
τε καὶ ἀεὶ καὶ εἰς τοὺς ἀπεράντους αἰῶνας τῶν αἰώνων. Ἀμήν.

\switchenglish

We beseech, therefore, the people of God, 
the faithful flock, to hold fast to the ecclesiastical traditions.
The gradual taking away 
of what has been handed down to us would 
be undermining the foundation stones, and 
would in no short time overthrow the whole 
structure. May we prove steadfast, unflinch 
ing, immovable, founded on the solid Rock 
which is Christ, to whom be praise, glory, and 
worship, with the Father and the Holy Ghost, 
now and for ever. Amen. 

\switchgreek

\section*{ΤΟΥ ΑΓΙΟΥ ΙΩΑΝΝΟΥ ΤΟΥ ΔΑΜΑΣΚΗΝΟΥ ΛΟΓΟΣ ΔΕΥΤΕΡΟΣ ΑΠΟΛΟΓΗΤΙΚΟΣ ΠΡΟΣ ΤΟΥΣ ΔΙΑΒΑΛΛΟΝΤΑΣ ΤΑΣ ΑΓΙΑΣ ΕΙΚΟΝΑΣ}

\switchenglish

\section*{SECOND APOLOGIA OF ST. JOHN DAMASCENE AGAINST THOSE WHO DECRY HOLY IMAGES}

\switchgreek

\lettrine{Δ}{ότε} συγγνώμην αἰτοῦντι,
δεσπόται μου, καὶ δέξασθε πληροφορίας λόγον παρ’ ἐμοῦ
τοῦ ἀχρείου καὶ ἐλαχίστου δούλου τῆς τοῦ θεοῦ ἐκκλησίας. Οὐ γὰρ δόξης ἕνεκεν ἢ
φανητιασμοῦ πρὸς τὸ λέγειν ὥρμησα–θεὸς μάρτυς–, ἀλλὰ ζήλῳ ἀληθείας· αὐτὸν γὰρ
μόνον ἐλπίδα σωτηρίας κέκτημαι καὶ σὺν αὐτῷ ὑπαντῆσαι τῷ δεσπότῃ Χριστῷ ἐλπίζω
καὶ εὔχομαι τοῦτον αὐτῷ προσφέρων, τῶν ἀτόπως μοι πεπλημμελημένων γενέσθαι
ἐξίλασμα. Ὁ μὲν γὰρ τὰ πέντε τάλαντα παρὰ τοῦ δεσπότου λαβὼν ἕτερα πέντε
κερδήσας προσήγαγε, καὶ ὁ τὰ δύο ἰσάριθμα δύο· ὁ δὲ τὸ ἓν εἰληφὼς κατορύξας
καὶ ἄκαρπον τοῦτο προσαγαγὼν πονηρὸς δοῦλος ἀκούσας εἰς τὸ ἐξώτερον
κατακέκριται σκότος. Ὅπερ ἐγὼ μὴ παθεῖν ὑφορώμενος τῷ δεσποτικῷ ὑπείκω
προστάγματι καὶ τὸ δεδομένον μοι παρ’ αὐτοῦ τοῦ λόγου τάλαντον ὑμῖν παρατίθημι
τοῖς φρονίμοις τραπεζίταις, ὅπως ἐλθὼν ὁ κύριός μου εὕροι πολυπλασιαζόμενον
καὶ τόκον καρποφοροῦν ψυχῶν καὶ δοῦλον πιστὸν εὑρὼν εἰσαγάγῃ με εἰς τὴν
πεποθημένην μοι γλυκυτάτην χαρὰν αὐτοῦ. Ἀλλὰ δότε μοι οὖς ἀκροάσεως καὶ τὰς
τραπέζας τῶν καρδιῶν ἀναπετάσαντες δέξασθέ μου τὸν λόγον καὶ εἰλικρινῶς
διακρίνατε τῶν λεγομένων τὴν δύναμιν. –Δεύτερον δὲ τοῦτον τὸν λόγον περὶ
εἰκόνων συνέταξα· τινὲς γὰρ τῶν τέκνων τῆς ἐκκλησίας ὑπέθεντό μοι τοῦτο
ποιῆσαι διὰ τὸ μὴ πάνυ εὐδιάγνωστον τοῖς πολλοῖς τὸν πρῶτον εἶναι. Ἀλλὰ καὶ ἐν
τούτῳ σύγγνωτέ μοι ὑπακοὴν ἐκπληρώσαντι.

\switchenglish

\lettrine{I}{ crave} your indulgence, my readers,
and ask you to receive the true statement 
of one who is an unprofitable servant, the least 
of all, in the Church of God. I have not been 
moved to speak by motives of vainglory, God 
is my witness, but by zeal for the truth. In 
this alone is my hope of salvation, and with it 
I trust and pray to go out to meet Christ our 
Lord, asking that it may be an expiation for 
my sins. The man who received five talents 
from his lord, brought other five which he had 
gained,\footnote{Matt. xxv, 20.} and the man with two, other two. 
The man who received one, and buried it, 
gave it back without interest, and being pro 
nounced a wicked servant, was banished into 
external darkness. Lest I should suffer in the 
same way, I obey God's commands, and with 
the talent of eloquence, which is His gift, I put 
before the wise among you a treasure table, so 
that when the Lord comes He may find me 
rich in souls, a faithful servant, whom He may 
take into that ineffable joy of His, which is my 
desire. Give me listening ears and willing
hearts. Receive my treatise, and ponder well 
the force of the arguments. This is the second 
part of my work on images. Certain children 
of the Church have urged me to do it because 
the first part was not sufficiently clear to all. 
Be indulgent with me on this account, for my 
obedience. 

\switchgreek

Ἔθος ἐστὶν τῷ πονηρῷ καὶ ἀρχεκάκῳ ὄφει, τῷ διαβόλῳ φημί, πολυτρόπως πολεμεῖν
τῷ κατ’ εἰκόνα θεοῦ πλαστουργηθέντι ἀνθρώπῳ καὶ διὰ τῶν ἐναντίων τὸν αὐτοῦ
κατεργάζεσθαι θάνατον. Εὐθὺς μὲν γὰρ ἐν ἀρχῇ ἐλπίδα καὶ ἐπιθυμίαν θεώσεως αὐτῷ
ἔσπειρε καὶ δι’ αὐτῆς εἰς τὸν τῶν ἀλόγων κατήγαγε θάνατον, οὐ μὴν ἀλλὰ καὶ
αἰσχραῖς καὶ ἀλόγοις ἡδοναῖς πολλάκις αὐτὸν ἐδελέασε. Πόσον δὲ τὸ διάμετρον
θεώσεως καὶ ἀλόγου ἐπιθυμίας. Ποτὲ μὲν εἰς ἀθεότητα ἤγαγε, καθώς φησιν ὁ
θεοπάτωρ Δαυίδ· «Εἶπεν ἄφρων ἐν καρδίᾳ αὐτοῦ· Οὐκ ἔστι θεός», ποτὲ δὲ εἰς
πολυθεΐαν· καὶ ποτὲ μὲν μηδὲ τῷ φύσει προσκυνεῖν θεῷ ἔπεισε, ποτὲ δὲ δαίμοσιν,
ἔτι δὲ οὐρανῷ τε καὶ γῇ, ἡλίῳ καὶ σελήνῃ καὶ ἀστράσι καὶ τῇ λοιπῇ κτίσει μέχρι
κνωδάλων καὶ ἑρπετῶν προσκυνεῖν παρεσκεύασεν. Ὁμοίως γάρ ἐστι χαλεπὸν καὶ τὸ
τοῖς τιμίοις τὴν ὀφειλομένην μὴ προσάγειν τιμὴν καὶ τοῖς ἀτίμοις τὴν μὴ
προσήκουσαν προσάπτειν δόξαν. Πάλιν τινὰς μὲν συνάναρχον τῷ θεῷ τὴν κακίαν
λέγειν ἐδίδαξε, τινὰς δὲ αἴτιον τῆς κακίας τὸν φύσει ἀγαθὸν θεὸν ὁμολογεῖν
ἐξηπάτησε. Καὶ οὓς μὲν μίαν φύσιν καὶ μίαν ὑπόστασιν τῆς θεότητος ἀφρόνως
λέγειν ἐπλάνησεν, οὓς δὲ τρεῖς φύσεις καὶ τρεῖς ὑποστάσεις ἀθέσμως σέβειν
ἐνόθευσε. Καὶ τισὶ μὲν μίαν ὑπόστασιν καὶ μίαν φύσιν τοῦ κυρίου ἡμῶν Ἰησοῦ
Χριστοῦ τοῦ ἑνὸς τῆς ἁγίας τριάδος, τισὶ δὲ δύο φύσεις καὶ δύο ὑποστάσεις τοῦ
αὐτοῦ δοξάζειν ὑπέθετο.

\switchenglish

The wicked serpent of old, Beloved, I mean 
the devil\textemdash is wont to wage war in many ways 
against man, who is made after God s image, 
and to work his destruction through opposition. 
In the very beginning he inspired man with the 
hope and desire of becoming a god, and through 
that desire he dragged man down to share the 
death of the brute creation. He has enticed 
man also by shameful and brutal pleasures. 
What a contrast between becoming a god and 
feeling brutal lust. And again, he led man 
into infidelity, as the royal\footnote{θεοπάτωρ} David 
says: ``The fool said in his heart there is no 
God.'' At one time he has brought man to 
worship too many gods, at another not even 
the true God, sometimes demons, and again, 
the heavens and the earth, the sun and moon 
and stars, and the rest of creation, wild beasts 
and reptiles. It is as bad to refuse due honour 
where honour is due, as to give it where it is 
not due. Again, he has taught some to call 
the uncreated god evil, and has deceived others 
by making them recognise God, who is good 
by nature, as the author of evil. Some he has 
deceived by the misconception of one nature 
and one substance of the Godhead ; some he 
has induced to honour three natures and three 
substances; some one substance in our Lord 
Jesus Christ, the Second Person of the Holy 
Trinity; some two natures and two substances. 

\switchgreek

Ἡ δὲ ἀλήθεια μέσην ὁδὸν βαδίζουσα πάντα ταῦτα ἀπαρνεῖται τὰ ἄτοπα καὶ διδάσκει
ἕνα θεὸν ὁμολογεῖν, μίαν φύσιν ἐν τρισὶν ὑποστάσεσι, Πατρὶ καὶ Υἱῷ καὶ ἁγίῳ
Πνεύματι. Τὴν δὲ κακίαν οὐκ οὐσίαν, ἀλλὰ συμβεβηκός φησιν, ἔννοιάν τινα καὶ
λόγον καὶ πρᾶξιν παρὰ τὸν νόμον τοῦ θεοῦ, ἐν τῷ ἐννοεῖσθαι καὶ λέγεσθαι καὶ
πράττεσθαι τὴν ὕπαρξιν ἔχουσαν καὶ ἅμα τῷ παύσασθαι ἀφανιζομένην. Ἔτι δὲ καὶ
τὸν ἕνα τῆς ἁγίας Τριάδος, τὸν Χριστόν, δύο φύσεις κηρύττει καὶ μίαν
ὑπόστασιν.

\switchenglish

But the truth, taking a middle course, 
sweeps away these misconceptions and teaches 
us to acknowledge one God, one nature in 
three persons, the Father, the 
Son, and the Holy Ghost. Evil is not a 
being,\footnote{
See St. Augustine, \emph{de Civitate Dei}: ``Nemo igitur 
qucerat efficientem causam malse voluntatis ; non enim 
est efficiens, sed deficiens, quia nee ilia efiectio sed de- 
fectio.'' (xii. c. vii)
}
but an accident, a certain conception, 
word, or deed against the law of God, taking 
its origin in this conception, speech, or doing, 
and ending with it. The truth proclaims 
also that in Christ, the second person of the 
Holy Trinity, there are two natures and one person.

\switchgreek

Ἀλλ’ ὁ τῆς ἀληθείας ἐχθρὸς καὶ τῆς σωτηρίας τῶν ἀνθρώπων πολέμιος ὅ ποτε
δαιμόνων καὶ ἀσεβῶν ἀνθρώπων καὶ πετεινῶν καὶ κνωδάλων καὶ ἑρπετῶν εἰκόνας
ποιεῖν καὶ ταύταις ὡς θεοῖς προσκυνεῖν οὐ μόνον τὰ ἔθνη, ἀλλὰ καὶ αὐτοὺς τοὺς
υἱοὺς Ἰσραὴλ πολλάκις πλανήσας νῦν εἰρήνην ἔχουσαν τὴν τοῦ Χριστοῦ ἐκκλησίαν
συνταράξαι σπουδάζει διὰ χειλέων ἀδίκων καὶ γλώσσης δολίας λόγοις θείοις τὴν
κακίαν παραρτύων καὶ ταύτης τὸ ἄσχημον καὶ σκοτεινὸν εἶδος ἐπικαλύπτειν
πειρώμενος καὶ τὰς καρδίας τῶν ἀστηρίκτων σαλεύειν ἐκ τῆς ἀληθοῦς καὶ
πατροπαραδότου συνηθείας· ἀνέστησαν γάρ τινες λέγοντες, ὡς οὐ δεῖ εἰκονίζειν
καὶ προτιθέναι εἰς θεωρίαν καὶ δόξαν καὶ θαῦμα καὶ ζῆλον τὰ τοῦ Χριστοῦ
σωτήρια θαύματά τε καὶ πάθη καὶ τὰς τῶν ἁγίων ἀνδραγαθίας κατὰ τοῦ διαβόλου.
Καὶ τίς ἔχων γνῶσιν θείαν καὶ σύνεσιν πνευματικὴν οὐκ ἐπιγινώσκει, ὅτι ὑποβολὴ
τοῦ διαβόλου ἐστίν; Οὐ θέλει γὰρ τὴν ἧτταν καὶ τὴν αἰσχύνην αὐτοῦ
δημοσιεύεσθαι οὐδὲ τὴν τοῦ θεοῦ καὶ τῶν ἁγίων αὐτοῦ δόξαν ἀνάγραπτον γίνεσθαι.

\switchenglish

Now, the devil, the enemy of the 
truth and of man's salvation, in suggesting 
that images of corruptible man, and of birds 
and beasts and reptiles, should be made and 
worshipped as gods, has often led astray not 
only heathens but the children of Israel. In 
these days he is eager to trouble the peace 
of Christ's Church through false and lying 
tongues, using divine words in favour of what 
is evil, and striving to disguise his wicked 
intent, and drawing the unstable away from 
true and patristic custom. Some have risen 
up and said that it was wrong to represent 
and set forth publicly for adoration the saving 
wounds of Christ, and the combats of the 
saints against the devil. Who with a know 
ledge of divine things and a spiritual sense 
does not perceive in this a deception of the 
devil? He is unwilling that his shame 
should be known and that the glory of God 
and of His saints should be published.

\switchgreek

Εἰ μὲν γὰρ τοῦ θεοῦ τοῦ ἀοράτου εἰκόνα ἐποιοῦμεν, ὄντως ἡμαρτάνομεν· ἀδύνατον
γὰρ τὸ ἀσώματον καὶ ἀσχη μάτιστον καὶ ἀόρατον καὶ ἀπερίγραπτον εἰκονισθῆναι.
Καὶ πάλιν· εἰ ἐποιοῦμεν εἰκόνας ἀνθρώπων καὶ ταύτας θεοὺς ἡγούμεθα καὶ ὡς
θεοῖς λατρεύομεν, ὄντως ἠσεβοῦμεν. Ἀλλ’ οὐδὲν τούτων ποιοῦμεν. Θεοῦ γὰρ
σαρκωθέντος καὶ ὀφθέντος ἐπὶ τῆς γῆς σαρκὶ καὶ ἀνθρώποις συναναστραφέντος δι’
ἄφατον ἀγαθότητα καὶ φύσιν καὶ πάχος καὶ σχῆμα καὶ χρῶμα σαρκὸς ἀναλαβόντος
τὴν εἰκόνα ποιοῦντες οὐ σφαλλόμεθα· ποθοῦμεν γὰρ αὐτοῦ ἰδεῖν τὸν χαρακτῆρα·
ὡς γάρ φησιν ὁ θεῖος ἀπόστολος· «Ἐν ἐσόπτρῳ καὶ ἐν αἰνίγματι νῦν βλέπομεν».
Καὶ ἡ εἰκὼν δὲ ἔσοπτρόν ἐστι καὶ αἴνιγμα ἁρμόζον τῇ τοῦ σώματος ἡμῶν
παχύτητι· πολλὰ γὰρ κάμνων ὁ νοῦς οὐ δύναται ἐκβῆναι τὰ σωματικά.

\switchenglish

If we made an image of the invisible God, 
we should in truth do wrong. For it is 
impossible to make a statue of one who is 
without body, invisible, boundless, and form 
less. Again, if we made statues of men, and 
held them to be gods, worshipping them as 
such, we should be most impious. But we 
do neither. For in making the image of God, 
who became incarnate and visible on earth, 
a man amongst men through His unspeakable 
goodness, taking upon Him shape and form 
and flesh, we are not misled. We long to see 
what He was like. As the divine apostle 
says, We see now in a glass, darkly. The 
image, too, is a dark glass, according to the 
denseness of our bodies. The mind, in much 
travail, cannot rid itself of bodily things. 

\switchgreek

Ὢ ἀπὸ σοῦ, φθονερὲ διάβολε, φθονεῖς ἡμῖν ἰδεῖν τὸ τοῦ δεσπότου ἡμῶν ὁμοίωμα
καὶ δι’ αὐτοῦ ἁγιασθῆναι καὶ ἰδεῖν αὐτοῦ τὰ σωτήρια πάθη καὶ θαυμάζειν αὐτοῦ
τὴν συγκατάβασιν καὶ θεωρεῖν αὐτοῦ τὰ θαύματα καὶ ἐπιγινώσκειν καὶ δοξάζειν
αὐτοῦ τὴν τῆς θεότητος δύναμιν. Φθονεῖς τοῖς ἁγίοις τῆς παρὰ θεοῦ δεδομένης
αὐτοῖς τιμῆς. Οὐ θέλεις ὁρᾶν ἡμᾶς τὴν αὐτῶν δόξαν ἀνάγραπτον καὶ ζηλωτὰς
γενέσθαι τῆς αὐτῶν ἀνδρείας καὶ πίστεως. Οὐ φέρεις τὴν ἐκ τῆς εἰς αὐτοὺς
πίστεως προσγενομένην ἡμῖν σωματικήν τε καὶ ψυχικὴν ὠφέλειαν. Οὐ πειθόμεθά
σοι, δαῖμον φθονερὲ καὶ μισάνθρωπε. Ἀκούσατε, λαοί, φυλαί, γλῶσσαι, ἄνδρες,
γυναῖκες, παῖδες, πρεσβύται, νεανίσκοι καὶ νήπια, τὸ ἔθνος τῶν χριστιανῶν τὸ
ἅγιον· Εἴ τις εὐαγγελίζεται ὑμᾶς, παρ’ ὃ παρέλαβεν ἡ καθολικὴ ἐκκλησία παρὰ
τῶν ἁγίων ἀποστόλων πατέρων τε καὶ συνόδων καὶ μέχρι τοῦ νῦν διεφύλαξε, μὴ
ἀκούσητε αὐτοῦ μηδὲ δέξησθε τὴν συμβουλὴν τοῦ ὄφεως, ὡς ἐδέξατο Εὔα καὶ
ἐτρύγησε θάνατον. Κἂν ἄγγελος, κἂν βασιλεὺς εὐαγγελίζηται ὑμᾶς, παρ’ ὃ
παρελάβετε, κλείσατε τὰς ἀκοάς· ὀκνῶ γὰρ τέως εἰπεῖν, ὡς ἔφη ὁ θεῖος
ἀπόστολος· «Ἀνάθεμα ἔστω», ἐκδεχόμενος τὴν διόρθωσιν.

\switchenglish

Shame upon you, wicked devil, for grudging 
us the sight of our Lord's likeness and our 
sanctification through it. You would not have 
us gaze at His saving sufferings nor wonder 
at His condescension, neither contemplate His 
miracles nor praise His almighty power. You 
grudge the saints the honour God gives to 
them. You would not have us see their glory 
put on record, nor allow us to become imitators 
of their fortitude and faith. We will not 
obey your suggestions, wicked and man-hating 
devil. Listen to me, people of all nations, 
men, women, and children, all of you who bear 
the Christian name: If any one preach to 
you something contrary to what the Catholic 
Church has received from the holy apostles 
and fathers and councils, and has kept down 
to the present day, do not heed him. Do not 
receive the serpent's counsel, as Eve did, to 
whom it was death. If an angel or an 
emperor teaches you anything contrary to 
what you have received, shut your ears. I 
have refrained so far from saying, as the holy 
apostle said, ``Let him be anathema,'' in the 
hope of amendment. 

\switchgreek

Ἀλλὰ λέγουσιν οἱ μὴ ἐρευνῶντες τὸν νοῦν τῆς Γραφῆς, ὅτι εἶπεν ὁ θεὸς διὰ
Μωσέως τοῦ νομοθέτου· «Οὐ ποιήσεις πᾶν ὁμοίωμα, ὅσα ἐν τῷ οὐρανῷ καὶ ὅσα ἐν
τῇ γῇ.» καὶ διὰ Δαυὶδ τοῦ προφήτου· «Αἰσχυνθήτωσαν πάντες οἱ προσκυνοῦντες
τοῖς γλυπτοῖς οἱ ἐγκαυχώμενοι ἐν τοῖς εἰδώλοις αὐτῶν», καὶ τοιαῦτα πολλὰ
ἕτερα. Ὅσα γὰρ ἂν ἐκ τῆς θείας Γραφῆς καὶ ἐκ τῶν ἁγίων πατέρων προενέγκωσι,
τῆς αὐτῆς ἐννοίας εἰσί.

\switchenglish

But say those who do not enter into the 
mind of Scripture, God said, through Moses 
the law-giver: ``Thou shalt not make to 
thyself the likeness of any thing that is in 
heaven above, or in the earth beneath''; and 
through the prophet David: ``Let them be all 
confounded that adore graven things, and 
that glory in their idols,'' and many similar 
passages. Whatever they have quoted from 
Holy Scripture and the fathers is to the same 
intent. 

\switchgreek

Τί οὖν ἡμεῖς φαμεν πρὸς ταῦτα; Τί ἄλλο εἰ μὴ τὸ ὑπὸ τοῦ κυρίου τοῖς Ἰουδαίοις
εἰρημένον «Ἐρευνᾶτε τὰς Γραφάς»;

\switchenglish

Now, what shall we say to these things ? 
What, if not that which God spoke to the 
Jews, ``Search the Scriptures.''

\switchgreek

Καλὴ γὰρ ἡ τῶν Γραφῶν ἔρευνα. Ἀλλ’ ἐνταῦθα νουνεχῶς προσέχετε. Ἀδύνατον, ὦ
ἀγαπητοί, Θεὸν ψεύσασθαι· εἷς γάρ ἐστι Θεός, εἷς νομοδότης Παλαιᾶς καὶ Καινῆς
Διαθήκης ὁ πάλαι λαλήσας πολυμερῶς καὶ πολυτρόπως τοῖς πατράσιν ἐν τοῖς
προφήταις καὶ ἐπ’ ἐσχάτων τῶν χρόνων ἐν τῷ μονογενεῖ αὐτοῦ Υἱῷ. Προσέχετε
τοίνυν μετὰ ἀκριβείας. Οὐκ ἐμὸς ὁ λόγος ἐστί. Τὸ Πνεῦμα τὸ ἅγιον διὰ Παύλου
τοῦ ἁγίου ἀποστόλου ἀπεφήνατο· «Πολυμερῶς καὶ πολυτρόπως πάλαι ὁ θεὸς λαλήσας
τοῖς πατράσιν ἐν τοῖς προφήταις.»
Ὅρα, ὅτι πολυμερῶς καὶ πολυτρόπως ἐλάλησεν ὁ θεός. Ὥσπερ γὰρ ἐπιστήμων ἰατρὸς
οὐ τὸ αὐτὸ εἶδος πᾶσι δίδωσιν πάντοτε, ἀλλ’ ἑκάστῳ τὸ ἐπιτήδειον παρέχει καὶ
πρόσφορον φάρμακον διακρίνων καὶ χώραν καὶ νόσον καὶ ὥραν τουτέστι καιρὸν καὶ
ἕξιν καὶ ἡλικίαν· καὶ τῷ μὲν νηπίῳ ἕτερον, τῷ δὲ τελείῳ κατὰ τὴν ἡλικίαν
ἕτερον· ἄλλο τῷ ἀσθενεῖ καὶ ἄλλο τῷ ὑγιαίνοντι, καὶ ἑκάστῳ τῶν ἀσθενούντων οὐ
τὸ αὐτό, ἀλλὰ πρὸς τὴν ἕξιν καὶ τὴν νόσον· καὶ ἄλλο τῷ θέρει καὶ τῷ χειμῶνι
ἕτερον, μετοπώρῳ τε καὶ ἔαρι, καὶ ἐν ἑκάστῳ τόπῳ κατὰ τὴν τοῦ τόπου
ἐπιτηδειότητα. Οὕτω καὶ ὁ ἄριστος τῶν ψυχῶν ἰατρὸς τοῖς ἔτι νηπίοις καὶ
ἀρρωστοῦσι τὴν πρὸς εἰδωλολατρείαν νόσον καὶ τὰ εἴδωλα θεοὺς ἡγουμένοις καὶ
ὡς θεοῖς αὐτοῖς προσκυνοῦσι καὶ ἀθετοῦσιν τὴν τοῦ θεοῦ προσκύνησιν καὶ τὴν
αὐτοῦ δόξαν τῇ κτίσει προσάγουσιν ἀπηγόρευσε τοῦ ποιεῖν εἰκόνας.

\switchenglish

It is good to examine the Scriptures, but 
let your mind be enlightened from the search. 
It is impossible, Beloved, that God should not 
speak truth. There is one God, one Law 
giver of the old and new dispensation, who 
spoke of old in many ways to the patriarchs 
through the prophets, and in these latter times 
through His only begotten Son. Apply your 
mind with discernment. It is not I who am 
speaking. The Holy Ghost declared by the 
holy apostle St Paul that God spoke of old 
in many different ways to the patriarchs 
through the prophets. Note, \emph{in many different ways}.
A skilful doctor does not invariably 
prescribe for all alike, but for each according 
to his state, taking into consideration climate 
and complaint, season and age, giving one 
remedy to a child, another to a grown man, 
according to his age ; one thing to a weak 
patient, another to a strong ; and to each 
sufferer the right thing for his state and 
malady : one thing in the summer, another in 
the winter, another in the spring or autumn, 
and in each place according to its requirements. 
So in the same way the good Physician of 
souls prescribed for those who were still 
children and inclined to the sickness of idolatry, 
holding idols to be gods, and worshipping 
them as such, neglecting the worship of God, 
and preferring the creature to His glory. 
He charged them not to do this. 

\switchgreek

Θεοῦ μὲν γὰρ τοῦ ἀσωμάτου καὶ ἀοράτου καὶ ἀύλου
καὶ μήτε σχῆμα μήτε περιγραφὴν μήτε
κατάληψιν ἔχοντος ἀδύνατον ποιεῖν εἰκόνα· πῶς γὰρ τὸ μὴ ὁραθὲν
εἰκονισθήσεται; «Θεὸν δὲ οὐδεὶς ἑώρακε πώποτε· ὁ μονογενὴς υἱὸς ὁ ὢν ἐν τοῖς
κόλποις τοῦ Πατρός, αὐτὸς ἐξηγήσατο.» καὶ «οὐδεὶς ὄψεται τὸ πρόσωπόν μου καὶ
ζήσεται», ἔφη ὁ Θεός.

\switchenglish

It is impossible to make an image of God, 
who is a pure spirit, invisible, boundless, having 
neither form nor circumscription. How can 
we make an image of what is invisible? ``No 
man hath seen God at any time; the only-begotten
Son who is in the bosom of the 
Father, He hath declared Him.'' And again, 
``No one shall see My face and live, saith the Lord.'' 

\switchgreek

Ὅτι δὲ τοῖς εἰδώλοις ὡς θεοῖς προσεκύνουν, ἄκουε, τί φησιν ἡ γραφὴ ἐν τῇ
ἐξόδῳ τῶν υἱῶν Ἰσραήλ, ὅτε ἀνῆλθε Μωσῆς εἰς τὸ ὄρος Σινᾶ καὶ ἐχρόνισε
προσεδρεύων τῷ θεῷ λαβεῖν τὸν νόμον, ὅτε ἐπανέστη ὁ ἀγνώμων λαὸς Ἀαρὼν τῷ τοῦ
θεοῦ θεράποντι λέγοντες· «Ποίησον ἡμῖν θεούς, οἳ προπορεύσονται ἡμῶν· ὁ γὰρ
ἄνθρωπος οὗτος ὁ Μωσῆς, οὐκ οἴδαμεν, τί γέγονεν αὐτῷ.» Εἶτα, ὅτε περι είλοντο
τὸν κόσμον τῶν γυναικῶν αὐτῶν καὶ ἐχώνευσαν, ἔφαγον καὶ ἔπιον καὶ ἐμεθύσθησαν
ὑπό τε τοῦ οἴνου καὶ τῆς πλάνης καὶ ἤρξαντο παίζειν ἐν ἀφροσύνῃ λέγοντες·
«Οὗτοι οἱ θεοί σου, Ἰσραήλ.»
% % These next lines are quite different both in the English translation
% % and the Migne edition. Greek website version:
% Ὁρᾷς, ὅτι θεοὺς εἶχον τὰ εἴδωλα· οὐ γὰρ ἐποίησαν
% Διὸς εἴδωλον ἢ τοῦδε ἢ τοῦδε, ἀλλ’, ὡς ἔτυχεν, ἔδωκαν τὸν χρυσὸν ἐπὶ τὸ
% ποιῆσαι εἴδωλον, ὅπερ ἂν τύχῃ, καὶ ἀνῆλθεν ἐκτύπωμα βουκράνου.
% Ὥστε αὐτὰ τὰ χωνευτὰ θεοὺς εἶχον καὶ τούτοις ὡς θεοῖς προσεκύνουν,
% ἅτινα δαιμόνων ἦσαν κατοικητήρια.
% % English:
% Do you see that they made gods of idols who were demons,
Ὁρᾷς, ὅτι θεοὺς εἶχον τὰ εἴδωλα, ἅτινα δαιμόνων ἦσαν κατοικητήρια;
καὶ ὅτι τῇ κτίσει ἐλάτρευον παρὰ τὸν Κτίσαντα;
Καθώς φησὶ καὶ ὁ θεῖος ἀπόστολος·
«Οἵτινες ἤλλαξαν τὴν δόξαν τοῦ ἀφθάρτου Θεοῦ ἐν ὁμοιώματι
φθαρτοῦ ἀνθρώπου καὶ πετεινῶν καὶ τετραπόδων καὶ ἑρπετῶν
καὶ ἐλάτρευσαν τῇ κτίσει παρὰ τὸν Κτίσαντα.»
Τούτου χάριν ἀπηγόρευσεν ὁ θεὸς ποιεῖν πᾶν ὁμοίωμα,
καθὼς Μωσῆς ἐν τῷ Δευτερονομίῳ φησί·
«Καὶ ἐλάλησε κύριος πρὸς ὑμᾶς ἐκ μέσου τοῦ πυρός·
φωνὴν ῥημάτων ὑμεῖς ἠκούσατε καὶ ὁμοίωμα οὐκ εἴδετε, ἀλλ’ ἢ φωνήν»
καὶ μετ’ ὀλίγα «καὶ φυλάξασθε σφόδρα τὰς ψυχὰς ὑμῶν, ὅτι ὁμοίωμα οὐκ εἴδετε
ἐν τῇ ἡμέρᾳ, ᾗ ἐλάλησε κύριος πρὸς ὑμᾶς ἐν Χωρὴβ ἐν τῷ ὄρει ἐκ μέσου τοῦ
πυρός, μήποτε ἀνομήσητε καὶ ποιήσητε ὑμῖν ἑαυτοῖς γλυπτὸν ὁμοίωμα, πᾶσαν
εἰκόνα, ὁμοίωμα ἀρσενικοῦ ἢ θηλυκοῦ, ὁμοίωμα παντὸς κτήνους τῶν ὄντων ἐπὶ τῆς
γῆς, ὁμοίωμα παντὸς ὀρνέου πτερωτοῦ»
καὶ τὰ ἑξῆς.
Καὶ μετὰ βραχέα «μήποτε ἀναβλέψας εἰς τὸν οὐρανὸν καὶ ἰδὼν τὸν ἥλιον καὶ τὴν
σελήνην καὶ τοὺς ἀστέρας καὶ πάντα τὸν κόσμον τοῦ οὐρανοῦ, πλανηθεὶς
προσκυνήσῃς αὐτοῖς καὶ λατρεύσῃς αὐτοῖς.»
Ὁρᾷς, ὡς εἷς ἐστιν ὁ σκοπός, ὥστε μὴ λατρεῦσαι τῇ κτίσει παρὰ τὸν Κτίσαντα
μηδὲ προσάγειν τὴν τῆς λατρείας προσκύνησιν, ἀλλ’ ἢ μόνῳ τῷ Δημιουργῷ. Διὸ
πανταχῆ συνάπτει τῇ προσκυνήσει τὴν λατρείαν.
Καὶ μετ’ ὀλίγα· «Οὐκ ἔσονταί σοι θεοὶ ἕτεροι πλὴν ἐμοῦ.
Οὐ ποιήσεις σεαυτῷ γλυπτὸν οὐδὲ πᾶν ὁμοίωμα.»
Καὶ πάλιν· «Καὶ θεοὺς χωνευτοὺς οὐ ποιήσεις σεαυτῷ.»
Ὁρᾷς, ὡς τῆς εἰδωλολατρείας ἕνεκα ἀπαγορεύει τὴν εἰκονογραφίαν καὶ ὅτι
ἀδύνατον εἰκονίζεσθαι Θεὸν, τὸν ἀσώματον, καὶ ἀόρατον, καὶ ἀπερίγραπτον;
«Οὐ γὰρ εἶδος αὐτοῦ», φησίν, «ἑωράκατε.»
Καθὰ καὶ Παῦλος ἑστὼς ἐν μέσῳ τοῦ Ἀρείου πάγου φησίν· «Γένος οὖν ὑπάρχοντες
τοῦ Θεοῦ οὐκ ὀφείλομεν νομίζειν χρυσίῳ ἢ ἀργυρίῳ ἢ λίθῳ, χαράγματι τέχνης καὶ
ἐνθυμήσεως ἀνθρώπου, τὸ Θεῖον εἶναι ὅμοιον.»

\switchenglish

That they \emph{did} worship idols there is no 
doubt from what the Scripture says about the 
going out of the children of Israel, when Moses 
went up to Mount Sinai, and persevered in 
prayer to God. Whilst receiving the law, the 
ungrateful people rose against Aaron, the 
priest of God, saying: ``Make us gods who 
may go before us.'' For as to Moses, we know 
not what has befallen him. Then, when they 
had looked over the trinkets of their wives, and 
brought them together, they ate and drank, 
and were inebriated with wine and madness, 
and began to make merry, saying in their 
foolishness, ``These are thy gods, O Israel.''
Do you see that they made gods of idols who 
were demons, and that they worshipped the 
creature instead of the Creator? As the holy 
apostle says: ``They changed the glory of the 
incorruptible God into the likeness of the 
image of a corruptible man and of birds, and 
of four-footed beasts, and of creeping things, 
and served the creature rather than the 
Creator.'' On this account God forbade them to 
make any graven image, as Moses says in 
Deuteronomy: ``And the Lord spoke to you 
from the midst of the fire; you heard the voice 
of His words, but you saw not any form at 
all.'' And a little further on: ``Keep therefore 
your souls carefully ; you saw not any similitude
in the day that the Lord God spoke to 
you in Horeb, from the midst of the fire, lest 
perhaps being deceived you might make you 
a graven similitude or image of male or female, 
the similitude of any beasts that are upon the 
earth, or of birds that fly under heaven.''
And again: ``Lest perhaps lifting up thy eyes to 
heaven, thou see the sun and the moon, and 
all the stars of heaven, and being deceived by 
error, thou adore and serve them.''
You see the one object in view is that the creature 
should not be worshipped instead of the 
Creator, and that the worship of latreia should 
be given to God alone. Thus in every case 
when he speaks of worship he means latreia. 
Again: ``Thou shalt not have strange gods in 
my sight; thou shalt not make to thyself a 
graven thing nor any likeness.''
Again: ``Thou shalt not make to thyself gods of metal.''
You see that He forbids image-making on 
account of idolatry, and that it is impossible 
to make an image of God, who is a Spirit, 
invisible, and uncircumscribed. ``You have 
not seen His likeness,'' He says; and St. Paul, 
standing in the midst of the Areopagus, says: 
``Being therefore the offspring of God, we must 
not suppose the divinity to be like unto gold, 
or silver, or stone, the graving of art, a device 
of man.''

\switchgreek

Καὶ ὅτι ταῦτα οὕτως ἔχει, ἄκουε· «Οὐ ποιήσεις σεαυτῷ», φησί, «γλυπτὸν οὐδὲ πᾶν
ὁμοίωμα.» Ταῦτα τοῦ θεοῦ προστάξαντος «ἐποίησαν», φησί, «τὸ καταπέτασμα τῆς
σκηνῆς τοῦ μαρτυρίου ἐξ ὑακίνθου καὶ πορφύρας καὶ κοκκίνου νενησμένου καὶ
βύσσου κεκλωσμένης, ἔργον ὑφαντὸν Χερουβίμ», καὶ «ἐποίησαν τὸ ἱλαστήριον
ἄνωθεν τῆς κιβωτοῦ ἐκ χρυσίου καθαροῦ καὶ τοὺς δύο χερουβίμ.» Τί ποιεῖς, ὦ
Μωσῆ; Σὺ λέγεις· «Οὐ ποιήσεις σεαυτῷ γλυπτὸν οὐδὲ πᾶν ὁμοίωμα», καὶ σὺ
καταπέτασμα κατασκευάζεις, «ἔργον ὑφαντὸν Χερουβὶμ» καὶ «δύο χερουβὶμ ἐκ
χρυσίου καθαροῦ»; Ἀλλ’ ἄκουε, τί πρός σε ὁ θεράπων τοῦ Θεοῦ Μωσῆς τοῖς
πράγμασιν ἀντιφθέγγεται. Ὦ τυφλοὶ καὶ μωροί, σύνετε τῶν λεγομένων τὴν δύναμιν
«καὶ φυλάξασθε σφόδρα τὰς ψυχὰς ὑμῶν.» Εἶπον, «ὅτι ὁμοίωμα οὐκ εἴδετε ἐν τῇ
ἡμέρᾳ, ᾗ ἐλάλησε κύριος πρὸς ὑμᾶς ἐν Χωρὴβ ἐν τῷ ὄρει ἐκ μέσου τοῦ πυρός,
μήποτε ἀνομήσητε καὶ ποιήσητε ὑμῖν ἑαυτοῖς γλυπτὸν ὁμοίωμα, πᾶσαν εἰκόνα», καὶ
«θεοὺς χωνευτοὺς οὐ ποιήσεις σεαυτῷ». Οὐκ εἶπον· Οὐ ποιήσεις εἰκόνα Χερουβὶμ
ὡς δούλων παρεστηκότων τῷ ἱλαστηρίῳ, ἀλλ’ «οὐ ποιήσεις σεαυτῷ θεοὺς
χωνευτούς», καὶ «οὐ ποιήσεις πᾶν ὁμοίωμα» ὡς Θεοῦ οὐδ’ οὐ μὴ λατρεύσῃς «τῇ
κτίσει παρὰ τὸν Κτίσαντα». Ὁμοίωμα μὲν οὖν Θεοῦ οὐκ ἐποίησα οὐδὲ μὴν ἑτέρου
τινὸς ὡς θεοῦ οὐδὲ «ἐλάτρευσα τῇ κτίσει παρὰ τὸν Κτίσαντα».

\switchenglish

Listen again that it is so. Thou shalt not 
make to thyself any brazen thing nor any 
likeness. These things, he says, they made 
by God's commandment a hanging of violet, 
purple, scarlet, and fine twisted linen in the 
entrance of the tabernacle, and the cherubim 
in woven work. And they made also the 
propitiatory, that is, the oracle of the purest 
gold, and the two cherubim. What will you 
say to this, O Moses? You say, thou shalt 
not make to thyself any graven thing nor any 
likeness, and you yourself fashion cherubim of 
woven work, and two cherubim of pure gold. 
Listen to the answer of God's servant Moses: 
``You blind and foolish people, mark the force 
of what is said, and keep your souls carefully. 
I said that you had seen no likeness on the 
day when the Lord spoke to you on Mount 
Horeb, in the midst of the fire, lest you should 
sin against the law and make for yourselves a 
brazen likeness: thou shalt not make any 
image or gods of metal. I never said thou 
shalt not make the image of cherubim in 
adoration before the propitiatory. What I 
said was: Thou shalt not make to thyself 
gods of metal, and thou shalt not make any 
likeness as of God, nor shalt thou adore 
the creature instead of the Creator, nor 
any creature whatsoever as God, nor have 
I served the creature rather than the Creator. ''

\switchgreek

Εἶδες, πῶς ἀνεφάνη ὁ σκοπὸς τῆς Γραφῆς τοῖς συνετῶς ἐρευνῶσι; Δεῖ γὰρ
γινώσκειν, ἀγαπητοί, ὅτι ἐν παντὶ πράγματι ἡ ἀλήθεια ζητεῖται καὶ τὸ ψεῦδος
καὶ ὁ σκοπὸς τοῦ ποιοῦντος, εἰ καλός ἐστιν ἢ κακός. Ἐν μὲν γὰρ τῷ Εὐαγγελίῳ
καὶ Θεὸς καὶ ἄγγελος καὶ ἄνθρωπος καὶ οὐρανὸς καὶ γῆ καὶ ὕδωρ καὶ πῦρ καὶ ἀὴρ
καὶ ἥλιος καὶ σελήνη καὶ ἄστρα καὶ φῶς καὶ σκότος καὶ σατανᾶς καὶ δαίμονες καὶ
ὄφεις καὶ σκορπίοι καὶ θάνατος καὶ ᾅδης καὶ ἀρεταὶ καὶ κακίαι καὶ πάντα καλά
τε καὶ κακά εἰσιν ἐγγεγραμμένα. Ἀλλ’ ὅμως ἐπειδὴ πάντα τὰ περὶ αὐτῶν λεγόμενα
ἀληθῆ εἰσι καὶ ὁ σκοπὸς πρὸς δόξαν Θεοῦ ἐστι καὶ τῶν ὑπ’ αὐτοῦ δοξαζομένων
ἁγίων καὶ σωτηρίαν ἡμῶν καὶ καθαίρεσιν καὶ αἰσχύνην τοῦ διαβόλου καὶ τῶν
δαιμόνων αὐτοῦ, προσκυνοῦμεν καὶ περιπτυσσόμεθα καὶ καταφιλοῦμεν καὶ ὀφθαλμοῖς
καὶ χείλεσι καὶ καρδίᾳ ἀσπαζόμεθα, ὁμοίως καὶ πᾶσαν τὴν παλαιὰν καὶ καινὴν
διαθήκην τούς τε λόγους τῶν ἁγίων καὶ ἐκκρίτων πατέρων, τὴν δὲ αἰσχρὰν καὶ
μυσαρὰν καὶ ἀκάθαρτον γραφὴν τῶν καταράτων Μανιχαίων τε καὶ Ἑλλήνων καὶ τῶν
λοιπῶν αἱρέσεων ὡς ψευδῆ καὶ μάταια περιέχουσαν καὶ πρὸς δόξαν τοῦ διαβόλου
καὶ τῶν δαιμόνων αὐτοῦ καὶ χαρὰν αὐτῶν ἐφευρεθεῖσαν ἀποπτύομεν καὶ
ἀποβαλλόμεθα καίτοι γε καὶ ὄνομα Θεοῦ περιέχουσαν. Οὕτως καὶ ἐν τῷ πράγματι
τῶν εἰκόνων χρὴ ἐρευνᾶν τήν τε ἀλήθειαν καὶ τὸν σκοπὸν τῶν ποιούντων καί, εἰ
μὲν ἀληθὴς καὶ ὀρθὸς καὶ πρὸς δόξαν Θεοῦ καὶ τῶν ἁγίων αὐτοῦ καὶ πρὸς ζῆλον
ἀρετῆς καὶ ἀποφυγὴν κακίας καὶ σωτηρίαν ψυχῶν γίνονται, ἀποδέχεσθαι καὶ τιμᾶν
ὡς εἰκόνας καὶ μιμήματα καὶ ὁμοιώματα καὶ βίβλους τῶν ἀγραμμάτων καὶ
προσκυνεῖν καὶ καταφιλεῖν καὶ ὀφθαλμοῖς καὶ χείλεσι καὶ καρδίᾳ ἀσπάζεσθαι ὡς
σεσαρκωμένου Θεοῦ ὁμοίωμα ἢ τῆς τούτου Μητρὸς ἢ τῶν ἁγίων τῶν κοινωνῶν τῶν
παθημάτων καὶ τῆς δόξης τοῦ Χριστοῦ καὶ νικητῶν καὶ καθαιρετῶν τοῦ διαβόλου
καὶ τῶν δαιμόνων καὶ τῆς πλάνης αὐτῶν.

\switchenglish

Note how the object of Scripture becomes 
clear to those who really search it. You must 
know, Beloved, that in every business truth and 
falsehood are distinguished, and the object of 
the doer, whether it be good or bad. In the 
gospel we find all things good and evil. God, 
the angels, man, the heavens, the earth, water 
and fire and air, the sun and moon and stars, 
light and darkness, Satan and the devils, the 
serpent and scorpions, death and hell, virtues 
and vices. And because everything told about 
them is true, and the object in view is the glory 
of God and the saints whom He has honoured, 
our salvation, and the shame of the devil, we 
worship and embrace and love these utterances, 
and receive them with our whole heart as we 
do the whole of the old and new dispensation, 
and all the spoken testimony of the holy 
fathers. Now, we reject the evil, abominable 
writings of heathens and Manicheans, and all 
other heretics, as containing foolishness and 
lies, promoting the advantage of Satan and his 
demons, and giving them pleasure, although 
they contain the name of God. So with regard 
to images we must manifest the truth, and take 
into account the intention of those who make 
them. If it be in very deed for the glory of 
God and of His saints to promote goodness, 
to avoid evil, and save souls, we should receive 
and honour and worship them as images, and 
remembrances, likenesses, and the books of the 
illiterate. We should love and embrace them 
with hand and heart as reminders of the 
incarnate God, or His Mother, or of the saints, 
the participators in the sufferings and the glory 
of Christ, the conquerors and overthrowers of 
Satan, and diabolical fraud.

\switchgreek

Εἰ δὲ Θεότητος τῆς ἀύλου καὶ ἀσωμάτου καὶ ἀοράτου καὶ ἀσχηματίστου καὶ
ἀχρωματίστου εἰκόνα τις τολμήσει ποιῆσαι, ὡς ψευδῆ ἀποβαλλόμεθα· καὶ ἐάν τις
ἐπὶ δόξῃ καὶ προσκυνήσει καὶ τιμῇ τοῦ διαβόλου ἢ τῶν δαιμόνων, καταπτύομεν καὶ
πυρὶ ἀναλίσκομεν. Καὶ ἐάν τις ἀνθρώπων ἢ πετεινῶν ἢ ἑρπετῶν ἢ ἄλλης κτίσεως
θεοποιήσῃ εἰκόνα, ἀναθεματίζομεν τοῦτον. Ὥσπερ γὰρ τὰ ἱερὰ καὶ τοὺς ναοὺς τῶν
δαιμόνων καθεῖλον οἱ ἅγιοι πατέρες καὶ ἐν τοῖς αὐτῶν τόποις ναοὺς ἐπ’ ὀνόματι
ἁγίων ἤγειραν, καὶ τούτους σέβομεν, οὕτως καὶ τὰς εἰκόνας τῶν δαιμόνων
καθεῖλον καὶ ἀντ’ ἐκείνων ἤγειραν εἰκόνας Χριστοῦ καὶ τῆς Θεοτόκου καὶ τῶν
ἁγίων. Καὶ ἐπὶ μὲν τῆς παλαιᾶς οὔτε ναοὺς ἐπ’ ὀνόματι ἀνθρώπων ἤγειρεν ὁ
Ἰσραὴλ οὔτε μνημόσυνον ἀνθρώπου ἑωρτάζετο–ἔτι γὰρ ὑπὸ κατάραν ἦν ἡ τῶν
ἀνθρώπων φύσις καὶ ὁ θάνατος κατάκρισις ἦν, διὸ καὶ ἐπενθεῖτο, καὶ τὸ σῶμα τοῦ
τεθνηκότος ἀκάθαρτον ἐλογίζετο καὶ ὁ ἁπτόμενος αὐτοῦ –, νῦν δέ, ἀφ’ οὗ ἡ
θεότης τῇ ἡμετέρᾳ φύσει συνεκράθη οἷόν τι ζωοποιὸν καὶ σωτήριον φάρμακον,
ἐδοξάσθη ἡ φύσις ἡμῶν καὶ πρὸς ἀφθαρσίαν μετεστοιχειώθη. Διὸ καὶ ὁ τῶν ἁγίων
θάνατος ἑορτάζεται καὶ ναοὶ αὐτοῖς ἐγείρονται καὶ εἰκόνες ἀναγράφονται.
Γινωσκέτω οὖν πᾶς ἄνθρωπος, ὡς ὁ τὴν εἰκόνα τὴν πρὸς δόξαν καὶ ὑπόμνησιν τοῦ
Χριστοῦ ἢ τῆς τούτου μητρὸς τῆς ἁγίας θεοτόκου ἤ τινος τῶν ἁγίων καὶ πρὸς
αἰσχύνην τοῦ διαβόλου καὶ τῆς ἥττης αὐτοῦ καὶ τῶν δαιμόνων αὐτοῦ ἐκ θείου
πόθου καὶ ζήλου γενομένην καταλύειν ἐπιχειρῶν καὶ μὴ προσκυνῶν καὶ τιμῶν καὶ
ἀσπαζόμενος ὡς εἰκόνα τιμίαν καὶ οὐχ ὡς Θεὸν ἐχθρός ἐστι τοῦ Χριστοῦ καὶ τῆς
ἁγίας Θεοτόκου καὶ τῶν ἁγίων καὶ ἐκδικητὴς τοῦ διαβόλου καὶ τῶν δαιμόνων
αὐτοῦ, ἔργῳ ἐπιδεικνύμενος τὴν λύπην, ὅτι ὁ Θεὸς καὶ οἱ ἅγιοι αὐτοῦ τιμῶνται
καὶ δοξάζονται, ὁ δὲ διάβολος καταισχύνεται· ἡ γὰρ εἰκὼν θρίαμβός ἐστι καὶ
φανέρωσις καὶ στηλογραφία εἰς μνήμην τῆς νίκης τῶν ἀριστευσάντων καὶ
διαπρεψάντων καὶ τῆς αἰσχύνης τῶν ἡττηθέντων καὶ καταβληθέντων δαιμόνων.

\switchenglish

If any one should 
dare to make an image of Almighty God, who 
is pure Spirit, invisible, uncircumscribed, we 
reject it as a falsehood. If any one make 
images for the honour and worship of the 
Devil and his angels, we abhor them and 
deliver them to the flames. Or if any one give 
divine honours to the statues of men, or birds, 
or reptiles, or any other created thing, we 
anathematise him. As our forefathers in the 
faith pulled down the temples of demons, and 
erected on the same spot churches dedicated 
to saints whom we honour, so they overturned 
the statues of demons, and set up instead the 
images of Christ, of His holy Mother, and the 
saints. Even in the old dispensation, Israel 
neither raised temples to human beings, nor 
held sacred the memory of man. At that time 
Adam s race was under a curse, and death was 
a penalty, therefore a mourning. A corpse 
was looked upon as unclean, and the man who 
touched it as contaminated. But since the 
Godhead has taken to Himself our nature, it 
has become glorified as a vivifying and efficacious
remedy, and has been transformed unto 
immortality. Thus the death of the saints is a 
rejoicing, and churches are raised to them, and 
their images are set up. Be assured that any 
one wishing to pull down an image erected out 
of pure zeal for the glory and enduring memory 
of Christ, or of His holy Mother, or any of the 
saints, to put the devil and his satellites to 
shame,\textemdash anyone, I say, refusing to honour and 
worship this image as sacred\textemdash it is not to be 
worshipped as God\textemdash is an enemy of Christ, of 
His blessed Mother, and of the saints, and is an 
advocate of the devil and his crew, showing 
grief by his conduct that the saints are honoured 
and glorified, and the devil put to shame. The 
image is a hymn of praise, a manifestation, a 
lasting token of those who have fought and con 
quered, and of demons humbled and put to flight. 

\end{paracol}

\end{document}









